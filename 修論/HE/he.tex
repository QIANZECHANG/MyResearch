
% Mohamed bin Zayed University of Artificial Intelligence (MBZUAI) Thesis Template for LaTeX 

% uOttawa (unofficial) Thesis Template for LaTeX

% Edited by Wail Gueaieb based on the uWaterloo’s Template
% The updated version of this template has been edited by Moayad Aloqaily and Mohsen Guizani.


% DON'T USE THIS TEMPLATE IF YOU DON'T KNOW WHAT YOU'RE DOING!
% Remember, it comes WITH NO WARRANTY!

% Please read the "00readme.txt" file first.
% Here is how to use this template:
%
% DON'T FORGET TO ADD YOUR OWN NAME AND TITLE in the "hyperref" package
% configuration in the "thesis-preample.tex" file. THIS INFORMATION GETS 
% EMBEDDED IN THE PDF FINAL PDF DOCUMENT.
% You can view the information if you view Properties of the PDF document.

% The template is based on the standard "book" document class which provides 
% all necessary sectioning structures and allows multi-part theses.




% N.B. The "pdftex" program allows graphics in the following formats to be
% included with the "\includegraphics" command: PNG, PDF, JPEG, TIFF
% Tip 1: Generate your figures and photos in the size you want them to appear
% in your thesis, rather than scaling them with \includegraphics options.
% Tip 2: Any drawings you do should be in scalable vector graphic formats:
% SVG, PNG, WMF, EPS and then converted to PNG or PDF, so they are scalable in
% the final PDF as well.
% Tip 3: Photographs should be cropped and compressed so as not to be too large.

% To create a PDF output that is optimized for double-sided printing: 
%
% 1) comment-out the \documentclass statement in the preamble below, and
% un-comment the second \documentclass line.
%
% 2) change the value assigned below to the boolean variable
% "PrintVersion" from "false" to "true".

% --------------------- Start of Document Preamble -----------------------

% Specify the document class, default style attributes, and page dimensions
% For hyperlinked PDF, suitable for viewing on a computer, use this:
\documentclass[letterpaper,12pt,titlepage,oneside,final,a4j,dvipdfmx]{book}
 
% For PDF, suitable for double-sided printing, change the PrintVersion variable below
% to "true" and use this \documentclass line instead of the one above:
% \documentclass[letterpaper,12pt,titlepage,openright,twoside,final]{book}


% This package allows if-then-else control structures.
\usepackage{ifthen}
\usepackage[dvipdfmx]{graphicx}
\usepackage{caption}
\usepackage{subcaption}
\usepackage{array}
\usepackage{multirow}
\usepackage{color}
\usepackage{url}


\newboolean{PrintVersion}
\setboolean{PrintVersion}{false} 
% \setboolean{PrintVersion}{true} 
% CHANGE THIS VALUE TO "true" as necessary, to improve printed results 
% for hard copies by overriding some options of the hyperref package.


% Load your needed packages and other commands of yours.
\input{thesis-preamble}

% This is where thesis margins and spaces are set.
\input{private/thesis-margins-and-spaces}
\usepackage{cleveref}
%======================================================================
%   L O G I C A L    D O C U M E N T -- the content of your thesis
%======================================================================
\begin{document}

% For a large document, it is a good idea to divide your thesis
% into several files, each one containing one chapter.
% To illustrate this idea, the "front pages" (i.e., title page,
% declaration, borrowers' page, abstract, acknowledgements,
% dedication, table of contents, list of tables, list of figures,
% nomenclature).
%----------------------------------------------------------------------
% FRONT MATERIAL
%----------------------------------------------------------------------
%
% C O V E R  P A G E
% ------------------

\input{private/frontpage}
%
%
% R E S T  O F  F R O N T  P A G E S
% ----------------------------------
%\input{private/examiners}
%\input{private/declarationpage}

% Edit the following 3 files with your abstract, acknowledgements, 
% and dedication.
% A B S T R A C T
% ---------------

\begin{center}\textbf{Abstract}\end{center}

As we all know, before the COVID-19, foreign visitors to Japan were likely to increase year after year. Given that Japan is prone to earthquakes, many surveys show that it is extremely difficult for foreigners to seek information and evacuate with appropriate behaviors during previous disasters in Japan. In addition, given the government's ongoing focus on security and safety issues in the tourism industry, it is necessary to understand foreign visitors' behaviors during disasters. To assist foreign visitors in Japan, the Japan Tourism Agency has developed an application called Safety Tips, which can notify disaster information in 14 languages.

The purpose of this study is to better understand the information-seeking and evacuation behavior of foreign visitors to Japan, as well as to explore their behavior patterns when a disaster occurs. This study also looked at how foreign visitors perceive Safety Tips and how their backgrounds influence their attitudes about them. The primary data for this study was an internet-based web survey that included demographic questions, personal experiences, and knowledge, also respondents' information seeking and evacuation behaviors in the Tokyo Metropolitan Earthquake scenarios, and finally their perception of Safety Tips. 

First, this study examined the usage experience of all respondents and discovered that Safety Tips is more popular and well-known in Indonesia, China, and Thailand than in the U.k. and Korea. Safety Tips are not used by more than 70\% of individuals who know about them or have heard about them before. We also figure out the differences among different nationalities and their different attitude based on their experience of usage. Secondly, this study used Structural Equation Modeling to investigate how personal attributes influence people's attitudes about safety tips. As a result of the findings, Disaster Prevention Consciousness has a negative impact on respondents' attitudes toward Safety Tips, while Knowledge and Perception on earthquakes has a positive impact. Regarding Training Experience, it does not have a significant impact on respondents' attitudes toward Safety Tips. What is more, this study also compared the differences between information-seeking and evacuation behaviors and showed that evacuation behaviors should be utilized more often than information-seeking actions. Evacuation behaviors have also been prioritized over information-seeking activities. Furthermore, non-face-to-face information-seeking activities should be utilized more frequently than face-to-face information-seeking behaviors. In the top three activities, following evacuation advice behaviors should be used more than self-evacuation behaviors. Finally, this study attempts to apply the findings of the study to provide Safety Tips with some acceptable recommendations for future development. 


\cleardoublepage
%\newpage

%\input{Chapters/Acknowledgements}


% No need to edit this file.
\input{private/toc-lot-lof}
%\input{Chapters/List_of_Abbreviations}

%
% No need to edit this file. But you may want to comment the whole line if you
% don't have or want a Nomenclature section.
%\input{private/list-of-symbols}  


% Change page numbering back to Arabic numerals
\pagenumbering{arabic}

%----------------------------------------------------------------------
% MAIN BODY
%---------------------------------------------------------------------- 
% Chapters 
% Include your "sub" source files here (must have extension .tex)
%%======================================================================
\chapter{Introduction}
%======================================================================

%----------------------------------------------------------------------
\section{Background}
%----------------------------------------------------------------------
As you may be aware, Japan has long been one of the most popular tourist destinations in Asia. Before the spread of Covid-19, the number of foreign visitors to Japan gradually increased. According to Japan National Tourism Organization (JNTO) statistics, the number of foreign visitors more than quadrupled between 2012 (8.36 million) and 2019 (31.9 million) that shown in Figure~\ref{fig1}. As a result, the Japan National Tourism Organization has been working hard over the years to assist foreign visitors to have a better experience in Japan. Considering that Japan is a relatively disaster-prone country, disaster prevention for foreign visitors during their stay in Japan has been an issue of great concern to the Japan National Tourism Organization.

\begin{figure*}[h]
  \includegraphics[width=\linewidth]{Figure/Figure.png}
  \centering
  \caption[Foreign visitors number in Japan by year.]{Foreign visitors number in Japan by year.\protect\footnotemark }
  \label{fig1}
\end{figure*}
\footnotetext{https://statistics.jnto.go.jp/en/graph/\#graph--inbound--travelers--transition}

After the 2018 Hokkaido Eastern Iburi earthquake, Survey Research Center, Inc. conducted a $survey [1]$ on the evacuation behavior of foreign visitors to Japan. Survey Research Center, Inc. completed this survey in collaboration with JTB Corporation. The survey was completed on September 8th (Saturday) and 9th (Sunday), 2018. The survey chose the Hokkaido Tourism Information Center on Tanukikoji Shopping Street as the survey spot, and interviews were conducted by foreign-language speaking surveyors using the survey questionnaire. The survey collected 185 valid samples from foreign tourists visiting Japan who stayed in Hokkaido on September 6, 2018, asking about their behavior during the earthquake, evacuation guidance provided by accommodation facilities, and problems encountered during the earthquake. Mainland China, Taiwan, Hong Kong, the United States, and South Korea accounted for 70\% of the respondents' nationalities. The study's findings reveal a number of significant findings, including difficulties in earthquakes, behaviors have taken after the disaster happened, and desired response in the event of an earthquake.

First, the survey result of difficulties encountered by foreign visitors during the earthquake was shown in $Figure 2 - (a)$. The inability to access information due to power outages and the inability to charge cell phones were the top-ranked difficulties. The third most difficult problem was a lack of supplies in supermarkets and convenience stores. The fourth-ranked difficulty was the schedule change caused by the earthquake. The fifth concern was not knowing where to go or what to do because of a language barrier. Lack of food/water supplies, uncertainty about the next trip, inability to understand earthquake information shown on TV, lack of information provided by transportation agencies/airports, and lack of earthquake evacuation manuals for foreigners that make it difficult to know what to do were the sixth to tenth-ranked difficulties. The lack of multilingual disaster/transportation/evacuation information in cell phones, the lack of evacuation instructions in hotels, the lack of information about the earthquake in Japan, the lack of information about what to bring to evacuate, and the lack of information from medical institutions were the eleventh to fifteenth difficulties. According to the results, the most common difficulties during the earthquake, were related to power outages, such as "power outages made it difficult to get information" and "power outages made it difficult to charge smartphones, etc" ( 67.0\%). Due to unforeseen circumstances caused by the earthquake, the response of "lack of supplies at convenience stores and supermarkets”  (46.5\%) was also common. Respondents were concerned about modifications to their itinerary as a result of transportation disruptions, such as "all my itineraries were disrupted and I had to pay a lot of money" (37.3\%) and "I couldn't predict what would happen to my itinerary in the future"(27.0\%). Another common difficulty was related to language issues like "I didn't know where to go because I didn't understand the language" (29.2\%).

Second, $Figure 2 - (b)$ shows the survey results of behaviors that occurred after the earthquake. Following the earthquake, the top three common actions were "tried to get information via the Internet or SNS" (49.7\%), "stayed where they were and checked on the situation" (44.3\%), and "kept in touch with family and friends via the Internet, e-mail, and SNS such as Facebook and Line" (39.5\%). Calling family/friends (31.9\%), getting information about the earthquake from TV or radio (31.4\%), contacting the hotel front desk (27\%), and contacting fellow travelers (20\%) was the fourth to seventh popular responses. So, based on the survey results, we can conclude that after the earthquake, people prefer to stay in the area to look into the matter while gathering information and confirming their safety via the Internet and social media. In particular, we can discover that there are two main ways for respondents to gather information. The first is face-to-face information-seeking behaviors, such as asking people around, hotel staff, and so on. The other type of information-seeking behavior is no-face-to-face information-seeking behaviors. The other is no-face-to-face information-seeking behaviors, which primarily rely on television/radio/social media/internet. We will also divide people's information-seeking behaviors into these two types in the follow-up study to see whether people's behavioral patterns are more inclined to contact people or not.

Third, as shown in $Figure 2 - (c)$, popular desired responses in the event of an earthquake were "Provide charging points, etc." (50.8\%) and "Enhance information centers" (42.2\%), followed by "Distribute manuals in native languages" (38.4\%). And the next most common responses can be roughly divided into three categories. The first is a requirement for multilingual services, such as "want evacuation guidance in a language I understand" (36.2\%) or "want disaster/traffic/evacuation information to be provided in multiple languages via smart phones, etc." (35.1\%), "Would like TVs and other media to display information in English" (30.8\%), "Would like information signs in my native language" (24.9\%). Another requirement was for a place of evacuation, such as "providing places to stay and other accommodations"(34.6\%), "Would like the hotel where I was staying to serve as a disaster information hub" (22.2\%). The last category was for providing information, such as "Would like information centers to be set up to provide information on transportation and flights" (25.4\%), "Provide telephone consultation services"(15.1\%), and "Would like pamphlets and other materials that show what to do after an evacuation"(14.1\%), and "Wish to learn more about medical institutions" (9.2\%).

Combined with the previous findings in the survey, it is clear that there is a need to provide sufficient places for foreign visitors to recharge in order to ensure that they can contact their family/friends and gather the necessary disaster information from social media/networks. The following step is to provide information in their language as well as evacuation assistance.

Considering the Japan Tourism Agency has constantly concern with issues of security and safety in the tourism industry. So, under the supervision of the Japan Tourism Agency, R.C. Solutions, Inc. developed a free application called Safety tips, which can notify foreign visitors of earthquake early warnings, tsunami warnings, eruption alerts, special warnings, heatstroke information, national protection information, evacuation advisories, and other disasters that occurred in Japan. $Figure 3$ shows the Safety Tips interface. During disasters, safety tips can provide a variety of purposes for foreign visitors to Japan. It is available in 14 languages (15 languages), including Japanese, English, Chinese (traditional and simplified), Korean, Spanish, Portuguese, Vietnamese, Thai, Indonesian, Tagalog, Nepali, Khmer, Burmese, and Mongolian (as shown in $Figure 4$). Safety Tips is an important part of this study. The study will compare the difference in attitudes toward Safety Tips among respondents of various nationalities, as well as the differences in attitudes toward Safety Tips among people from various upbringing backgrounds.

\section{Problem Identification}
\section{Research Goal and Objective}
\section{Hypothesis}
\section{Significance of the study}
\section{Thesis Structure}


%% Some LaTeX commands I define for my own nomenclature.
% If you have to, it's better to change nomenclature once here than in a 
% million places throughout your thesis!


%======================================================================
\chapter{Literature Review}
\label{c2}
%======================================================================

As people's awareness of disaster prevention has increased in recent years, the number of disaster prevention and evacuation studies has increased year after year. On the other hand, using statistical knowledge to analyze some sociological issues is also one of the popular research directions. This research is a hybrid of the two domains mentioned above. This study focuses on people's evacuation behavior in terms of disasters, to promote a link between the respondents' demographic information elements, prior experiences with travel/evacuation dramas/disasters, and their behavior patterns in disasters. On the other side, the researchers measured how the above characteristics influenced their perceptions of the disaster prevention application called Safety Tips. Therefore, the analysis methods of many sociology-oriented studies for questionnaires, especially those containing scale-based questions, were also referred to. 

After the 2018 Hokkaido Eastern Iburi earthquake, Survey Research Center, Inc. Center, Inc. completed this survey conducted a survey~\cite{ref50} on the evacuation behavior of foreign visitors to Japan. The survey was completed in September 2018,  interviewed 185 foreign tourists visiting Japan who stayed in Hokkaido on September 6, 2018, asking about their behavior during the earthquake, evacuation guidance provided by accommodation facilities, and problems encountered during the earthquake. The survey shows that the inability to access information and charge cell phones due to power outages were the biggest difficulties for foreign visitors. Also, the language barrier makes it difficult to know what to do and to understand the information available locally. Combined with the lack of direct evacuation information available on smartphones and the lack of evacuation guidelines in hotels, these caused significant difficulties for foreign visitors to evacuate. This is also mentioned in Henry, M., (2014)~\cite{ref55} that foreign visitors become a vulnerable group of people during the disaster that happened in Japan. 

Choi, et al. (2021)~\cite{ref54} mention that based on people's past experiences in shelters, the things that most affect the satisfaction of the experience are Internet access and privacy space. The main needs of people for shelters are also the availability of necessities, availability of information, and personal private space. In addition, Japanese and foreign visitors can coexist in the shelters, and they do not show a high degree of exclusivity.

Kawasaki, et al. (2013)~\cite{ref52} verified that the language ability of foreign visitors affects their information-gathering channels. Foreign visitors who understand Japanese will always look for information from Japanese and English media, but foreign visitors who only speak English will have fewer media channels to rely on. This is also mentioned in Kawasaki (2012)~\cite{ref48}, where foreign visitors often encounter more language difficulties in collecting disaster information. For those foreign visitors who could not rely on local intelligence sources due to language barriers, turning to overseas sources was their only recourse. However, when the information from overseas sources is inconsistent with local intelligence, it is easy to create panic in their hearts. 



However, after the earthquake, people prefer to stay in the area to look into the matter while gathering information and confirming their safety via the Internet and social media~\cite{ref50}. In particular, we can discover that there are two main ways for respondents to gather information. The first is face-to-face information-seeking behaviors, such as asking people around, hotel staff, and so on. The other is no-face-to-face information-seeking behaviors, which primarily rely on television/radio/social media/internet. We will also divide people's information-seeking behaviors into these two types in the follow-up study to see whether people's behavioral patterns are more inclined to contact people or not. 

Furthermore, Wang (2021)~\cite{ref9} focuses on the relationship between people's evacuation behaviors and a variety of factors. To establish the hypotheses used in this study, it is important to refer to the results of previous studies on the relationship between people's selected behavior in evacuation and various factors. Many previous papers have addressed the subject of whether and how various factors affect evacuation behavior in evacuation, as stated in this study. The authors present a clearer framework in this research to explain how factors related to evacuation behavior influence people's decision to evacuate. Personal, environmental, and intervention variables are the three types of factors. Details will be discussed in Chapter~\ref{c5} step2. 

Sato (2020)~\cite{ref8} provides an overview of how visitors to Japan obtain information about natural disasters from tourist guides and other sources. Furthermore, it investigates how foreign visitors to Japan are effectively provided with information about natural disasters in Japan through the behavior of foreign tourists and the responses of government agencies and other administrative bodies during the 2016 Kumamoto earthquake and the 2018 Hokkaido earthquake. While tourists can get some information on disaster preparedness from guidebooks and other sources, the paper mentions that foreign visitors who have never experienced a real earthquake in the past may feel anxious when a disaster occurs in an unfamiliar country. One point raised in the paper that had previously gone unnoticed was that, while the priority in a disaster is to prevent death or injury, the post-disaster care needs of foreign visitors differ from those of Japanese citizens. The most immediate post-disaster needs of most foreign visitors in the Kumamoto and Hokkaido earthquakes were a desire to leave the disaster area as soon as possible and an attempt to gather transportation information. This is also a revelation for this study, which is why the analysis was conducted to investigate the differences between Japanese and foreign visitors. Because of the differences in backgrounds, developing Safety Tips based on the habits and habits of Japanese people may not be effective in assisting foreign visitors. As a result of this thesis, we have made it a priority to investigate the differences in evacuation behavior between Japanese and foreign visitors.

In addition, people need more accurate, timely, and transparent information dissemination during a crisis~\cite{ref49}. Leelawat (2017)~\cite{ref51} also highlights the importance of information dissemination by outlining the evacuation process. The combination of information and communication technology plays a vital role in evacuation, which is one of the most effective ways to disseminate information. Together with modern communication technology, this can make the delivery of rescue requests more time-sensitive and can also provide an efficient communication platform in times of emergency. The lessons learned from the Kumamoto earthquake for Thai citizens can be applied in future disaster relief programs. 

 Sato (2020)~\cite{ref8}  investigates how to provide timely and accurate information to foreign visitors to Japan in the immediate aftermath of a natural disaster, despite language barriers. The authors make three points: 1) it is critical to provide information to foreign visitors in the event of a disaster; 2) there is a need to propose a method that makes it easier to provide information to foreign visitors across language barriers; and 3) because foreign visitors' reactions to disasters are influenced by previous disaster experiences, there is a need to provide a detailed description of the disaster in the context of the visitors' own culture. Furthermore, because Japanese maps, kanji, addresses, and other symbols were difficult for foreign visitors to understand, it was difficult for foreign visitors to locate evacuation centers simply by translating the maps into English. Furthermore, even if they find a structure that appears to be an evacuation shelter, it is difficult for them to determine whether the structure is a safe place to evacuate. As a result, national languages and easily recognizable signs were required to indicate that the building was an 'evacuation center.' This would not only confirm that the shelter was open to foreigners but would also assist the Japanese in recognizing it as a location where foreigners could flee. These suggestions made by the authors are significant for the present study. This is because this study will eventually use the results of the analysis to make some recommendations that will be beneficial to the development of Safety Tips. Several of the ideas mentioned in the paper can be further refined to be reflected in Safety Tips in order to better assist foreign visitors.

In order to foster the formation of disaster awareness, many countries actually hold some educational activities and disaster drills. As we all know, due to the many natural disasters in Japan, disaster prevention drills are regularly held inside Japanese schools, which helps to develop disaster awareness among children. Other countries have also started to strengthen this awareness in recent years. Ozeki et al., (2018)~\cite{ref45} explored respondents' disaster awareness through a scale questionnaire and then summarized 24 elements that can represent disaster awareness in three dimensions. The three dimensions were knowledge, behavior, and psychology. Therefore, in this study, five elements were also selected to represent the variable of disaster awareness.

Furthermore, a person's age, income, and education level were found to be influenced in the dissemination of refuge information by Kawasaki (2013)~\cite{ref53}. Disadvantaged groups in terms of income and education level will have a relatively tough time collecting information on disaster information mountain. This is a factor that should be measured in the development of Safety Tips.
%% Some LaTeX commands I define for my own nomenclature.
% If you have to, it's better to change nomenclature once here than in a 
% million places throughout your thesis!


%======================================================================
\chapter{Survey and Data}
%======================================================================

\section{Survey Introduction}
This research is based on an Internet-based survey of Foreign visitors and Japanese who have visited Tokyo, conducted by the Economic and Social Research Institute. A total of 1800 people were surveyed, including 300 people from each country. The detailed information is shown in Table~\ref{table1}. 

%%%%%%%%%%%%%%%%%
%\iffalse
\begin{table}[h]
  \caption{Survey descriptions}
  \label{table1}
  \centering
  \begin{tabular}{|c|l|}
  \hline
  Respondent &  Foreign visitors and Japanese who have visited Tokyo \\
  \hline
  Nationality  &  China, Korea, Thailand, Indonesia, UK, Japan \\
  \hline
  Method       &  Internet-based web survey \\
  \hline
  \begin{tabular}{c}Items\end{tabular}          &  
\begin{tabular}{c}
\begin{minipage}[t]{0.76\textwidth}
  \begin{itemize}
      \setlength{\itemsep}{-0.1cm} 
      \item[\textbf{1.}]  Demographics  
      \item[\textbf{2.}]  Disaster prevention consciousness 
      \item[\textbf{3.}]  Disaster response education, experience on earthquakes
      \item[\textbf{4.}]  Knowledge and perception on earthquakes 
      \item[\textbf{5.}]  Information seeking and evacuation behavior in scenarios
      \item[\textbf{6.}]  Perception on Safety Tips
   \end{itemize} 
\end{minipage}
\end{tabular}\\
   \hline
   Samples   &  300 samples/country (Total: 1,800) \\ 
   \hline
   Survey period &  October 2019 (7 days) \\
   \hline
  \end{tabular}
\end{table}
%\fi

\section{Question description}
The survey was divided into 6 items, the detailed description and available answers of each question in \crefrange{table2item1}{table2item6}. 
The first item, FQ1-FQ7, consists of seven demographic questions, including nationality, gender, age, travel experience, and Japanese proficiency. The second to fourth items are Q1-Q10, which includes disaster consciousness, disaster training experience, earthquake experience, earthquake knowledge, and disaster response knowledge. The fifth item is Q11-Q14, which is about respondents' response actions in four scenarios during the "Tokyo Metropolitan Earthquake." ( I'll go over scenarios in 3.3). The sixth item is Q15-Q17, which is about respondents' attitudes toward Safety Tips, including respondents' usage experiences as well as their perceptions of Safety Tips.

%%%%%%%%%%%%%%%%%%%%%%%%
%\iffalse
\begin{table}[h]
  \caption[Questions descriptions of Item 1]{Questions descriptions of Item 1 (FQ1 to FQ7)}
  \label{table2item1}
  \centering
  \begin{tabular}{c|c|l}
    No.      & Question & \multicolumn{1}{|c}{Description}  \\
  \hline
    FQ1     & Country &  \begin{tabular}{l}Answer 1 to 6 as Japan, China, South Korea, Thailand,\\Indonesia, the UK \end{tabular} \\
  \hline
    FQ2     & Gender  &  \begin{tabular}{l}Answer 1 as Male, 2 as Female. \end{tabular} \\
  \hline
    FQ3     & Age &  \begin{tabular}{l}Answer 1 to 8 as age under 15, age 16-19, age 20-29, age\\30-39, age 40-49, age 50-59, age 60-69, age over 70. \end{tabular} \\
  \hline
    FQ4     & Visited Country &  \begin{tabular}{l}The total number of visits to the following countries/regions\\in the past year, including Japan, China Mainland, China\\Hong Kong, China Macau, Korea, Thailand, Malaysia,\\Singapore, Indonesia, India. The answer is presented as the\\total number of visits, or 0 if no visit was recorded. \end{tabular} \\
  \hline
    FQ5     & Visited Japan &  \begin{tabular}{l}The total number of visits to each city in Japan in the past\\year, including Hokkaido, Chiba, Tokyo, Yokohama, Nagoya,\\Kyoto, Osaka, Nara, Hakata, Okinawa. The answer is\\presented as the total number of visits and 0 for no visits. \end{tabular} \\
  \hline
    FQ6     & Visit Experience &  \begin{tabular}{l}This question asks for the number of visits to Tokyo (for\\Japanese) / Number of visits to Japan (for overseas).\\Answer 1 to 6 as 1 time, 2 times, 3 to 4 times, 5 to 6\\times, 7 to 9 times, over 10 times, and 0 for no visits. \end{tabular} \\
  \hline
    FQ7     &  Japanese Level &  \begin{tabular}{l}Answer 1 to 4 as Japanese Level: Cannot understand,\\Basic, Intermediate, Up Level. \end{tabular} \\
   \hline

  \end{tabular}
\end{table}

\begin{table}[h]
  \caption[Questions descriptions of Item 2]{Questions descriptions of Item 2 (Q1 to Q5): Answer based on the scale of 0 to 6, detailed as not at all applicable, mostly not applicable, somewhat not applicable, somewhat agree, mostly agree, strongly agree.)}
  \label{table2item2}
  \centering
  \begin{tabular}{c|l}
    No.      & \multicolumn{1}{|c}{Description}  \\
  \hline
    \textbf{Q1}       & \textbf{Disastrous Imagination} \\
  \hline  
    Q1\_1  & \begin{tabular}{l}Can imagine what people around me will do in the event of a disaster.\end{tabular} \\
    Q1\_2  & \begin{tabular}{l}Can imagine the supplies I will need in the event of a disaster.\end{tabular} \\
    Q1\_3  & \begin{tabular}{l}Can imagine what you would do in the event of a disaster.\end{tabular} \\
    Q1\_4  & \begin{tabular}{l}Can imagine what kind of damage the city would suffer in the event of\\a disaster.\end{tabular} \\
  \hline
    \textbf{Q2}       & \textbf{Sense of crisis} \\
  \hline
    Q2\_1  & \begin{tabular}{l}A disaster could happen tomorrow.\end{tabular} \\
    Q2\_2  & \begin{tabular}{l}Once a disaster strikes, I will be in trouble.\end{tabular} \\
    Q2\_3  & \begin{tabular}{l}It is difficult to reduce the damage caused by disasters through personal\\efforts alone.\end{tabular} \\
    Q2\_4  & \begin{tabular}{l}Disaster prevention should not be completed only in my own area, but\\also in cooperation with other areas. \end{tabular} \\
  \hline
    \textbf{Q3} & \textbf{Other-directed type} \\
  \hline
   Q3\_1   & \begin{tabular}{l}Like to communicate with others.\end{tabular} \\
   Q3\_2   & \begin{tabular}{l}Like places where people gather.\end{tabular} \\
   Q3\_3   & \begin{tabular}{l}Want to make many different kinds of friends.\end{tabular} \\
   Q3\_4   & \begin{tabular}{l}Want to do something for other people. \end{tabular}\\
  \hline
   \textbf{Q4} & \textbf{Anxiety} \\
  \hline
   Q4\_1  & \begin{tabular}{l}Often feel anxious.\end{tabular} \\
   Q4\_2  & \begin{tabular}{l}I think I am a worrier.\end{tabular} \\
   Q4\_3  & \begin{tabular}{l}When I start thinking about disasters, I fantasize about different patterns\\of damage.\end{tabular} \\
   Q4\_4  & \begin{tabular}{l}Always worried about the dangers around me.\end{tabular} \\
  \hline
   \textbf{Q5} & \textbf{Apathy about disasters} \\
  \hline
   Q5\_1  & \begin{tabular}{l}Don't want to do anything that is not in my best interest.\end{tabular} \\
   Q5\_2  & \begin{tabular}{l}Only think about things that are likely to happen in my immediate\\surroundings.\end{tabular} \\
   Q5\_3  & \begin{tabular}{l}Don't usually think about disasters.\end{tabular} \\
   Q5\_4  &  \begin{tabular}{l}Physical measures such as reinforcing earthquake-proof buildings and\\building breakwaters are enough to prevent disasters.\end{tabular} \\
   \hline

  \end{tabular}
\end{table}

\begin{table}[h]
  \caption[Questions descriptions of Item 3]{Questions descriptions of Item 3 (Q6 to Q8)}
  \label{table2item3}
  \centering
  \begin{tabular}{c|l}
    No.      & \multicolumn{1}{|c}{Description}  \\
  \hline
    \textbf{Q6} & \begin{tabular}{l}\textbf{Disaster training experience. (1 for yes, 0 for no)}\\\textbf{(12 questions for each of the 4 disasters include}\\\textbf{Q6\_1Earthquake, Q6\_2tsunami, Q6\_3typhoon, Q6\_4fire)}\end{tabular} \\
  \hline  
    (1)  & \begin{tabular}{l}Have you participated in drills (evacuation drills, disaster drills, etc.)?\end{tabular} \\
    (2)  & \begin{tabular}{l}Have received training at school or participated in a seminar.\end{tabular} \\
    (3)  & \begin{tabular}{l}Have received training at work or participated in a training session.\end{tabular} \\
    (4)  & \begin{tabular}{l}Have received training by the government or local government,\\or participated in a training session\end{tabular} \\
    (5)  & \begin{tabular}{l}Have received training or participated in training sessions by local\\community/community association, etc.\end{tabular} \\
    (6)  & \begin{tabular}{l}Have received education or participated in a training session by\\a private organization or group.\end{tabular} \\
    (7)  & \begin{tabular}{l}Have received training or participated in a workshop other than\\the above.\end{tabular} \\
    (8)  & \begin{tabular}{l}Have seen or heard disaster information on TV or radio programs.\end{tabular} \\
    (9)  & \begin{tabular}{l}I have seen disaster information in newspapers or magazines.\end{tabular} \\
    (10)  & \begin{tabular}{l}Have seen disaster information on bulletin boards, etc.\end{tabular} \\
    (11)  & \begin{tabular}{l}Have seen disaster information on the Internet (disaster\\prevention-related websites of government or public organizations)\end{tabular} \\
    (12)  & \begin{tabular}{l}Have you ever seen disaster information on the Internet (disaster-related\\websites of private organizations or local communities)?\end{tabular} \\
  \hline
    \textbf{Q7}       &  \begin{tabular}{l}\textbf{Number of disaster training (Answer 1 to 4 as one time,}\\\textbf{2 to 3 times, 4 to 6 times, over 7 times.)} \end{tabular} \\
  \hline
   Q7\_1   & \begin{tabular}{l}Earthquake\end{tabular} \\
   Q7\_2   & \begin{tabular}{l}Tsunami \end{tabular} \\
   Q7\_3   & \begin{tabular}{l}Typhoon\end{tabular} \\
   Q7\_4   & \begin{tabular}{l}Fire\end{tabular}\\
  \hline
   \textbf{Q8} & \begin{tabular}{l}\textbf{Earthquake experience (Q8 is about the severity of the}\\\textbf{earthquake experienced. Answer 1 to 8 as MMI intensity}\\\textbf{5 or less / intensity 3 or less; MMI intensity 6 / intensity}\\\textbf{4; MMI intensity 7 / intensity 5 weak; MMI intensity}\\\textbf{8 / intensity 5 strong; MMI intensity 9 / intensity 6 weak;}\\\textbf{MMI intensity 10 / intensity 6 strong; MMI intensity 11}\\\textbf{to 12 / intensity 7; no experience)}\end{tabular} \\
   \hline
  \end{tabular}
\end{table}

\begin{table}[h]
  \caption[Questions descriptions of Item 4]{Questions descriptions of Item 4 (Q9 and Q10): Answer based on the scale of 0 to 6, detailed as don't know anything about it, almost don't know, somewhat unfamiliar, somewhat familiar, know almost about it, know very much.}
  \label{table2item4}
  \centering
  \begin{tabular}{c|l}
    No.      & \multicolumn{1}{|c}{Description}  \\
  \hline
    \textbf{Q9} & \textbf{Knowledge about earthquakes} \\
  \hline  
    (1)  & \begin{tabular}{l}Magnitude is the strength of the tremor and differs from place to place.\end{tabular} \\
    (2)  & \begin{tabular}{l}Magnitude is the size of an earthquake.\end{tabular} \\
    (3)  & \begin{tabular}{l}A small increase in magnitude can result in an unimaginably large\\earthquake.\end{tabular} \\
    (4)  & \begin{tabular}{l}The magnitude of an earthquake and the amount of damage caused by an\\earthquake are affected not only by the size of the earthquake but also by\\the distance from the epicenter and the characteristics of the ground.\end{tabular} \\
    (5)  & \begin{tabular}{l}The method of measurement is different between the Japanese seismic\\intensity and the global seismic intensity (Mercari seismic intensity scale).\end{tabular} \\
    (6)  & \begin{tabular}{l}In Japan, seismic intensity information is important when considering\\damage caused by earthquakes.\end{tabular} \\
  \hline
    \textbf{Q10}       & \textbf{Knowledge of how to respond to a disaster}  \\
  \hline
    (1)  & \begin{tabular}{l}Protect your head, move away from large furniture, and hide under a\\sturdy desk.\end{tabular} \\
    (2)  & \begin{tabular}{l}Do not run outdoors in a panic.\end{tabular} \\
    (3)  & \begin{tabular}{l}Open doors and windows to create an escape route.\end{tabular} \\
    (4)  & \begin{tabular}{l}Follow the instructions of the staff.\end{tabular} \\
    (5)  & \begin{tabular}{l}Do not rush to the exits or stairs.\end{tabular} \\
    (6)  & \begin{tabular}{l}Stop at the nearest floor and get off immediately.\end{tabular} \\
    (7)  & \begin{tabular}{l}Stay away from block walls, vending machines, buildings, etc.\end{tabular} \\
    (8)  & \begin{tabular}{l}Protect your head and move with caution to avoid falling signs, broken\\windows, etc.\end{tabular} \\
    (9)  & \begin{tabular}{l}Check your surroundings carefully and act calmly.\end{tabular} \\
   \hline
  \end{tabular}
\end{table}


\begin{table}[h]
  \caption[Questions descriptions of Item 6]{Questions descriptions of Item 6 (Q15 to Q17): About Safety Tips}
  \label{table2item6}
  \centering
  \begin{tabular}{c|l}
    No.      & \multicolumn{1}{|c}{Description}  \\
  \hline
     \        & \begin{tabular}{l}\textbf{Usage experience}\end{tabular}  \\
  \hline
   Q15      & \begin{tabular}{l}Do you know Safety tips or not? (Answer 1 to 3 as don't know, only heard\\the name before, know exactly.) (Answer 2 and 3 goes to Q16, Answer 1\\directly goes to Q17.)\end{tabular} \\
   Q16      & \begin{tabular}{l}Did you use Safety tips before or not? (Answer 1 is never used before, 2 is\\used before.) \end{tabular} \\
  \hline
   \textbf{Q17} &\begin{tabular}{l}\textbf{Attitude toward Safety Tips. (Answer based on the scale of 0 to 6,}\\\textbf{detailed as not at all applicable, mostly not applicable, somewhat}\\\textbf{not applicable, somewhat agree, mostly agree, strongly agree.)} \end{tabular}  \\
  \hline
   Q17\_1  & \begin{tabular}{l}Will you trust Safety tips more than information from your own country?\end{tabular} \\
   Q17\_2 & \begin{tabular}{l}Will you use Safety tips before searching for information from your own\\country?\end{tabular}\\
   Q17\_3  & \begin{tabular}{l}Do you think Safety tips could be useful during evacuation?\end{tabular} \\
   Q17\_4 & \begin{tabular}{l}Will you use Safety tips in the future?\end{tabular}\\
  \hline
  \end{tabular}
\end{table}
%\fi

\section{Scenario description}
In item No.5, all respondents should answer their Information seeking and evacuation behavior in each of the following scenarios. There are two types of differences between the scenarios, resulting in four different scenarios. The first type of difference is network-related and is divided into "Telephone/internet is available" and "Telephone/internet is not available (A temporary power outage occurs)". The second type of difference is location-related, divided into "Staying in a tourist attraction" and "Moving by public transportation". Thus the 4 scenarios are shown in Table~\ref{table3}.


%%%%%%%%%%%%%%%%%%%%%
%\iffalse
\begin{table}[h]
  \caption{Scenarios descriptions}
  \label{table3}
  \centering
  \begin{tabular}{l|c|c}
               &  \begin{tabular}{c}  Staying in a\\tourist attraction \end{tabular} & \begin{tabular}{c}  Moving by public\\transportation \end{tabular}  \\
    \hline
    \begin{tabular}{c}Telephone / Internet is available\end{tabular}  & Scenario 1 & Scenario 3 \\
    \hline
    \begin{tabular}{c} Telephone / Internet is not available\\(A temporary power outage occurs)\end{tabular} & Scenario 2 & Scenario 4 \\
 
  \end{tabular}
\end{table}
%\fi


In order to answer response actions during the disaster, all respondents were requested to watch a simulation video of the "Tokyo Metropolitan Earthquake". The video is provided by the Cabinet Office,  and the simulation video has been translated into their native languages. The simulation video shows an earthquake of magnitude 7.3 that occurred in the southern part of Tokyo, happened on a winter evening in the year 20xx, and in order to show the damage, the landscape is represented brighter than it actually is. After watching the video, the respondents were requested to select 5 response actions with an order in each of the scenarios. Some screenshots of the simulation video are shown in Figure~\ref{fig5}.

In the pre-survey, we collected some common evacuation behaviors of foreign visitors and Japanese people. It's worth noting that selections 1-7 are only available in scenarios 1 and 3 because they require a phone/internet connection. Table~\ref{table4} contains a list of all selections as well as a detailed description of each one.

%%%%%%%%%%%%%%%%%%%%%
%\iffalse
\begin{table}[h]
  \caption{Available selections of Behaviors}
  \label{table4}
  \centering
  \begin{tabular}{c|l}
  \hline
  Selection & Actions - Information seeking behavior \\
  \hline
  1             & \begin{tabular}{l}Collect Information on the official websites of Japanese government agencies\\(Japan Meteorological Agency, National Police Agency, Fire and Disaster\\Management Agency, etc.)\end{tabular}\\
  2             & Collect Information with the disaster prevention app on your smartphone \\
  3             & Collect Information on news sites and disaster prevention portal sites \\
  4             & Collect Information on SNS (Twitter, Facebook, LINE, etc.) \\
  5             & Call the embassy of your country to collect Information \\
  6             & Collect Information from TV and radio \\
  7             & Check maps and digital signage to collect Information \\
  8             & Gather Information by calling out to Japanese people nearby \\
  9             & Contact staff at tourist Information centers to collect Information \\
  10           & Contact the hotel staff to collect Information \\
  11           & Contact public transport staff to collect Information \\
 \hline
 Selection & Actions - Evacuation behavior \\
 \hline
 12            & Stay at your current location \\
 13            & Secure necessary supplies (food, drinks, etc.) \\
 14            & Move to an open space, such as a nearby park \\
 15            & Move according to evacuation guidance \\
 16            & Move to the evacuation center on your own \\
 17            & Move in sync with the movements of people around you \\
 18            & Observe the surroundings because you don't know what to do \\
 \hline
 Selection  & Actions - others \\
 \hline
 19            & Others \\
 20            & Do nothing / Can do nothing \\
 21            & I don't know \\
 \hline
  \end{tabular}
\end{table}

\begin{figure*}[h]
  \begin{subfigure}{0.32\textwidth}
    \centering
    \includegraphics[width=\textwidth]{Figure/Figure5a.jpg}
    \caption{Difficulty in returning home}
    \label{fig5a}
  \end{subfigure}\hfill
  \begin{subfigure}{0.32\textwidth}
    \centering
    \includegraphics[width=\linewidth]{Figure/Figure5b.jpg}
    \caption{Dense area of wooden houses; Fire occurred easily}
    \label{fig5b}
  \end{subfigure}\hfill
  \begin{subfigure}{0.32\textwidth}
    \centering
    \includegraphics[width=\linewidth]{Figure/Figure5c.jpg}
    \caption{Damage to transportation facilities in many places}
    \label{fig5c}
  \end{subfigure}\hfill
  \begin{subfigure}{0.32\textwidth}
    \centering
    \includegraphics[width=\linewidth]{Figure/Figure5d.jpg}
    \caption{Tokyo-Wooden houses (Dense area); Seismic intensity 7}
    \label{fig5d}
  \end{subfigure}\hfill
  \begin{subfigure}{0.32\textwidth}
    \centering
    \includegraphics[width=\linewidth]{Figure/Figure5e.jpg}
    \caption{Tokyo-Bay Area; Seismic intensity 7}
    \label{fig5e}
  \end{subfigure}\hfill
  \begin{subfigure}{0.32\textwidth}
    \centering
    \includegraphics[width=\linewidth]{Figure/Figure5f.jpg}
    \caption{Tokyo; Seismic intensity 6-}
    \label{fig5f}
  \end{subfigure}
  \begin{subfigure}{0.32\textwidth}
    \centering
    \includegraphics[width=\linewidth]{Figure/Figure5g.jpg}
    \caption{Tokyo-Mountainous region; Seismic intensity 6+}
    \label{fig5g}
  \end{subfigure}
  \centering
  \caption[Simulation video of  "Tokyo Metropolitan Earthquake".]{Simulation video of "Tokyo Metropolitan Earthquake".\protect\footnotemark }
  \label{fig5}
\end{figure*}
 \footnotetext{http://wwwc.cao.go.jp/lib\_012/syuto\_02.html}
%\fi



%% Some LaTeX commands I define for my own nomenclature.
% If you have to, it's better to change nomenclature once here than in a 
% million places throughout your thesis!



%======================================================================
\chapter{Methodology }
%======================================================================

\section{Methodology for achieving Objective - 1 }

\textbf{Objective 1: To understand foreign visitors' attitudes toward Safety Tips.}

For research objective 1, the nationality data in item 1 and item 6 were used, containing two tasks. While considering that the main users of Safety Tips are foreign visitors, only the sample of foreign respondents was selected for this part of the analysis, and the sample of Japanese respondents was excluded in this part. So the total number of samples was 1500.

The first task is to summarize the questionnaire's results from Q15 to Q17. The findings will show the popularity among foreign visitors, past usage experience of foreign respondents, and foreign respondents' attitude toward Safety Tips in each country, as well as the differences among respondents from the following five countries which are China, South Korea, Indonesia, Thailand, and the UK.

The second task was based on the answers to Q15 and Q16. As Q15 was 'do you know Safety tips or not?', and Q16 was 'Did you use safety tips before or not?', these two questions can clarify respondents' past usage of Safety Tips. The answers to Q15 had three options: Know exactly, Heard safety tips before, and Do not know. If the respondent answered 'Know exactly' or 'Heard Safety tips before', the respondent will continue to answer Q16, if the respondent answers 'Do not know', the respondent will directly skip to Q17. The answers to Q16 are 'used Safety tips before.' and 'never used it before. Therefore, based on the answers to these two questions, all respondents were divided into the following five groups: 'Know exactly and used Safety tips before', 'Know exactly but never used Safety tips before. before', 'Heard Safety tips before and used it before, 'Heard Safety tips before but never used it before, and 'Do not know and never used before. The sample sizes for each group are shown in Figure~\ref{fig6}, which are 'Know exactly and used Safety tips before': 357; 'Know exactly but never used Safety tips before': 90; 'Heard Safety tips before and used it before: 134; 'Heard Safety tips before but never used it before': 465 people; 'Do not know and never used before': 454 people. The results will show whether the two factors of respondents' past awareness and whether they used it before had an impact on their attitudes toward Safety Tips.


%%%%%%%%%%%%%%%%%%%%%%%
%\iffalse
\begin{figure*}[h]
  \includegraphics[width=\linewidth]{Figure/Figure6.jpg}
  \centering
  \caption{Number of respondents in each group. }
  \label{fig6}
\end{figure*}
%\fi




\section{Methodology for achieving Objective - 2 }
\textbf{Objective 2: To explore how respondents' attitudes toward Safety Tips are influenced by their characteristics. }

For research purpose 2, the data of Item 1, Item 2-4, and Item 6 were used. This part will continue to look into the impact of foreign respondents' personality characteristics on their attitude toward Safety Tips. As this part continues to focus on foreign visitors, and considering the attitudes of the respondents who have used this application about Safety Tips is even more telling, so for the Structural Equation Modeling, we only selected the sample data that give 'used before' answer in Q16. Therefore, the data sample amount is 491. Based on the discussion of sample size, the following suggestions for minimum sample sizes are offered based on the model complexity and basic measurement model characteristics~\cite{ref15}:

\begin{itemize}
\item Minimum sample size--100: Models containing five or fewer constructs, each with more than three items (observed variables) and with high item communality (.6 or higher).
\item Minimum sample size-150: Models with seven constructs or less, modest communalities(.5), and no under-identified constructs.
\item Minimum sample size-300: Models with seven or fewer constructs, lower communality(below .45), and/or multiple under-identified (fewer than three) constructs.
\item Minimum sample size-500: Models with large numbers of constructs, some with lower communality, and/or having fewer than three measured items.
\end{itemize}
The current sample data amount is available in this study. 

\textbf{Structural Equation Modeling.} Structural Equation Modeling will be used to analyze this part. Structural Equation Modeling is a statistical method based on Regression Models for Latent Variables, and SEM is a multiple regression model that allows us to test the theoretical model by testing the hypothesis to better understand the variables in our hypothesis with a clearer causal relationship between them~\cite{ref13}. Structural Equation Modeling can consider and deal with multiple dependent variables at the same time and allow for measurement error in both independent and dependent variables. Also, Structural Equation Modeling can estimate both factor structure and factor relationships, which could be more useful for this research. In summary, the variables used in SEM include unobservable latent variables and observable indicator variables. Latent variables can usually be measured by several indicator variables~\cite{ref14}. Therefore, this study will use Structural Equation Modeling to explore whether factors will have an impact on respondents' attitudes toward Safety Tips or not. For the Implementation of Structural Equation Modeling, there are 9 steps as follows. 

\begin{itemize}
\item Constructing theoretical models
\item Formulate the research hypothesis
\item Define variables
\item Sample data collection and processing
\item Reliability and validity testing (EFA/CFA)
\item Model Fit Test
\item Model adjustment and modification
\item Path coefficient analysis
\item Hypothesis testing and conclusion analysis
\end{itemize}

\subsection{Step 1. Constructing theoretical models}
The first step in constructing a Structural Equation Modeling is to construct a theoretical model, identify the research topic, transfer the research problem into several academic concepts, and then construct the basic framework among the different concepts. For this study, the research topic was to explore what factors influence foreign visitors' attitudes toward Safety Tips. Therefore, the basic framework extracted from this is 'Attitude toward Safety Tips' and some related factors. From Table~\ref{table1} we can find that the survey questions are divided into items, and these divided items are the latent variables that we could use in Structural Equation Modeling. Item 1 is demographic information; Item 2 is disaster prevention consciousness; Item 3 is disaster response education, experience on earthquakes; Item 4 is knowledge and perception on earthquakes; Last item 6 is the perception on Safety Tips. Combining the questions in the online survey mentioned before, we initially formulated four latent variables using for constructing Structural Equation Modeling, which are Disaster Prevention Consciousness, Disaster Knowledge, Training experience, and Attitude toward Safety Tips. For some manifest variables, we need to first determine whether the differences in demographic factors are significantly different in the responses of 'Attitude toward Safety Tips' by independent-samples T-test and Analysis of variance (ANOVA). If the results show a significant difference, they can be placed as manifest variables in the SEM. If no significant differences were shown, there was no need to include them as manifest variables. The results of the Independent-samples T-test and ANOVA will be explained in Chapter~\ref{c5}. 


\subsection{Step 2. Formulate the research hypothesis}
The second step in constructing Structural Equation Modeling is to formulate the research hypothesis. Based on the theoretical framework, the path relationships need to extract meaningful latent variables to enrich the path relationships. To construct Structural Equation Modeling for this research, it is necessary to make hypotheses between Disaster Prevention Consciousness, Disaster Knowledge, Training experience, and Attitude toward Safety Tips based on previous research. 

\begin{itemize}
\item Individual characteristics (risk belief, connectedness, knowledge, and past experience with hurricanes), travel-related variables, and the socio-demographic characteristics of tourists influence their decision regarding whether or not to evacuate in the event of a hurricane.~\cite{Cahyanto2014AnEE}
\item The tourists' evacuations were also influenced by tourists' hurricane knowledge and past experience.~\cite{Cahyanto2016StatedPO}
\end{itemize}

Based on these two studies, we can construct a relationship that knowledge, travel-related variables, and socio-demographic characteristics could relate to Evacuation Behaviors, shown in Figure~\ref{fig7}.

%%%%%%%%%%%%%%%%%%%%%%%
%\iffalse
\begin{figure*}[h]
  \includegraphics[width=0.5\linewidth]{Figure/Figure7.png}
  \centering
  \caption{Hypothesis base on previous research - 1 }
  \label{fig7}
\end{figure*}
%\fi

\begin{itemize}
\item Evacuation is significantly related to Socio-demographic factors, such as age, gender, etc, Socioeconomic factors, like educational attainment or household characteristics, etc, Personal characteristics, like hazard experience, knowledge, abilities/impairments, etc. Also, evacuees tend to make use of their familiarity with the surroundings based on their knowledge.~\cite{Wang2021IncorporatingHF}
\end{itemize}

Based on this study, we can construct a relationship that age, gender, hazard experience,  educational attainment, knowledge, abilities could relate to Evacuation Behaviors, shown in Figure~\ref{fig8}. 

\begin{figure*}[h]
  \includegraphics[width=0.5\linewidth]{Figure/Figure8.png}
  \centering
  \caption{Hypothesis base on previous research - 2 }
  \label{fig8}
\end{figure*}

In this study, we initially formulated three hypotheses (H1- H3 ) corresponding to the causal relationships between four latent variables based on the above results, which are shown in Figure~\ref{fig30}.

\begin{itemize}
\item[\textbf{H1}] Disaster Prevention Consciousness has a positive impact on Attitude toward Safety Tips.
\item[\textbf{H2}] Disaster Knowledge has a positive impact on Attitude toward Safety Tips.
\item[\textbf{H3}] Training Experience has a positive impact on Attitude toward Safety Tips.
\end{itemize}

%%%%%%%%%%%%%%%%%%%%%%%
%\iffalse
\begin{figure*}[h]
  \includegraphics[width=0.5\linewidth]{Figure/Figure30.jpg}
  \centering
  \caption{Initial hypothesis used for SEM}
  \label{fig30}
\end{figure*}
%\fi

Each latent variable is represented by multiple indicators, and the summary statistics of these indicators are shown in Table~\ref{table5}. The latent variable 'Disaster Prevention Consciousness' has 5 manifest variables, which are disastrous imagination (Q1), sense of crisis (Q2), other-directed type (Q3), anxiety (Q4), apathy about disasters (Q5). The latent variable 'Disaster Knowledge' has 2 manifest variables, which are knowledge about earthquakes (Q9) and knowledge of how to respond to a disaster (Q10). The latent variable 'Training Experience' has 2 manifest variables, which are the total score of earthquake/tsunami/typhoon/fire training experiences. (Q6) and the number of times participating in earthquake/tsunami/typhoon/fire disaster training (Q7). Latent variable 'Attitude toward Safety Tips' has 4 manifest variables, which are trust level (Q17\_1), the priority of use (Q17\_2), usefulness (Q17\_3), and the possibility of future use (Q17\_4). In addition to this, there are some directly observable manifest variables, such as demographic variables, etc. These variables will be involved in the SEM as manifest variables. Since it is uncertain whether these variables affect their attitude towards Safety Tips, a sample test will be conducted subsequently. There are 6 manifest variables, which are Age (FQ3), Gender (FQ2), number of visits to Japan within 1 year (FQ5), number of visits to any country in the world within 1 year (FQ4), Japanese level (FQ7), and the severity of the earthquake experienced (Q8). Therefore, the final Structural Equation Modeling shows in Figure~\ref{fig9}.

%%%%%%%%%%%%%%%%%%
%\iffalse
\begin{table}[h]
  \caption{Latent variables and manifest variables used for SEM. }
  \label{table5}
  \centering
  \begin{tabular}{|c|l|c|}
  \hline
  Latent variables &  \multicolumn{1}{c|}{Manifest Variables} &  \begin{tabular}{c}Number of\\variables\\included \end{tabular} \\
  \hline
  \multirow{5}{*}{\begin{tabular}{c}Disaster prevention\\consciousness \end{tabular}} & Disastrous Imagination (Q1) & 1\\
  \cline{2-3}
        & Sense of crisis (Q2) & 1 \\
  \cline{2-3}
        & Other-directed type (Q3) & 1\\
  \cline{2-3}
        & Anxiety (Q4) & 1\\
  \cline{2-3}
        & Apathy about disasters (Q5) & 1\\
  \hline
  \multirow{2}{*}{Disaster knowledge} & Knowledge about earthquakes (Q9) & 1\\
  \cline{2-3}
        & Knowledge of how to respond to a disaster (Q10) & 1\\
  \hline
  \multirow{2}{*}{Training experiences} & \begin{tabular}{l}Total score of earthquake/tsunami/typhoon/\\fire training experiences. (Q6)\end{tabular} & 4 \\
  \cline{2-3}
        & \begin{tabular}{l}Number of times participating earthquake/\\tsunami/typhoon/fire disaster training (Q7)\end{tabular} & 4\\
  \hline
   \multirow{4}{*}{\begin{tabular}{c}Attitude toward\\Safety Tips\end{tabular}} & Trust level & 1 \\
  \cline{2-3}
                                            & Priority of use & 1\\
  \cline{2-3}
                                            & Usefulness & 1 \\
  \cline{2-3}
                                            & Possibility of future use & 1 \\
  \hline
   / & Age & 1\\
  \hline
   / & Gender & 1\\
   \hline
   / & Number of visit Japan within 1 year & 1\\
   \hline
   / & \begin{tabular}{l}Number of visit any country in the world\\within 1 year\end{tabular} & 1 \\
   \hline
   / & Japanese level & 1 \\ 
   \hline
   / & The severity of the earthquake experienced & 1 \\
   \hline
  \end{tabular}
\end{table}
%\fi
%%%%%%%%%%%%%%%%%%%%%%%%%%%%%%%%
%\iffalse


\begin{figure*}[h]
  \includegraphics[width=\linewidth]{Figure/Figure9.png}
  \centering
  \caption{Final Hypothesis used for SEM }
  \label{fig9}
\end{figure*}


\subsection{Step 3. Define variables}
The third step is to define each variable. This part requires a definition of all the variables. The definitions of manifest variables are shown in Table~\ref{table2item2}, Table~\ref{table2item3}, Table~\ref{table2item4}, and Table~\ref{table2item6}, including Disaster Prevention Consciousness, Disaster Knowledge, Training experience, and Attitude toward Safety Tips. The definitions of manifest variables are shown in Table~\ref{table2item1} and  Table~\ref{table2item3}, including Age (FQ3), Gender (FQ2), number of visits to Japan within 1 year (FQ5), number of visits to any country in the world within 1 year (FQ4), Japanese level (FQ7), and the severity of the earthquake experienced (Q8).


\subsection{Step 4. Sample data collection and processing}
The data are offered by the Economic and Social Research Institute mentioned in Chapter~\ref{c3}, and the data processing methods are shown in the following. 

\begin{itemize}
\item Item 1 (FQ2-FQ5,FQ7)
\end{itemize}

FQ2: (Male) = 1; (Female) = 2;

FQ3: (Age\,Under\,15) = 1; (Age 16-19) = 2; (Age 20-29) = 3; (Age 30-39) = 4; (Age 40-49) = 5; (Age 50-59) = 6; (Age 60-69) = 7; (Age Over 70) = 8;

FQ4\&FQ5: (0 time) = 0; (1 time) = 1; (2 times) = 2; (3 to 4 times) = 3; (5 to 6 times) = 4; (7 to 9 times) = 5; (Over 10 times) =6;


FQ7: (Cannot understand) = 1; (Basic) = 2; (Intermediate) = 3; (Up Level) = 4; 

\begin{itemize}
\item Item 2 (Q1-Q5)
\end{itemize}

Since each of Q1-Q5 has four sub-problems. Here will use the mean values of the four sub-problems as the final data. For example, Q1 = $mean$ (Q1\_1, Q1\_2, Q1\_3, Q1\_4), Q2-Q5 are processed in the same way.

\begin{itemize}
\item Item 3 (Q6-Q8)
\end{itemize}

Q6 is about past disaster training participation experience. There are four types of disasters: Q6\_1 earthquake, Q6\_2 tsunami, Q6\_3 typhoon, and Q6\_4 fire. Each of them has 12 different types of training experience, shown as Q6\_1/2/3/4\_1 to Q6\_1/2/3/4\_12. Respondents answered with 'Yes' or 'No' in these questions. If none of them were experienced before, 'Yes' was selected in Q6\_1/2/3/4\_13 to indicate that the respondent did not have any of the 12 experiences mentioned above. 

Therefore, 

Q6\_1 = $sum$ (Q6\_1\_1 + Q6\_1\_2 +\dots+ Q6\_1\_12);

Q6\_2 = $sum$ (Q6\_2\_1 + Q6\_2\_2 +\dots+ Q6\_2\_12);

Q6\_3 = $sum$ (Q6\_3\_1 + Q6\_3\_2 +\dots+ Q6\_3\_12);

Q6\_4 = $sum$ (Q6\_4\_1 + Q6\_4\_2 +\dots+ Q6\_4\_12).

Q7 is times of past disaster training experiences. There are also four types of disasters: Q7\_1 earthquake, Q7\_2 tsunami, Q7\_3 typhoon, and Q7\_4 fire.
 
(one time) = 1; (2-3 times) = 2; (4-6 times) = 3; (Over 7 times) =4;

Q8 is about the severity of the earthquake experienced.

(MMI intensity 5 or less / intensity 3 or less) = 1; (MMI intensity 6 / intensity 4) = 2; (MMI intensity 7 / intensity 5 weak) = 3; (MMI intensity 8 / intensity 5 strong) = 4; (MMI intensity 9 / intensity 6 weak) = 5; (MMI intensity 10 / intensity 6 strong) = 6; (MMI intensity 11 to 12 / intensity 7) = 7; (no experience) = 8.


\begin{itemize}
\item Item 4 (Q9-Q10)
\end{itemize}

Q9 has six sub-problems, Q10 has nine sub-problems. Here will use the mean values of the sub-problems as the final data. 

Therefore, 

Q9 = $mean$ (Q9\_1, Q9\_2, \dots, Q9\_6);

Q10 = $mean$ (Q10\_1, Q10\_2, \dots, Q10\_9).

\begin{itemize}
\item Item 6 (Q15-Q17)
\end{itemize}

Q15: (Don't know) = 1; (Only heard name before) = 2; (Know exactly) = 3;

Q16: (Never used before) = 0; (Used before) = 1;
 
Q17\_1, Q17\_2, Q17\_3, Q17\_4: use original data.

\subsection{Step 5. Reliability and validity testing (EFA/CFA)}

The first measurement theory that we need to understand in Structural Equation Modeling is that the observed values equal true values plus bias and noise. Reliability means the degree to which the results are consistent when the same method is used to measure the same object repeatedly. As a result, Reliability indicates how much the measure is free from random error (noise). Reliability is the consistency, stability, and reliability of test results, and is generally expressed in terms of the internal consistency of the test. The higher the reliability coefficient, the more consistent, stable, and reliable the test results are. In addition, the systematic error has little effect on reliability in Reliability tests. Because systematic errors always affect the measurement values, in the same way, they do not cause inconsistency. On the contrary, random errors may cause inconsistency and thus reduce the difficulty. Reliability will be tested through the evaluation of Cronbach's alpha value by SPSS. 

Validity refers to the degree to which a measurement tool or instrument can accurately measure what is to be measured. As a result, Validity indicates how much the measure is free from systematic error (bias). Validity means that the results measured reflect what is intended to be measured and that the results measured are what was intended to be measured. Validity is an indicator of the degree of validity of a measurement instrument, i.e., the degree to which the instrument can measure the characteristics to be measured, and can be simply interpreted as the accuracy and usefulness of a test. 

There are two types of Validity Analysis, one is Exploratory Factor Analysis (EFA), which is generally used in SPSS. EFA is measured through dimensionality reduction. For example, if the questionnaire has 100 questions, then how many dimensions these 100 questions should be grouped into can be achieved by EFA. But most self-designed questionnaires, in fact, when designing the questionnaire, there is already a potential dimensional division. The other one is Confirmatory Factor Analysis (CFA). Compared with EFA, CFA can be used to check the compliance of the three types of validity, including the compliance of the model fit, the consistency of the indicators within each dimension, and the differentiation of the indicators from external indicators. CFA estimates latent variables based on correlated changes in the dataset, which can reduce data dimensionality, standardize the size of multiple metrics, and account for correlations inherent in the dataset~\cite{ref16}. CFA is generally used by Amos, and it is measured by Construct Validity, Convergent Validity, and Discriminant Validity. The main purpose of Construct Validity is to check the fitness of the whole model. If the fit is low, the model, the latent variables, and the relationships between the latent variables are adjusted to improve the fit. The adjustments are explained in subsection~\ref{s7}. 

The Convergent Validity looks at the strength of the correlation between several topics within the same dimension. 
Regarding Convergent Validity, it is used to assess the internal consistency of several items and has similarity with Cronbach's alpha. However, when there are multiple dimensions of the scale, it is not appropriate to use Cronbach's alpha to calculate its internal consistency reliability~\cite{ref29,ref30}. Convergent Validity is generally assessed using Composite Reliability (CR) and Average of Variance Extracted (AVE) to assess Convergent Validity. Convergent Validity~\cite{ref33} is estimated by the standardized factor loading and the respective error variance, the equation shown in the following Formula~\ref{for2}. $\lambda$ denotes the standardized factor loading for item $i$ and $\epsilon$ denotes the respective error variance for item $i$. The error variance is estimated based on the value of the standardized loading ($\lambda$) as Formula~\ref{for3}. And the R-square value is the percent of the variance, which is explained by the latent variable. It is estimated based on the value of the standardized loading ($\lambda$) as Formula~\ref{for4}. In this study, we will calculate CR value by Composite Reliability Calculator~\footnote{https://www.thestatisticalmind.com/composite-reliability/}. Higher CR indicates higher internal consistency and convergence of the conformation. The commonly used evaluation criterion in research is the one mentioned in the book Multivariate data analysis. 5th Edition~\cite{ref32}, that is CR value above 0.7 is acceptable. However, Fornell and Larcker (1981)~\cite{ref31} also suggested that a CR value above 0.6 is acceptable. 

\begin{equation}
\label{for2}
CR=\frac{(\sum \lambda_i )^2}{(\sum \lambda_i )^2+(\sum \epsilon _i )}
\end{equation}

\begin{equation}
\label{for3}
\epsilon_i = 1-\lambda_i^2  
\end{equation}

\begin{equation}
\label{for4}
r^2 = \lambda_i^2 = 1- \epsilon_i  
\end{equation}

Regarding AVE, AVE refers to the average of the explanatory power of the latent variables on the observed variables, and the equation is shown in the following Formula~\ref{for5}~\footnote{https://en.wikipedia.org/wiki/Average\_variance\_extracted}. Here, $k$ is the number of items, $\lambda_i$ is the factor loading of item $i$, and the $Var(e_i)$ is the variance of the error of item $i$. The higher the AVE, the higher the Convergent Validity. According to Fornell and Larcker (1981)~\cite{ref31}, the AVE value needs to be greater than 0.5. Squared Multiple Correlation (SMC) and Std. Factor Loading is used in the calculation of CR and AVE. A higher SMC indicates a higher proportion of true scores and can be used to calculate the confidence level.

\begin{equation}
\label{for5}
AVE = \frac{\sum_{i=1}^{k} \lambda_i^2 }{\sum_{i=1}^{k} \lambda_i^2+\sum_{i=1}^{k} Var(e_i)}
\end{equation}

Convergent Validity and Discriminant Validity can be viewed relatively. The Discriminant Validity is to see whether the differentiation between different dimensions and between topics meets the standard. There are three ways to achieve Discriminant Validity in CFA. The first is to compare the correlation coefficients of the two latent variables, and if their 95\% confidence intervals cover 1.00, it indicates that the constructs lack discriminant power. The second one is to compare two CFA models, one model is a validity model, and the other sets the correlation coefficient of the two potential variables to 1. That is, the fully correlated model/single-factor model. Theoretically, the latter will have a lower fit. If the former is significantly better than the latter, the discriminative power of the area between the two constructs is qualified. If the former is not significantly better than the latter, it means that the two constructs lack zone discrimination~\cite{ref34,ref35}. The third comparison was made using AVE to compare whether the mean AVE of two latent variables was greater than the squared correlation coefficient of the two latent variables~\cite{ref31}. The third method was adopted in this study. Because the questionnaire used in this study has a clear division of dimensions, EFA is not required for this study, and the results of Reliability Analysis and CFA will be presented in Chapter~\ref{c5}.


\subsection{Step 6. Model Fit Test}
Model Fit Test is primarily used for the CFA's Construct Validity test. The fit indices of the single-path coefficient test, which are the p-values and standard errors, and the overall model fit, which are the $\chi^2$ and RMSEA values, are used to evaluate the SEM~\cite{ref16}.

\begin{itemize}
\item \textbf{Chi-square Test ($\chi^2$):} It investigates the possibility of a discrepancy between the model-implied covariance matrix and the original covariance matrix. As a result, the non-significant difference is preferred. The Chi-square test should not be taken too seriously~\cite{ref22,ref23,ref24}, because it is very sensitive to sample size and is not comparable across different SEMs.
\item \textbf{Root Mean Square Error of Approximation (RMSEA):} RMSEA indicates 'badness of fit'. The RMSEA value of 0 means perfect fit, a higher value means short of fit~\cite{ref25}. RMSEA could be less sensitive to sample size than the Chi-square test, so RMSEA can be more useful when detecting model misspecification.
\item \textbf{Comparative Fit Index (CFI):} The amount of variance accounted for in a covariance matrix is represented by the CFI. Its value ranges from 0.0 to 1.0. A higher CFI value indicates a more accurate model fit. Compared with the Chi-square test, CFI can be less sensitive to sample size~\cite{ref26}.
\end{itemize}

In general, the more fit indices that are applied to SEM, the more likely it is that a misspecified model will be rejected, and the less likely it is that a good model will be rejected. As a result, the study must employ at least two distinct fit indices~\cite{ref17}. Some indices have recommended cutoff values, but no indices can be utilized as the standard for all applications~\cite{ref18,ref19,ref20,ref21}. Fan (2016)~\cite{ref16} mentioned other types of evaluation values for model fit, but this study decided to use the common evaluation values that were detailed mentioned above. 

\subsection{Step 7. Model adjustment and modification}
\label{s7}
Based on the structural validity results discussed earlier, the next step is to revise and adjust the Structural Equation Modeling. Therefore, the core of model revision and adjustment is to revise and optimize the measurement model for each latent variable to ensure that the overall model fitness is up to standard. The method of optimizing the measurement model, in addition to removing the less significant variables through the reliability results of SPSS, is to modify the indicator correction suggestions through the output of Amos. In general, there are two indicators. The first one is the Standardized Regression Weights (SRW), or Squared Multiple Correlation (SMC), which is the square of the SRW, and the SRW should be greater than 0.7 and the SMC should be greater than 0.5~\cite{ref15,ref27}. Another one is the Modification Indices. By M.I. value of covariances in the modification indices results. M.I. value refers to the amount of change that can be reduced by the corrected chi-square value, and the higher the value, the more it helps to correct the model. However, because the correlation between residuals is not academically interpretable. This is because the premise assumptions state that the residuals are independent. Therefore, correcting the model is done by removing the problem itself where the residuals have the greatest impact. As mentioned before, since the survey used in this study has a clear dimensional division, EFA was not performed. Therefore, this part of the model fitness adjustment will be performed with the first method as the main one.


\subsection{Step 8. Path coefficient analysis}
First look at Unstandardized Estimates, which describes how many units the dependent variable increases for each unit increase in the independent variable. Before evaluating the SEM fit, the estimated coefficients must be checked by Offending estimates to determine whether they are outside the acceptable range. Specifically, the first should look at the estimated coefficients of the error terms, which should not be negative. Secondly, we have to look at the factor loadings and whether the path coefficients are significant. Next, we look at the Standardized Estimates, which describes how many standard deviations the dependent variable will increase for each standard deviation increase in the independent variable. The factor loading of the measurement model should be greater than 0.7 and the SMC greater than 0.5, which proves to be very good; However, in general, the factor loading of the scales developed by social science researchers is not too high considering that it may be limited by the nature of the measurement (e.g., the range of attitude measurement is too wide and not easily focused, the construct is too vague and not easily defined), the external interference and measurement error, or even the nature of the construct. Tabachnica and Fidell, (2007)~\cite{ref28} suggest that a factor loading of 0.55 or higher and an SMC of 0.3 or higher can be declared good. As for the structural model of SMC, the main focus on significance is sufficient. To summarize, Unstandardized Estimates are used to see if they are significant, and Standardized Estimates are used to see the magnitude of the influence factors.


\subsection{Step 9. Hypothesis testing and conclusion analysis}
The hypothesis testing and conclusion analysis section is equivalent to a summary of SEM. We need to describe in detail the SEM model, the results of the model fitness, and the significance of the hypothesis relationships. The results are used to summarize the similarities/differences between the original hypothesis of the study and the results given by SEM, and to give an analysis based on the results. Finally, it is considered that this study needs to propose some suggestions for the future development of Safety Tips, so a proposal will be presented based on the conclusions presented in the results.





\section{Methodology for achieving Objective - 3}

\textbf{Objective 3: To explore patterns of information seeking and evacuation behaviors.}

For research objective 3, this study wanted to explore respondents' information-seeking behaviors and evacuation behaviors through their selections of behaviors. In order to understand which of the behaviors are preferred and which are selected more frequently, we chose to measure both the selected rate and the selected score. The calculation of the selected rate and the selected score will both do three times, for 300 Japanese samples, 1500 foreign visitors samples, and 1800 of all respondents' samples.

\subsection{Selected Rate}
Because the selected rate wants to explore which behaviors are used more, this part does not take the order factor into account. No matter the behavior is selected in which order, it will count as 1 point. The selected rate is equal to the Sum selected point divided by the sample number, shown in Figure~\ref{fig10}.

%%%%%%%%%%%%%%%%%%%%%%%%%%%%
%\iffalse
\begin{figure*}[h]
  \includegraphics[width=\linewidth]{Figure/Figure10.png}
  \centering
  \caption{Selected rate }
  \label{fig10}
\end{figure*}
%\fi

\subsection{Selected Score}

Since the Selected score wants to explore which behaviors are used first, this part needs to take the order factor into account. The higher the preference is, the higher the score will be. So the first selected action is scored as 5, the second is scored as 4, the third is scored as 3, the fourth is scored as 2, and the last selected action is scored as 1. The Selected score is equal to the total score divided by the Sum selected point, shown in Figure~\ref{fig11}.

%%%%%%%%%%%%%%%%%%%%%%%%%%%
%\iffalse
\begin{figure*}[h]
  \includegraphics[width=\linewidth]{Figure/Figure11.png}
  \centering
  \caption{Selected score}
  \label{fig11}
\end{figure*}
%\fi

\subsection{Behavior Pattern}

In the analysis of the selected score and selected rate of study objective 3, we will find that all the results will be relatively scattered. This is because, in this study, the number of available selections is relatively large, which makes the results more scattered. However, from the results, we can see that there could be similarities in the behavior patterns of people. Here the word 'pattern' means behavior patterns, not specific behavior. For example, the behavior of  'collecting information' is the same, the difference is how to collect information, from official websites, from disaster prevention software, from disaster prevention websites, from SNS, etc. So how to divide detailed behaviors into patterns? From the available selection, we can find that the behavior is mainly divided into two kinds, which are information-seeking behavior and evacuation behavior. First, regarding information-seeking behavior, we can find that there are two main patterns, one is 'No-face-to-face information seeking' and the other is 'Face-to-face information seeking'. And for the evacuation behavior, we can also find two main patterns, one is ' Self-evacuation behaviors ' and the other is ' Following evacuation guidance behavior '. Then, we divided all the selections into the patterns they belong to, which can be found in Figure~\ref{fig12}.

%%%%%%%%%%%%%%%%%%%%%%%%%%%
%\iffalse
\begin{figure*}[h]
  \includegraphics[width=\linewidth]{Figure/Figure12.png}
  \centering
  \caption{behavior patterns}
  \label{fig12}
\end{figure*}
%\fi

After attributing all the actions to the 4 patterns, we went through the flow of the respondents' actions with the help of the Sankey diagram. The Sankey diagram is a flow chart that shows the flow from each set of values to another set of values. The thickness of the lines expresses the number of values present in the group. In the Sankey diagram, the number indicates the No. of action, so 1-5 means the first action to the fifth action. Capital letters indicate behavior patterns.' No-face-to-face information seeking' is A; 'Face-to-face information seeking' is B; 'Self-evacuation behaviors' is C; 'Following evacuation guidance behavior' is D. Thus, the behavior from the first to the fifth cohort can be clearly represented in the results of the Sankey diagram. As an example, A1 indicates that the 1st response action during the disaster is behavior pattern A, which is 'No-face-to-face information seeking'.





























%% Some LaTeX commands I define for my own nomenclature.
% If you have to, it's better to change nomenclature once here than in a 
% million places throughout your thesis!
%======================================================================
\chapter{Results and Discussion}
\label{c5}
%======================================================================

\section{Results for Objective 1 }
\subsection{First Task}
For Q15, do the respondents know about SafetyTips before, the result is shown in Figure~\ref{fig13}. we can find that around 50\% of the respondents from the UK and Korea do not know Safety Tips before, around 80\% of the respondents from China and Thailand know or at least heard Safety Tips before, while around 90\% of the respondents from Indonesia know or at least heard Safety Tips before. 

\begin{figure*}[h]
  \includegraphics[width=0.8\linewidth]{Figure/Figure13.jpg}
  \centering
  \caption[Survey result of Q15]{Survey result of Q15(Do you know Safety Tips or not ?)}
  \label{fig13}
\end{figure*}

For Q16, did the respondents use SafetyTips before, the result shows in Figure~\ref{fig14}. we can find that more than 50\% of the respondents from China(66.2\%), Korea(72\%), and Thailand(55.1\%) did not use Safety Tips before, while more than 50\% of respondents from Indonesia(65.8\%) and the UK(54.7\%) have used Safety Tips before. Among all countries, respondents from Indonesia have the highest usage rate of Safety Tips at 65.8\%. Koreans had the lowest, at 28\%. 

\begin{figure*}[h]
  \includegraphics[width=0.8\linewidth]{Figure/Figure14.jpg}
  \centering
  \caption[Survey result of Q16]{Survey result of Q16(Do you use Safety Tips before or not ?)}
  \label{fig14}
\end{figure*}

For Q17\_1, Will the respondents trust Safety Tips more than information from their country, the result shows in Figure~\ref{fig15}. we can find that over 90\% of respondents from Thailand(91.1\%) and Indonesia(91.7\%) said they trusted information from Safety Tips more than from their own countries, and more than 80\% of respondents from China(80.3\%) feel that the information on Safety Tips could be trusted more than their own countries, while respondents from the UK(78\%) and Korea(77.3\%) have a relatively low level of trust in Safety Tips compared to the other three countries, but still around 70\%. 

\begin{figure*}[h]
  \includegraphics[width=0.8\linewidth]{Figure/Figure15.jpg}
  \centering
  \caption[Survey result of Q17\_1]{Survey result of Q17\_1(Will you trust Safety Tips more than information from your own country ?)}
  \label{fig15}
\end{figure*}

For Q17\_2, Will the respondents use Safety Tips before searching information from their country, the result shows in Figure~\ref{fig16}. we can find that over 90\% of respondents from Thailand(90.7\%) and Indonesia(94.1\%) said they use Safety Tips to search information before their own country's, and more than 80\% of respondents from China(88\%), the UK(82.3\%) and Korea(82.3\%) will use Safety Tips to find information before their own country. 

\begin{figure*}[h]
  \includegraphics[width=0.8\linewidth]{Figure/Figure16.jpg}
  \centering
  \caption[Survey result of Q17\_2]{Survey result of Q17\_2(Will you use Safety Tips before searching information from your own country ?)}
  \label{fig16}
\end{figure*}

For Q17\_3, do the respondents think Safety Tips could be useful during the evacuation, the result is shown in Figure~\ref{fig17}. we can find that over 90\% of respondents from Thailand(95.4\%), China(95.7\%), and Indonesia(96.6\%) think Safety Tips could be useful during the evacuation, and more than 80\% of respondents from the UK(87\%) and Korea(84.9\%) think Safety Tips could be useful during evacuation. 

\begin{figure*}[h]
  \includegraphics[width=0.8\linewidth]{Figure/Figure17.jpg}
  \centering
  \caption[Survey result of Q17\_3]{Survey result of Q17\_3(Do you think Safety Tips could be useful during evacuation ?)}
  \label{fig17}
\end{figure*}

For Q17\_4, will the respondents use Safety Tips in the future, the result shows in Figure~\ref{fig18}. we can find that over 90\% of respondents from Thailand(93.3\%), China(95.3\%), and Indonesia(95.3\%) think they will use Safety Tips in the future, and more than 80\% of respondents from the UK(87\%) and Korea(84.9\%) think they will use Safety Tips in the future.

\begin{figure*}[h]
  \includegraphics[width=0.8\linewidth]{Figure/Figure18.jpg}
  \centering
  \caption[Survey result of Q17\_4]{Survey result of Q17\_4(Will you use Safety Tips in the future ?)}
  \label{fig18}
\end{figure*}

From the above results, we can conclude that from the usage experience, Safety Tips could be more popular and well-known in Indonesia(90\%), China(77\%), and Thailand(79\%) rather than in the UK(53\%) and Korea(50\%). Also, among those respondents that know Safety Tips or heard them before, their usage rate is lower than 70\%. China(33.8\%), Korea(28\%), Thailand(44.9\%), Indonesia(65.8\%) and the UK(54.7\%). Then from the attitude toward Safety Tips, we can conclude that over 77\% of the respondents say that they trust Safety Tips more than information from their own countries, over 82\% of the respondents say that they will use Safety Tips to search information before from their own countries, and over 84\% of the respondents say that they believe Safety Tips could be useful during evacuation and will use Safety Tips in the future.

\subsection{Second Task}
In the second task, we aim to find whether the two factors of respondents' past awareness and whether they used it before had an impact on their attitudes toward Safety Tips. After dividing all respondents into 5 groups, we can find the differences between groups. 

First, from the results of the grouping, we can see that 80\% of those who know Safety Tips have used Safety Tips before. For those who had only heard of Safety Tips, only 22\% of the respondents had used Safety Tips before. Comparing the two sets of data, it is clear that the usage rate has decreased significantly.

For Q17\_1, Will the respondents trust Safety Tips more than information from their country, the result shows in Figure~\ref{fig19}. we can find that respondents who know exactly and used Safety Tips before have shown the highest trust toward Safety Tips, as more than 75\% of the respondents said they trust the information on Safety Tips rather than from their own countries. Respondents who know exactly but never used Safety Tips before and respondents who heard Safety Tips before but never used Safety Tips before are more likely to trust the information on Safety Tips. Respondents who heard and used Safety Tips before and respondents who do not know and never used this application before have shown a relatively negative attitude toward Safety Tips. 

\begin{figure*}[h]
  \includegraphics[width=0.8\linewidth]{Figure/Figure19.jpg}
  \centering
  \caption[5 groups of respondents' survey result of Q17\_1]{5 groups of respondents' survey result of Q17\_1(Will you trust Safety Tips more than information from your own country ?)}
  \label{fig19}
\end{figure*}

For Q17\_2, Will the respondents use Safety Tips before searching information from their country, the result shows in Figure~\ref{fig20}. we can find that the respondents who know exactly and used Safety Tips before have shown the highest usage possibility on Safety Tips, as more than 80\% of the respondents said Safety Tips have a higher priority of usage rather than from their own countries. Respondents that know exactly but never used Safety Tips before and respondents who heard Safety Tips before but never used Safety Tips before have shown similar attitudes on the priority of using Safety Tips. Respondents who do not know and never used before and respondents who heard Safety Tips before and used Safety Tips before have shown a little bit lower usage priority. 

\begin{figure*}[h]
  \includegraphics[width=0.8\linewidth]{Figure/Figure20.jpg}
  \centering
  \caption[5 groups of respondents' survey result of Q17\_2]{5 groups of respondents' survey result of Q17\_2(Will you use Safety Tips before searching information from your own country ?)}
  \label{fig20}
\end{figure*}

For Q17\_3, do the respondents think Safety Tips could be useful during the evacuation, the result is shown in Figure~\ref{fig21}. we can find that respondents that know exactly and used Safety Tips before and respondents who heard Safety Tips before but never used Safety Tips before could be more likely to believe Safety Tips could be useful during evacuation. Respondents that do not know and never used before, respondents who heard and used Safety Tips before, respondents who know exactly but never used Safety Tips before could be more likely to show lower usefulness of Safety Tips. 

\begin{figure*}[h]
  \includegraphics[width=0.8\linewidth]{Figure/Figure21.jpg}
  \centering
  \caption[5 groups of respondents' survey result of Q17\_3]{5 groups of respondents' survey result of Q17\_3(Do you think Safety Tips could be useful during evacuation ?)}
  \label{fig21}
\end{figure*}

For Q17\_4, will the respondents use Safety Tips in the future, the result shows in Figure~\ref{fig22}. we can find that respondents that know exactly and used Safety Tips before and respondents who heard but never used Safety Tips before have shown higher usage possibly Safety Tips in the future. Respondents that do not know and never used before, respondents who heard and used Safety Tips before, respondents who know exactly but never used Safety Tips before have shown relatively lower usage possibility.

\begin{figure*}[h]
  \includegraphics[width=0.8\linewidth]{Figure/Figure22.jpg}
  \centering
  \caption[5 groups of respondents' survey result of Q17\_4]{5 groups of respondents' survey result of Q17\_4(Will you use Safety Tips in the future ?)}
  \label{fig22}
\end{figure*}

From the above results, we can conclude that respondents that know exactly and used Safety Tips before could show higher trust and higher priority of use on Safety Tips, also they are more likely to believe Safety Tips can be useful during the evacuation, and they will use it in the future. And, respondents that heard but never used  Safety Tips before have shown better attitudes on Safety Tips rather than respondents who do not know and never used  Safety Tips before, respondents who heard and used Safety Tips before, respondents who know exactly but never used it Safety Tips before.

The results of the grouping indicate that 80\% of those who clearly know Safety Tips have actually used Safety Tips before. For those who had only heard of Safety Tips, only 22\% of the respondents had used Safety Tips before. Comparing the two sets of data, it is clear that the usage rate has decreased significantly. This shows that people who have a more detailed awareness of Safety Tips are more likely to use this application, so if we want to increase the usage of Safety Tips, it would be helpful to increase foreign visitors' awareness of this application.
\cleardoublepage
%%%%%%%%%%%%%%%%%%%
\section{Results for Objective 2 }
\subsection{Test for manifest variables }
Table~\ref{table27} shows the results of the statistical description with the maximum value, minimum value, mean, and standard deviation for each variable.

\begin{table}[h]
  \caption[Statistical Description]{Statistical Description (N=491)}
  \label{table27}
  \centering
  \begin{tabular}{l|cccc}
 \hline
\multicolumn{1}{c|}{Variable} & Min value  & Max value & Mean & Standard Deviation \\
 \hline
Country	& 2&6&4.32&1.30\\
gender&1&2&1.48&0.50\\
age&2&7&4.37&1.26\\
Visit\_country&1&10&4.39&2.61\\
Visit\_Japan&1&11&4.99&2.92\\
Japanese\_Level&1&4&2.59&0.77\\
Q1&1&6&4.94&1.02\\
Q2&1&6&4.95&1.01\\
Q3&1&6&5.08&0.99\\
Q4&1&6&4.55&1.03\\
Q5&1&6&3.83&1.41\\
Q6\_1\_earthquake&0&12&4.95&3.74\\
Q6\_2\_tsunami&0&12&4.53&3.33\\
Q6\_3\_typhoon&0&12&4.6&3.43\\
Q6\_4\_fire&0&12&5.01&4.06\\
Q7\_1\_earthquake&0&4&1.66&1.60\\
Q7\_2\_tsunami&0&4&0.80&1.27\\
Q7\_3\_typhoon&0&4&0.72&1.29\\
Q7\_4\_fire&0&4&1.48&1.61\\
Q8\_experience\_earthquake&1&8&3.15& 1.87\\
Q9&1&6&4.88&0.97\\
Q10	&1&6&5.01&0.95\\
Q15 Safetytips&2&3&2.73&0.45\\
Q16 Safetytips\_use&1&1&1&0\\
Q17 Safetytips\_trust\_1&1&6&4.8&1.31\\
Q17 Safetytips\_trust\_2&1&6&4.95&1.08\\
Q17 Safetytips\_trust\_3&1&6&5.10&1.01\\
Q17 Safetytips\_trust\_4&1&6&5.11&1.02\\
 \hline
  \end{tabular}
\end{table}

For some group-based data, the frequency descriptions of these variables are shown in the following. \crefrange{table28a}{table28f} is the frequency description of Item 1, including Country, Gender, Age, number of visited countries, number of visited Japan, Japanese Level. Table~\ref{table28g} is the frequency description of Item 3, that is the severity of the earthquake experience.

\begin{table}[h]
  \caption[Frequency Description of Country]{Frequency Description of Country (N=491)}
  \label{table28a}
  \centering
  \begin{tabular}{l|cc}
 \hline
\multicolumn{1}{c|}{Category}&Number&Rate\\
 \hline
China&78&15.90\%\\
South Korea&42&8.60\%\\
Thailand&105&21.40\%\\
Indonesia&179&36.50\%\\
the UK&87&17.70\%\\
 \hline
  \end{tabular}
\end{table}

\begin{table}[h]
  \caption[Frequency Description of Gender]{Frequency Description of Gender (N=491)}
  \label{table28b}
  \centering
  \begin{tabular}{l|cc}
 \hline
\multicolumn{1}{c|}{Category}&Number&Rate\\
 \hline
Male   & 253 & 51.50\% \\
Female & 238 & 48.50\% \\
 \hline
  \end{tabular}
\end{table}

\begin{table}[h]
  \caption[Frequency Description of Age]{Frequency Description of Age (N=491)}
  \label{table28c}
  \centering
  \begin{tabular}{l|cc}
 \hline
\multicolumn{1}{c|}{Category}&Number&Rate\\
 \hline
Age 16-19   & 13  & 2.60\%  \\
Age 20-29   & 136 & 27.70\% \\
Age 30-39   & 126 & 25.70\% \\
Age 40-49   & 113 & 23\%    \\
Age 50-59   & 77  & 15.70\% \\
Age 60-69   & 26  & 5.30\%  \\
Age over 70 & 0   & 0\% \\
 \hline
  \end{tabular}
\end{table}

\begin{table}[h]
  \caption[Frequency Description of number of visited country ]{Frequency Description of number of visited country (N=491)}
  \label{table28d}
  \centering
  \begin{tabular}{l|cc}
 \hline
\multicolumn{1}{c|}{Category}&Number&Rate\\
 \hline
0 time        & 0                    & 0\%                  \\
1 time        & 73                   & 14.90\%              \\
2 times       & 63                   & 12.80\%              \\
3 to 4 times  & 152                  & 31\%                 \\
5 to 6 times  & 95                   & 19.30\%              \\
7 to 9 times  & 99                   & 20.20\%              \\
Over 10 times & 9                    & 1.80\%               \\
 \hline
  \end{tabular}
\end{table}

\begin{table}[h]
  \caption[Frequency Description of number of visited Japan]{Frequency Description of number of visited Japan (N=491)}
  \label{table28e}
  \centering
  \begin{tabular}{l|cc}
 \hline
\multicolumn{1}{c|}{Category}&Number&Rate\\
 \hline
0 time        & 0   & 0\%     \\
1 time        & 47  & 9.60\%  \\
2 times       & 65  & 13.20\% \\
3 to 4 times  & 147 & 29.90\% \\
5 to 6 times  & 85  & 17.30\% \\
7 to 9 times  & 81  & 16.50\% \\
Over 10 times & 66  & 13.40\% \\
 \hline
  \end{tabular}
\end{table}

\begin{table}[h]
  \caption[Frequency Description of Japanese Level]{Frequency Description of Japanese Level (N=491)}
  \label{table28f}
  \centering
  \begin{tabular}{l|cc}
 \hline
\multicolumn{1}{c|}{Category}&Number&Rate\\
 \hline
Cannot understand                    & 25  & 5.10\%  \\
Basic        & 214 & 43.60\% \\
Intermediate & 190 & 38.70\% \\
Up level     & 62  & 12.60\% \\
 \hline
  \end{tabular}
\end{table}

\begin{table}[h]
  \caption[Frequency Description of severity of the earthquake experienced]{Frequency Description of severity of the earthquake experienced (N=491)}
  \label{table28g}
  \centering
  \begin{tabular}{l|cc}
 \hline
\multicolumn{1}{c|}{Category}&Number&Rate\\
 \hline
MMI intensity 5 or less / intensity 3 or less & 110 & 22.40\% \\
MMI intensity 6 / intensity 4                 & 90  & 18.30\% \\
MMI intensity 7 / intensity 5 weak            & 119 & 24.20\% \\
MMI intensity 8 / intensity 5 strong          & 80  & 16.30\% \\
MMI intensity 9 / intensity 6 weak            & 30  & 6.10\%  \\
MMI intensity 10 / intensity 6 strong                                 & 18  & 3.70\%  \\
MMI intensity 11 to 12 / intensity 7                                  & 30  & 6.10\%  \\
no earthquake experience                                              & 14  & 2.90\%  \\
 \hline
  \end{tabular}
\end{table}

For the scale questions in the questionnaire, we conducted a one-sample t-test, and the results of the study can indicate whether people have clear attitudes in their responses to these questions. For question Q17 about the attitude toward Safety Tips, the answers to the scale questions were divided into 6 dimensions, so we set 3.5 as the test value. Table~\ref{table29} shows the results of the one-sample t-test. For Q17\_1 to Q17\_1, the mean values of 491 respondents are 4.80$\pm$1.31; 4.95$\pm$1.08; 5.10$\pm$1.01; 5.11$\pm$1.02; All of the p values are less than 0.01, mean all are statistically significant at $p<0.001$ level. Compared to the test value of 3.5, indicating that all have significant differences in the attitude toward Safety Tips, and all mean values are higher than the test value of 3.5, implying that respondents show positive attitudes to all of the questions towards Safety Tips. 

\begin{table}[h]
  \caption[Result of one-sample t-test]{Result of one-sample t-test (N=491)}
  \label{table29}
  \centering
  \begin{tabular}{l|cccc}
 \hline
                  & Mean                 & Test Value & t                          & p                        \\
Q17Safetytips\_trust\_1 & 4.80$\pm$1.31            & 3.5                            & 29.96 & 0.00 \\
Q17Safetytips\_trust\_2 & 4.95$\pm$1.08            & 3.5                            & 29.96 & 0.00 \\
Q17Safetytips\_trust\_3 & 5.10$\pm$1.01            & 3.5                            & 35.18 & 0.00 \\
Q17Safetytips\_trust\_4 & 5.11$\pm$1.02            & 3.5                            & 34.81 & 0.00 \\
 \hline
  \end{tabular}
\end{table}

On the other hand, this study also used the Chi-squared test and ANOVA to test whether those manifest variables could show significant differences in respondents' attitudes towards Safety Tips. Table~\ref{table30} shows the statistical results of the sample data for the manifest variable Gender and the four attitude-related questions in Q17. We can see that in Q17\_2/3/4, there are less than 5 male and female respondents. Therefore, in order to observe the significant differences more clearly, here we group the Q17 answers into two categories, 1 to 3 as negative attitude and 4 to 6 as a positive attitude. \crefrange{table31a}{table31d} show the Chi-squared test results. From the results, we find that the p-values between gender and all four attitude related questions are bigger than 0.05, (gender*Q17\_1: $p=0.525>0.05$; gender*Q17\_2: $p=0.705>0.05$; gender*Q17\_3: $p=0.490>0.05$; gender*Q17\_4: $p=0.852>0.05$;)which indicates that gender differences are not significantly different in Q17's responses. 

\begin{table}[h]
  \caption{Sample data of Gender*Q17}
  \label{table30}
  \centering
\begin{tabular}{cc|ccccccc}
\hline
Question & Answer & 1  & 2  & 3  & 4  & 5   & 6   & Total \\
\hline
\multirow{3}{*}{Q17\_1}   & Female & 10 & 6  & 15 & 34 & 76  & 97  & 238                       \\
         & Male   & 11 & 8  & 19 & 50 & 83  & 82  & 253                       \\
         & Total  & 21 & 14 & 34 & 84 & 159 & 179 & 491                       \\
\hline
\multirow{3}{*}{Q17\_2}   & Female & 0  & 9  & 17 & 23 & 108 & 81  & 238                       \\
         & Male   & 3  & 7  & 15 & 50 & 85  & 93  & 253                       \\
         & Total  & 3  & 16 & 32 & 73 & 193 & 174 & 491                       \\
\hline
\multirow{3}{*}{Q17\_3}   & Female & 0  & 2  & 13 & 34 & 73  & 116 & 238                       \\
         & Male   & 3  & 4  & 13 & 49 & 85  & 99  & 253                       \\
         & Total  & 3  & 6  & 26 & 83 & 158 & 215 & 491                       \\
\hline
\multirow{3}{*}{Q17\_4}   & Female & 2  & 3  & 10 & 25 & 91  & 107 & 238                       \\
         & Male   & 4  & 5  & 8  & 46 & 89  & 101 & 253                       \\
         & Total  & 6  & 8  & 18 & 71 & 180 & 208 & 491                     \\
\hline         
\end{tabular}
\end{table}

\begin{table}[h]
  \caption{Chi-square test result of Gender*Q17\_1 }
  \label{table31a}
  \centering
\begin{tabular}{c|ccc}
\hline
Gender & negative & positive & Total \\
\hline
Female & 31                           & 207                          & 238                       \\
Male   & 38                           & 215                          & 253                       \\
Total  & 69                           & 422                          & 491                       \\
\hline
p-value      &        &      & 0.525   \\
\hline                   
\end{tabular}
\end{table}

\begin{table}[h]
  \caption{Chi-square test result of Gender*Q17\_2 }
  \label{table31b}
  \centering
\begin{tabular}{c|ccc}
\hline
Gender & negative & positive & Total \\
\hline
Female & 26                           & 212                          & 238                       \\
Male   & 25                          & 228                         & 253                       \\
Total  & 51                           & 440                          & 491                       \\
\hline
p-value      &        &      & 0.705   \\
\hline                   
\end{tabular}
\end{table}

\begin{table}[h]
  \caption{Chi-square test result of Gender*Q17\_3 }
  \label{table31c}
  \centering
\begin{tabular}{c|ccc}
\hline
Gender & negative & positive & Total \\
\hline
Female & 15                           & 223                          & 238                       \\
Male   & 20                          & 233                       & 253                       \\
Total  & 35                           & 456                          & 491                       \\
\hline
p-value     &        &      & 0.490   \\
\hline                   
\end{tabular}
\end{table}

\begin{table}[h]
  \caption{Chi-square test result of Gender*Q17\_4 }
  \label{table31d}
  \centering
\begin{tabular}{cccc}
\hline
Gender & negative & positive & Total \\
\hline
Female & 15                           & 223                          & 238                       \\
Male   & 17                         & 236                      & 253                       \\
Total  & 32                           & 459                          & 491                       \\
\hline
p-value      &        &      & 0.852   \\
\hline                   
\end{tabular}
\end{table}

\crefrange{table32a}{table32e} shows the ANOVA results of age, number of visited countries, number of visited Japan, Japanese level, the severity of the earthquake experienced. Through ANOVA, we analyzed several manifest variables for the test of differences in attitudes. The results of Table~\ref{table32a} shows that as p-value is less than 0.05 for the first three questions (age*Q17\_1: $p=0.000<0.05$; age*Q17\_2: $p=0.035<0.05$; age*Q17\_3: $p=0.000<0.05$), the age difference is significantly different on their attitudes toward Safety Tips . However, for the possibility of future use, there is no significant difference because p is bigger than 0.05 (age*Q17\_4: $p=0.076>0.05$). For these first three questions with significant differences, the mean values were able to give the following results. For trust level, the older the age, the more positive the attitude. For the priority of use, the attitudes of the respondents from Age 20 to 39 are lower than those of the other age groups, and from Age over 40, the attitudes tend to change more positively with age. For usefulness, the results do not show any tendency to change. However, Age 60 to 69 respondents gave the most positive attitudes, while Age 30 to 39 respondents had the most negative attitudes on average. The results of Table~\ref{table32b} show that the differences in the number of visited countries do not reflect significant differences in trust level and priority of use because the p-values for the first two questions are bigger than 0.05 (number of visited countries*Q17\_1: $p=0.140>0.05$; the number of visited countries*Q17\_2: $p=0.270>0.05$;). However, for usefulness and Possibility of future use, there are significant differences since the p-value is less than 0.05 (number of visited countries*Q17\_3: $p=0.009<0.05$; the number of visited countries*Q17\_4: $p=0.049<0.05$;). For these two questions with significant differences, the mean results did not show any tendency, however, the respondents with more than 5 visits could have better attitudes. The results of Table~\ref{table32c} shows that the p-value for each question is less than 0.05 (number of visited Japan*Q17\_1: $p=0.000<0.05$; the number of visited Japan*Q17\_2: $p=0.002<0.05$; the number of visited Japan*Q17\_3: $p=0.000<0.05$; the number of visited Japan*Q17\_4: $p=0.000<0.05$;), so the difference in the Number of visited Japan reflects a significant difference in the responses to attitudes toward Safety Tips. The average results show that the more the number of visits to Japan, the more positive the respondents' attitudes toward Safety Tips. The results of Table~\ref{table32d} shows that the p-value for each question is less than 0.05 (Japanese level*Q17\_1: $p=0.000<0.05$; Japanese level*Q17\_2: $p=0.032<0.05$; Japanese level*Q17\_3: $p=0.000<0.05$; Japanese level*Q17\_4: $p=0.021<0.05$;), so the difference in Japanese language proficiency is reflected in the significant difference in the attitudes toward Safety Tips responses. The results of the averages basically show that the higher the Japanese language level, the more positive the attitudes toward Safety Tips of the respondents. The results of Table~\ref{table32e} show that the p-value for each question is greater than 0.05 (severity of experienced earthquakes*Q17\_1: $p=0.548>0.05$; severity of experienced earthquakes*Q17\_2: $p=0.127>0.05$; severity of experienced earthquakes*Q17\_3: $p=0.389>0.05$; severity of experienced earthquakes*Q17\_4: $p=0.121>0.05$;), so the difference in the severity of experienced earthquakes do not reflect a significant difference in attitudes toward Safety Tips, which means that whether people have experienced a severe earthquake in the past does not significantly affect the respondents' answers to Q17. This means that whether people have experienced a serious earthquake or not does significantly affect their answers to Q17.

\begin{table}[h]
  \caption{Chi-square test result of Age*Q17}
  \label{table32a}
  \centering
  \begin{tabular}{l|cccc}
 \hline
        \multicolumn{1}{c|}{Age}          & Q17\_1               & Q17\_2 & Q17\_3    & Q17\_4      \\
\hline
Age 16 to 19 & 4.54$\pm$1.27                    & 5.00$\pm$1.41                    & 5.00$\pm$1.35                    & 4.92$\pm$1.38                    \\
Age 20 to 29 & 4.55$\pm$1.40                    & 4.82$\pm$1.10                    & 4.90$\pm$1.05                    & 4.98$\pm$1.01                    \\
Age 30 to 39 & 4.67$\pm$1.29                    & 4.81$\pm$1.11                    & 4.87$\pm$1.04                    & 4.99$\pm$1.18                    \\
Age 40 to 49 & 4.83$\pm$1.32                    & 5.02$\pm$1.08                    & 5.32$\pm$0.93                    & 5.25$\pm$0.83                    \\
Age 50 to 59 & 5.13$\pm$1.19                    & 5.22$\pm$0.10                    & 5.30$\pm$0.89                    & 5.26$\pm$1.04                    \\
Age 60 to 69 & 5.73$\pm$0.53                    & 5.27$\pm$0.53                    & 5.81$\pm$0.40                    & 5.38$\pm$0.57                    \\
\hline
p-value&           0.000&         0.035&         0.000&   0.076     \\
 \hline
  \end{tabular}
\end{table}

\begin{table}[h]
  \caption{Chi-square test result of Number of visited countries*Q17}
  \label{table32b}
  \centering
  \begin{tabular}{l|cccc}
 \hline
        \multicolumn{1}{c|}{Number of visited countries}          & Q17\_1               & Q17\_2 & Q17\_3    & Q17\_4       \\
\hline
1 time        & 4.47$\pm$1.72           & 4.77$\pm$1.37  & 4.85$\pm$1.35 & 4.88$\pm$1.39  \\
2 times       & 4.62$\pm$1.22 & 5.03$\pm$0.88 & 5.14$\pm$0.96 & 4.97$\pm$0.98  \\
3 to 4 times  & 4.72$\pm$1.27 & 4.89$\pm$1.09 & 4.95$\pm$0.95& 5.07$\pm$0.99 \\
5 to 6 times  & 4.94$\pm$1.18 & 4.94$\pm$0.93& 5.31$\pm$0.88 & 5.17$\pm$0.88 \\
7 to 9 times  & 5.14$\pm$1.05 & 5.15$\pm$1.00 & 5.28$\pm$0.88 & 5.34$\pm$0.88 \\
over 10 times & 4.78$\pm$1.79& 4.89$\pm$1.36 & 5.22$\pm$1.09 & 5.33$\pm$0.71\\
\hline
p-value&           0.140&         0.270&         0.009&   0.049     \\
 \hline
  \end{tabular}
\end{table}

\begin{table}[h]
  \caption{Chi-square test result of Number of visited Japan*Q17}
  \label{table32c}
  \centering
  \begin{tabular}{l|cccc}
 \hline
        \multicolumn{1}{c|}{Number of visited Japan}          & Q17\_1               & Q17\_2 & Q17\_3    & Q17\_4       \\
\hline
1 time        & 3.91$\pm$1.59 & 4.47$\pm$1.16   & 4.45$\pm$1.19   & 4.55$\pm$1.41 \\
2 times       & 4.82$\pm$1.36 & 4.94$\pm$1.18  & 5.06$\pm$1.18   & 5.09$\pm$1.13 \\
3 to 4 times  & 4.61$\pm$1.30 & 4.89$\pm$1.10 & 4.98$\pm$0.94 & 4.92$\pm$1.03  \\
5 to 6 times  & 4.79$\pm$1.22 & 4.91$\pm$1.04 & 5.14$\pm$0.94 & 5.21$\pm$1.01 \\
7 to 9 times  & 5.35$\pm$1.01 & 5.16$\pm$0.90 & 5.46$\pm$0.85 & 5.36$\pm$0.66 \\
over 10 times & 5.17$\pm$1.10& 5.27$\pm$0.95 & 5.39$\pm$0.82 & 5.5$\pm$0.66 \\           
\hline
p-value&           0.000&         0.002&         0.000&   0.000   \\
 \hline
  \end{tabular}
\end{table}

\begin{table}[h]
  \caption{Chi-square test result of Japanese Level*Q17}
  \label{table32d}
  \centering
  \begin{tabular}{l|cccc}
 \hline
        \multicolumn{1}{c|}{Japanese Level}          & Q17\_1               & Q17\_2 & Q17\_3    & Q17\_4        \\
\hline
Cannot understand & 3.96$\pm$1.79& 4.4$\pm$1.44 & 4.44$\pm$1.42 & 4.56$\pm$1.61 \\
Basic             & 4.64$\pm$1.30 & 4.91$\pm$1.04 & 5.04$\pm$1.03  & 5.08$\pm$1.06  \\
Intermediate      & 4.99$\pm$1.16 & 5.04$\pm$1.05 & 5.28$\pm$0.90 & 5.15$\pm$0.90  \\
Up level          & 5.06$\pm$1.37& 5.05$\pm$1.06 & 5.05$\pm$0.93 & 5.29$\pm$0.89 \\        
\hline
p-value&           0.000&         0.032&         0.000&   0.021   \\
 \hline
  \end{tabular}
\end{table}

\begin{table}[h]
  \caption{Chi-square test result of severity of the earthquake experienced*Q17}
  \label{table32e}
  \centering
  \begin{tabular}{l|cccc}
 \hline
        \multicolumn{1}{c|}{\begin{tabular}{c}Severity of the\\earthquake experienced\end{tabular}}          & Q17\_1               & Q17\_2 & Q17\_3    & Q17\_4       \\
\hline
\begin{tabular}{l}no experience\end{tabular}  & 4.57$\pm$0.94 & 5.07$\pm$1.07  & 5.00$\pm$0.78   & 5.00$\pm$0.96 \\
\begin{tabular}{l}MMI intensity 5 or less / \\intensity 3 or less\end{tabular} & 4.71$\pm$1.27  & 4.87$\pm$1.07 & 5.14$\pm$0.97 & 5.21$\pm$0.96 \\
\begin{tabular}{l}MMI intensity 6 /\\ intensity 4\end{tabular}                 & 4.83$\pm$1.23  & 5.04$\pm$1.04 & 5.13$\pm$0.95& 5.28$\pm$0.85 \\
\begin{tabular}{l}MMI intensity 7 /\\ intensity 5 weak\end{tabular}            & 4.82$\pm$1.40  & 4.97$\pm$1.16 & 5.05$\pm$1.10 & 5.01$\pm$1.20  \\
\begin{tabular}{l}MMI intensity 8 /\\ intensity 5 strong\end{tabular}          & 4.63$\pm$1.56  & 4.74$\pm$1.16 & 4.93$\pm$1.12 & 4.86$\pm$0.95  \\
\begin{tabular}{l}MMI intensity 9 /\\ intensity 6 weak\end{tabular}            & 4.97$\pm$1.03 & 4.87$\pm$0.73 & 5.20$\pm$1.00 & 5.03$\pm$1.10  \\
\begin{tabular}{l}MMI intensity 10 /\\ intensity 6 strong\end{tabular}         & 5.17$\pm$1.25& 5.11$\pm$1.08 & 5.17$\pm$0.92& 5.39$\pm$0.61 \\
\begin{tabular}{l}MMI intensity 11 to 12 /\\ intensity 7\end{tabular}          & 5.10$\pm$1.00 & 5.43$\pm$0.82& 5.47$\pm$0.68 & 5.23$\pm$1.19 \\ 
\hline
\begin{tabular}{l}p-value\end{tabular}&           0.548&         0.127&         0.389&   0.121   \\
 \hline
  \end{tabular}
\end{table}

Thus, of the six manifest variables mentioned in the hypothesis underlying our construction of SEM in Chapter~\ref{c4}, the differences in gender and severity of experienced earthquakes did not show a significant effect in the responses of Attitude toward Safety Tips. This also means that these two manifest variables would not need to be considered in SEM. Therefore, the four manifest variables used in the SEM were age, number of visited countries, number of visited Japan, and Japanese level. Among them, only the ANOVA results of the manifest variables number of visited Japan and Japanese level showed that the differences were significant for all four attitude related questions. The results of age, the number of visited countries, and other manifest variables only show that the differences are significant for some of the questions, not for all of them. However, since the results cannot be deleted directly, after all, there are some questions with good significance test results, so they are still retained in the SEM.
\cleardoublepage
\subsection{Test for latent variables }
The results of the correlation test of three latent variables which are Disaster prevention consciousness, Training experience, Knowledge, and perception of earthquakes are presented in Table~\ref{table33}. As ** means significant correlation at $p<0.01$ level, so from the results we can know that there are significant correlations between Disaster prevention consciousness and Knowledge and perception on earthquakes, also Knowledge and perception on earthquakes and Training experience.

\begin{table}[h]
  \caption{Correlation test result of latent variiables}
  \label{table33}
  \centering
\begin{tabular}{c|ccc}
\hline
         & Consciousness & Knowledge & Training\_Experience \\
\hline
Consciousness        & 1             &           &                     \\
Knowledge            & .669**        & 1         &                     \\
Training\_Experience   & 0.056         & .352**    & 1            \\
\hline
\multicolumn{4}{l}{** means significant correlation at $p<0.01$.}
\end{tabular}
\end{table}


\subsection{Reliability }



The survey selected for this study contained three sets of questions, which are Disaster Prevention Consciousness, Knowledge and Perception on earthquakes, and Attitude toward Safety Tips. Disaster Prevention Consciousness contains 5 sub-problems with a total of 20 questions. Knowledge and Perception on earthquakes contain 2 sub-questions with a total of 15 items. Attitude toward Safety Tips contains 4 sub-questions with a total of 4 items. In order to ensure the internal consistency of the scale, it is necessary to pass the reliability test first before conducting CFA. The internal consistency was tested by calculating the internal consistency reliability coefficient Cronbach's alpha value of the scale. When multiple questions are asked about a characteristic and the sum of the responses (scale scores) is used as the characteristic scale, the reliability coefficient that assesses whether each questionnaire item (variable) measures the same concept or object as a whole (internal consistency) is called Cronbach's alpha. Cronbach's alpha is calculated by the following Formula~\ref{for1}.

\begin{equation}
\label{for1}
\alpha = \frac{m}{m-1} \left(1 - \frac{\displaystyle \sum_{i = 1}^m{{\sigma_i}^2}}{{\sigma_x}^2} \right)
\end{equation}
$m$ means the number of items in the question; ${\sigma_i}^2$ means the variance of each question item; ${\sigma_x}^2$ means the variance of the total scale score for each question item. Cronbach's alpha has a value between 0 and 1, and the closer the value is to 1, the more reliable it is. The evaluation of Cronbach's alpha is shown in Figure~\ref{fig24}, as suggested by George and Mallery (2003).~\cite{ref1}. Cronbach's alpha  Cronbach's alpha of 0.9 indicates excellent internal consistency, 0.8 indicates good, 0.7 indicates acceptable, 0.6 indicates poor, and 0.5 unacceptable. The results of the reliability test are shown in Table~\ref{table34}, the Cronbach's alpha values of Disaster Prevention Consciousness is 0.902, for Knowledge and Perception on earthquakes is 0.952, and for Attitude toward Safety Tips is 0.875. The total Cronbach's alpha value for all above is 0.955. We can find that the Cronbach's alpha value of any one of them is satisfying the evaluation criteria.




\begin{table}[h]
  \caption[Reliability test]{Reliability test (N=491)}
  \label{table34}
  \centering
\begin{tabular}{c|c|cc|c|c}
\hline
 \multicolumn{2}{c}{}           & Question & \multicolumn{1}{c}{Mean}      & \multicolumn{1}{c}{Object number}         & \begin{tabular}{c}Cronbach's\\alpha value\end{tabular}   \\
\hline
                                                           &                                                             & Q1\_1    & 4.89$\pm$1.21 &                       &                          \\
                                                           &                                                             & Q1\_2    & 4.94$\pm$1.10  &                       &                          \\
                                                           &                                                             & Q1\_3    & 4.99$\pm$1.14 &                       &                          \\
                                                           & \multirow{-4}{*}{\begin{tabular}{c}Disastrous\\Imagination\end{tabular}}                    & Q1\_4    & 4.94$\pm$1.18 &                       &                          \\
\cline{2-4}
                                                           &                                                             & Q2\_1    & 4.85$\pm$1.29 &                       &                          \\
                                                           &                                                             & Q2\_2    & 4.96$\pm$1.21 &                       &                          \\
                                                           &                                                             & Q2\_3    & 4.93$\pm$1.19 &                       &                          \\
                                                           & \multirow{-4}{*}{\begin{tabular}{c}Sense of\\crisis\end{tabular}}                           & Q2\_4    & 5.07$\pm$1.12 &                       &                          \\
\cline{2-4}
                                                           &                                                             & Q3\_1    & 5.12$\pm$1.11 &                       &                          \\
                                                           &                                                             & Q3\_2    & 4.97$\pm$1.20  &                       &                          \\
                                                           &                                                             & Q3\_3    & 5.13$\pm$1.11 &                       &                          \\
                                                           & \multirow{-4}{*}{\begin{tabular}{c}Other-directed\\type\end{tabular}}                       & Q3\_4    & 5.11$\pm$1.09 &                       &                          \\
\cline{2-4}
                                                           &                                                             & Q4\_1    & 4.05$\pm$1.59 &                       &                          \\
                                                           &                                                             & Q4\_2    & 4.25$\pm$1.52 &                       &                          \\
                                                           &                                                             & Q4\_3    & 4.87$\pm$1.19 &                       &                          \\
                                                           & \multirow{-4}{*}{Anxiety}                                   & Q4\_4    & 5.04$\pm$1.11 &                       &                          \\
\cline{2-4}
                                                           &                                                             & Q5\_1    & 3.81$\pm$1.73 &                       &                          \\
                                                           &                                                             & Q5\_2    & 4.00$\pm$1.55  &                       &                          \\
                                                           &                                                             & Q5\_3    & 3.59$\pm$1.66 &                       &                          \\
\multirow{-20}{*}{\begin{tabular}{c}Disaster\\Prevention\\Consciousness\end{tabular}}       & \multirow{-4}{*}{\begin{tabular}{c}Apathy about\\disasters\end{tabular}}                    & Q5\_4    & 3.92$\pm$1.63 & \multirow{-20}{*}{20} & \multirow{-20}{*}{0.902} \\
\hline
                                                           &                                                             & Q9\_1    & 4.85$\pm$1.29 &                       &                          \\
                                                           &                                                             & Q9\_2    & 4.88$\pm$1.12 &                       &                          \\
                                                           &                                                             & Q9\_3    & 4.90$\pm$1.12  &                       &                          \\
                                                           &                                                             & Q9\_4    & 4.87$\pm$1.10  &                       &                          \\
                                                           &                                                             & Q9\_5    & 4.82$\pm$1.15 &                       &                          \\
                                                           & \multirow{-6}{*}{\begin{tabular}{c}Knowledge\\about\\earthquakes\end{tabular}}               & Q9\_6    & 4.94$\pm$1.11 &                       &                          \\
\cline{2-4}
                                                           &                                                             & Q10\_1   & 5.03$\pm$1.27 &                       &                          \\
                                                           &                                                             & Q10\_2   & 4.86$\pm$1.22 &                       &                          \\
                                                           &                                                             & Q10\_3   & 4.97$\pm$1.15 &                       &                          \\
                                                           &                                                             & Q10\_4   & 5.05$\pm$1.08 &                       &                          \\
                                                           &                                                             & Q10\_5   & 5.05$\pm$1.09 &                       &                          \\
                                                           &                                                             & Q10\_6   & 5.01$\pm$1.12 &                       &                          \\
                                                           &                                                             & Q10\_7   & 4.98$\pm$1.12 &                       &                          \\
                                                           &                                                             & Q10\_8   & 5.15$\pm$1.02 &                       &                          \\
\multirow{-15}{*}{\begin{tabular}{c}Knowledge and\\Perception on\\earthquakes\end{tabular}} & \multirow{-9}{*}{\begin{tabular}{c}Knowledge of\\how to\\respond to\\a disaster\end{tabular}} & Q10\_9   & 5.01$\pm$1.08 & \multirow{-15}{*}{15} & \multirow{-15}{*}{0.952} \\
\hline
                                                           & Trust level                                                 & Q17\_1   & 4.80$\pm$1.31  &                       &                          \\
\cline{2-4}
                                                           & Priority of use & Q17\_2   & 4.95$\pm$1.08 &                       &                          \\
\cline{2-4}
                                                           & Usefulness       & Q17\_3   & 5.10$\pm$1.01  &                       &                          \\
\cline{2-4}
\multirow{-4}{*}{\begin{tabular}{c}Attitude\\toward\\Safety Tips\end{tabular}}              & \begin{tabular}{c}Possilibity\\of future use\end{tabular}    & Q17\_4   & 5.11$\pm$1.02 & \multirow{-4}{*}{4}   & \multirow{-4}{*}{0.875}  \\
\hline
\multicolumn{1}{c}{Total}                    &                                   \multicolumn{3}{c}{}      & \multicolumn{1}{c}{39}                    & 0.955     \\
\hline              
\end{tabular}
\end{table}


\begin{figure*}[h]
  \includegraphics[width=0.5\linewidth]{Figure/Figure24.jpg}
  \centering
  \caption{Evaluation of score reliability coefficient test}
  \label{fig24}
\end{figure*}

\cleardoublepage
\subsection{SEM model 1}
\subsubsection{Path coefficient analysis for model 1}

SEM model 1 is shown in Figure~\ref{fig23}. The result of Regression Weights for Model 1 is shown in Table~\ref{table9}, and the result of Standard Regression Weights for Model 1 is shown in Table~\ref{table10}. From the result, we can find that Disaster Prevention Consciousness, Knowledge and Perception on earthquakes and Training Experience could show significant relationships with respondents' attitude toward Safety Tips, as p values are all less than 0.001. Also, Knowledge and Perception on earthquakes could show a positive relationship with respondents' attitude toward Safety Tips, as the estimated values are positive numbers (estimate value of Knowledge=3.066). While, Disaster Prevention Consciousness and Training Experience could show a negative relationship with respondents' attitude toward Safety Tips, as the estimate values are negative number (estimate value of Consciousness$=-2.191$; estimate value of Training Experience$=-0.433$;) 

\begin{figure*}[h]
  \includegraphics[width=0.5\linewidth]{Figure/Figure23.jpg}
  \centering
  \caption{SEM model 1}
  \label{fig23}
\end{figure*}

For four manifest variables age, number of visited Japan, number of visited countries, and Japanese level don't show significant relationships with respondents' attitudes toward Safety Tips, as p values are larger than 0.001 ($p_{age} =0.633$; $p_{visitedcountries} =0.628$; $p_{visitedJapan} =0.071$; $p_{JapaneseLevel} =0.265$). 


\begin{table}[h]
  \caption{Regression Weights of SEM model 1 }
  \label{table9}
  \centering
  \begin{tabular}{lcl|c|c}
 \hline
 \multicolumn{3}{c|}{Regression Weights} & Estimate & p-value \\
 \hline
SafetyTips              &$\longleftarrow$ & age                  & 0.013  & 0.633                \\
SafetyTips              &$\longleftarrow$ & Training\_Experience & -0.501 & ***                  \\
SafetyTips              &$\longleftarrow$ & Consciousness        & -2.785 & ***                  \\
SafetyTips              &$\longleftarrow$ & Knowledge            & 4.336  & ***                  \\
SafetyTips              &$\longleftarrow$ & VisitCountry         & -0.012 & 0.628                \\
SafetyTips              &$\longleftarrow$ & Japanese\_Level      & -0.049 & 0.265                \\
SafetyTips              &$\longleftarrow$ & VisitJapan           & 0.041  & 0.071                \\
Q1                      &$\longleftarrow$ & Consciousness        & 1      &  \\
Q2                      &$\longleftarrow$ & Consciousness        & 1.028  & ***                  \\
Q3                      &$\longleftarrow$ & Consciousness        & 1.029  & ***                  \\
Q4                      &$\longleftarrow$ & Consciousness        & 0.72   & ***                  \\
Q5                      &$\longleftarrow$ & Consciousness        & 0.157  & 0.052                \\
Q6\_1\_earthquake       &$\longleftarrow$ & Training\_Experience & 3.234  & ***                  \\
Q6\_2\_tsunami          &$\longleftarrow$ & Training\_Experience & 3.070   & ***                  \\
Q6\_3\_typhoon          &$\longleftarrow$ & Training\_Experience & 3.178  & ***                  \\
Q6\_4\_fire             &$\longleftarrow$ & Training\_Experience & 3.388  & ***                  \\
Q7\_1\_earthquake       &$\longleftarrow$ & Training\_Experience & 0.685  & ***                  \\
Q7\_2\_tsunami          &$\longleftarrow$ & Training\_Experience & 0.682  & ***                  \\
Q7\_3\_typhoon          &$\longleftarrow$ & Training\_Experience & 0.777  & ***                  \\
Q7\_4\_fire             &$\longleftarrow$ & Training\_Experience & 1      & \\
Q17Safetytips\_trust\_1 &$\longleftarrow$ & SafetyTips           & 1      &  \\
Q17Safetytips\_trust\_2 &$\longleftarrow$ & SafetyTips           & 0.826  & ***                  \\
Q17Safetytips\_trust\_3 &$\longleftarrow$ & SafetyTips           & 0.786  & ***                  \\
Q17Safetytips\_trust\_4 &$\longleftarrow$ & SafetyTips           & 0.688  & ***                 \\
 \hline
\multicolumn{5}{l}{*** means significant correlation at $p<0.001$.}
  \end{tabular}
\end{table}

\begin{table}[h]
  \caption{Standardized Regression Weights of SEM model 1 }
  \label{table10}
  \centering
  \begin{tabular}{lcl|c}
 \hline
 \multicolumn{3}{c|}{Standardized Regression Weights} & Estimate  \\
 \hline
SafetyTips              &$\longleftarrow$ & age                  & 0.015  \\
SafetyTips              &$\longleftarrow$ & Training\_Experience & -0.433 \\
SafetyTips              &$\longleftarrow$ & Consciousness        & -2.191 \\
SafetyTips              &$\longleftarrow$ & Knowledge            & 3.066  \\
SafetyTips              &$\longleftarrow$ & VisitCountry         & -0.016 \\
SafetyTips              &$\longleftarrow$ & Japanese\_Level      & -0.036 \\
SafetyTips              &$\longleftarrow$ & VisitJapan           & 0.058  \\
Q1                      &$\longleftarrow$ & Consciousness        & 0.809  \\
Q2                      &$\longleftarrow$ & Consciousness        & 0.843  \\
Q3                      &$\longleftarrow$ & Consciousness        & 0.855  \\
Q4                      &$\longleftarrow$ & Consciousness        & 0.576  \\
Q5                      &$\longleftarrow$ & Consciousness        & 0.092  \\
Q6\_1\_earthquake       &$\longleftarrow$ & Training\_Experience & 0.786  \\
Q6\_2\_tsunami          &$\longleftarrow$ & Training\_Experience & 0.838  \\
Q6\_3\_typhoon          &$\longleftarrow$ & Training\_Experience & 0.842  \\
Q6\_4\_fire             &$\longleftarrow$ & Training\_Experience & 0.758  \\
Q7\_1\_earthquake       &$\longleftarrow$ & Training\_Experience & 0.388  \\
Q7\_2\_tsunami          &$\longleftarrow$ & Training\_Experience & 0.489  \\
Q7\_3\_typhoon          &$\longleftarrow$ & Training\_Experience & 0.548  \\
Q7\_4\_fire             &$\longleftarrow$ & Training\_Experience & 0.566  \\
Q9                      &$\longleftarrow$ & Knowledge            & 0.785  \\
Q10                     &$\longleftarrow$ & Knowledge            & 0.804  \\
Q17Safetytips\_trust\_1 &$\longleftarrow$ & SafetyTips           & 0.816  \\
Q17Safetytips\_trust\_2 &$\longleftarrow$ & SafetyTips           & 0.822  \\
Q17Safetytips\_trust\_3 &$\longleftarrow$ & SafetyTips           & 0.834  \\
Q17Safetytips\_trust\_4 &$\longleftarrow$ & SafetyTips           & 0.717  \\
 \hline
  \end{tabular}
\end{table}

The result of the correlation relationship between Disaster Prevention Consciousness with Knowledge and Perception on earthquakes, also Knowledge and Perception on earthquakes with Training Experience was shown in Table~\ref{table13} and Table~\ref{table14}. The result confirmed that Disaster Prevention Consciousness with Knowledge and Perception on earthquakes, also Knowledge and Perception on earthquakes with Training Experience could both show a significant correlation, as p values are all less than 0.001. And both are positive correlations, as the estimated values are positive numbers (estimate the value of Consciousness and Knowledge=0.961; estimate the value of Training Experience and Knowledge=0.18;). Among them, Disaster Prevention Consciousness with Knowledge and Perception on earthquakes could show a higher correlation than Knowledge and Perception on earthquakes with Training Experience. 

\begin{table}[h]
  \caption{Covariances of SEM model 1}
  \label{table13}
  \centering
  \begin{tabular}{lcl|c|c}
  \hline
   \multicolumn{3}{c|}{Covariances} & Estimate & p-value \\
  \hline
  Consciousness & $\longleftrightarrow$ & Knowledge & 0.589 & *** \\
  Training\_Experience & $\longleftrightarrow$ & Knowledge & 0.121 & *** \\
  \hline
\multicolumn{5}{l}{*** means significant correlation at $p<0.001$.}
  \end{tabular}
\end{table}

\begin{table}[h]
  \caption{Correlations of SEM model 1}
  \label{table14}
  \centering
  \begin{tabular}{lcl|c}
  \hline
   \multicolumn{3}{c|}{Correlations} & Estimate \\
  \hline
  Consciousness & $\longleftrightarrow$ & Knowledge & 0.961 \\
  Training\_Experience & $\longleftrightarrow$ & Knowledge & 0.180 \\
  \hline
  \end{tabular}
\end{table}
\cleardoublepage
\subsubsection{Model Fit Test for SEM model 1}
The model fit result of Model 1 is shown in Table~\ref{table15}. From the results, CMIN/DF equals 8.352, an indicator that reflects model variability RMSEA equals 0.122, and an indicator that reflects model similarity CFI equals 0.738. Referring to some of the criteria mentioned in subsection~\ref{step6}, we can find that the model fit test of model 1 is not up to standard. Therefore, we need to modify model 1.

\begin{table}[h]
  \caption{Model fit test for SEM model 1}
  \label{table15}
  \centering 
  \begin{tabular}{|c|}
  \hline
  RMSEA = 0.122 \\
  CFI = 0.738 \\
  CMIN/DF = 8.352 \\
  \hline
  \end{tabular}
\end{table}

\subsubsection{Model adjustment and modification for model 1 }

We can get a better estimate of the true correlation by disattenuated the variables, and according to D. Streiner (2006)~\cite{Streiner2006BuildingAB}, two things happen when we add extra variables. First, the model's ability to account for more variance grows. Each new variable, on the other hand, enhances the error variance. As a result, the previous model will struggle to accommodate the additional data, which is a consequence of adding more variables to the model. 

We can see that Q6 1/2/3/4 and Q7 1/2/3/4 both have a significant relationship with Training Experience based on the results of Table~\ref{table9} Regression Weights. However, the estimated values of Q7 (Q7\_1=0.388; Q7\_2=0.489; Q7\_3=0.548; Q7\_4=0.566) are much lower than the estimated value of Q6 (Q6\_1=0.786; Q6\_2=0.838; Q6\_3=0.842; Q7\_4=0.758) , as seen in Table~\ref{table10} Standardized Regression Weights. This means that Q6 could be better than Q7 at expressing the latent variable Training Experience. The two manifest variables are similar in structure: Q6 is for the experience of a given event, and the total score is used as data, whereas Q7 is for the number of times. Because the similarity of the two variables causes a significant amount of inaccuracy in the expression of the latent variable, we'll delete Q7 and keep only Q6 as the manifest variable to express latent variable Training experience as a way to improve the model. 

Another point that could be improved in Q5. From the regression weights results presented in Table~\ref{table9}, we can find that the only manifest variable that does not significant is Q5, as the p-value of Q5 with Disaster Prevention Consciousness is $p=0.052>0.001$. This means that Q5 does not work well as a manifest variable for latent variables Disaster Prevention Consciousness. Therefore, model 2 will remove Q5 and use only Q1 to Q4 as the manifest variables.

Then, since the results show that manifest variables age, number of visited Japan, number of visited countries, and Japanese level have no significant relationship with Attitude toward Safety Tips, deleting these four variables could be also a way to improve the model. 
Based on the above three improvement ideas shown on the left side of Figure~\ref{fig25}, we construct Model 2 shown on the right side of Figure~\ref{fig25}.

\begin{figure*}[t]
  \includegraphics[width=\linewidth]{Figure/Figure25.png}
  \centering
  \caption[SEM model 2]{Left side: the diagram contains the improvement points made based on Model 1,Right side: Model 2}
  \label{fig25}
\end{figure*}




\subsection{SEM model 2}

\subsubsection{Model Fit Test for SEM model 2}
The model fit result of Model 2 is shown in Table~\ref{table16}. From the results, CMIN/DF equals 5.620, an indicator that reflects model variability RMSEA equals 0.097, an indicator that reflects model similarity CFI equals 0.925.

\begin{table}[h]
  \caption{Model fit test for SEM model 2}
  \label{table16}
  \centering 
  \begin{tabular}{|c|}
  \hline
  RMSEA = 0.097 \\
  CFI = 0.925 \\
  CMIN/DF = 5.620 \\
  \hline
  \end{tabular}
\end{table}

\subsubsection{Path coefficient analysis for model 2}
The result of Regression Weights for Model 2 is shown in Table~\ref{table11}, and the result of Standard Regression Weights for Model 2 is shown in Table~\ref{table12}. From the result, we can find that Disaster Prevention Consciousness and Knowledge and Perception on earthquakes could show significant relationships with respondents' attitude toward Safety Tips, as p values are both less than 0.001. But in model 2, Training Experience doesn't show significant relationships with respondents' attitude toward Safety Tips, as p values are larger than 0.001 ($p=0.002>0.001$). In other words, the regression weight for Training\_Experience in the prediction of SafetyTips is significantly different from zero at the 0.01 level (two-tailed). Then, from the result, we can find that Disaster Prevention Consciousness and Knowledge and Perception on earthquakes show opposite effects on respondents' attitude toward Safety Tips. Disaster Prevention Consciousness could show a negative relationship with respondents' attitude toward Safety Tips, as the estimated values are negative numbers (estimate value of Consciousness=-2.253). Knowledge and Perception on earthquakes could show a positive relationship with respondents' attitude toward Safety Tips, as the estimated values are positive numbers (estimate value of Knowledge=3.127). Also, we can find that all manifest variables could significantly express latent variables, as the p-value of all manifest variables could be less than 0.001. 

\begin{table}[t]
  \caption{Regression Weights of SEM model 2 }
  \label{table11}
  \centering
  \begin{tabular}{lcl|c|c}
 \hline
 \multicolumn{3}{c|}{Regression Weights} & Estimate & p-value \\
 \hline
SafetyTips              &$\longleftarrow$ & Training\_Experience & -0.147 & 0.002                \\
SafetyTips              &$\longleftarrow$ & Consciousness        & -2.876 & ***                  \\
SafetyTips              &$\longleftarrow$ & Knowledge            & 4.441  & ***                  \\
Q1                      &$\longleftarrow$ & Consciousness        & 1      &  \\
Q2                      &$\longleftarrow$ & Consciousness        & 1.027  & ***                  \\
Q3                      &$\longleftarrow$ & Consciousness        & 1.031  & ***                  \\
Q4                      &$\longleftarrow$ & Consciousness        & 0.715  & ***                  \\
Q6\_1\_earthquake       &$\longleftarrow$ & Training\_Experience & 1      &  \\
Q6\_2\_tsunami          &$\longleftarrow$ & Training\_Experience & 0.942  & ***                  \\
Q6\_3\_typhoon          &$\longleftarrow$ & Training\_Experience & 0.964  & ***                  \\
Q6\_4\_fire             &$\longleftarrow$ & Training\_Experience & 1.047  & ***                  \\
Q9                      &$\longleftarrow$ & Knowledge            & 1      &  \\
Q10                     &$\longleftarrow$ & Knowledge            & 1.003  & ***                  \\
Q17Safetytips\_trust\_1 &$\longleftarrow$ & SafetyTips           & 1      &  \\
Q17Safetytips\_trust\_2 &$\longleftarrow$ & SafetyTips           & 0.827  & ***                  \\
Q17Safetytips\_trust\_3 &$\longleftarrow$ & SafetyTips           & 0.786  & ***                  \\
Q17Safetytips\_trust\_4 &$\longleftarrow$ & SafetyTips           & 0.688  & ***                 \\
 \hline
\multicolumn{5}{l}{*** means significant correlation at $p<0.001$.}
  \end{tabular}
\end{table}

\begin{table}[h]
  \caption{Standardized Regression Weights of SEM model 2 }
  \label{table12}
  \centering
  \begin{tabular}{lcl|c}
 \hline
 \multicolumn{3}{c|}{Standardized Regression Weights} & Estimate  \\
 \hline
SafetyTips              &$\longleftarrow$ & Training\_Experience & -0.414 \\
SafetyTips              &$\longleftarrow$ & Consciousness        & -2.253 \\
SafetyTips              &$\longleftarrow$ & Knowledge            & 3.127  \\
Q1                      &$\longleftarrow$ & Consciousness        & 0.809  \\
Q2                      &$\longleftarrow$ & Consciousness        & 0.842  \\
Q3                      &$\longleftarrow$ & Consciousness        & 0.856  \\
Q4                      &$\longleftarrow$ & Consciousness        & 0.572  \\
Q6\_1\_earthquake       &$\longleftarrow$ & Training\_Experience & 0.792  \\
Q6\_2\_tsunami          &$\longleftarrow$ & Training\_Experience & 0.838  \\
Q6\_3\_typhoon          &$\longleftarrow$ & Training\_Experience & 0.833  \\
Q6\_4\_fire             &$\longleftarrow$ & Training\_Experience & 0.763  \\
Q9                      &$\longleftarrow$ & Knowledge            & 0.785  \\
Q10                     &$\longleftarrow$ & Knowledge            & 0.804  \\
Q17Safetytips\_trust\_1 &$\longleftarrow$ & SafetyTips           & 0.817  \\
Q17Safetytips\_trust\_2 &$\longleftarrow$ & SafetyTips           & 0.824  \\
Q17Safetytips\_trust\_3 &$\longleftarrow$ & SafetyTips           & 0.835  \\
Q17Safetytips\_trust\_4 &$\longleftarrow$ & SafetyTips           & 0.718  \\
 \hline
  \end{tabular}
\end{table}

The result of the correlation relationship between Disaster Prevention Consciousness with Knowledge and Perception on earthquakes, also Knowledge and Perception on earthquakes with Training Experience was shown in Table~\ref{table21} and Table~\ref{table22}. The result confirmed that Disaster Prevention Consciousness with Knowledge and Perception on earthquakes, also Knowledge and Perception on earthquakes with Training Experience could still both show a significant correlation, as p-values are all less than 0.001. And both are positive correlations, as the estimated values are positive numbers (estimate the value of Consciousness and Knowledge=0.963; estimate the value of Training Experience and Knowledge=0.175;). Among them, Knowledge and Perception on earthquakes with Disaster Prevention Consciousness could show a higher correlation than it with Training Experience, which is as same as in SEM model 1.

\begin{table}[h]
  \caption{Covariances of SEM model 2}
  \label{table21}
  \centering
  \begin{tabular}{lcl|c|c}
  \hline
   \multicolumn{3}{c|}{Covariances} & Estimate & p-value \\
  \hline
  Consciousness & $\longleftrightarrow$ & Knowledge & 0.590 & *** \\
  Training\_Experience & $\longleftrightarrow$ & Knowledge & 0.383 & *** \\
  \hline
\multicolumn{5}{l}{*** means significant correlation at $p<0.001$.}
  \end{tabular}
\end{table}

\begin{table}[h]
  \caption{Correlations of SEM model 2}
  \label{table22}
  \centering
  \begin{tabular}{lcl|c}
  \hline
   \multicolumn{3}{c|}{Correlations} & Estimate \\
  \hline
  Consciousness & $\longleftrightarrow$ & Knowledge & 0.963 \\
  Training\_Experience & $\longleftrightarrow$ & Knowledge & 0.175 \\
  \hline
  \end{tabular}
\end{table}
\cleardoublepage
\subsubsection{Confirmatory Factor Analysis (CFA)}
CFA include Construct Validity, Convergent Validity and Discriminant Validity, as explained in subsection~\ref{step5}.

\textbf{Construct Validity.} Referring to some of the criteria mentioned in subsection~\ref{step6}, from the model fit test From the model fit result shown in Table~\ref{table16}, we can find that the CFI estimate is satisfied in model 2, as a CFI of 0.9 or higher can be considered a good match~\cite{ref43,ref44}. And according to Schumacker and Lomax (2004)~\cite{ref39}, they suggested a more lenient fit around 5 or less, and the current CMIN/DF is near to 5, but a little bit higher than 5. Finally for RMSEA, according to Takahiro HOSHINO (2005)  (SEMres), as current RMSEA$=0.097<0.1$, the model shows a moderate fit. For measuring model Disaster Prevention Consciousness shown in Figure~\ref{fig31}, the results of Standardized Regression Weights (SRW) and Squared Multiple Correlations (SMC) are shown in Table~\ref{table24} and Table~\ref{table25}. From the results, we can find that all the SRW values are greater than 0.55, which means that SMC is greater than 0.30, satisfying the criteria mentioned in subsection~\ref{s7}, proving good. The model fit result of measuring model Disaster Prevention Consciousness is shown in Table~\ref{table26}. From the results, we can find that all model fit estimates are highly satisfying the criteria mentioned in subsection~\ref{step6}, as CMIN/DF equals 2.963, RMSEA equals 0.063, CFI equals 0.996. 


\begin{figure*}[t]
  \includegraphics[width=0.5\linewidth]{Figure/figure31.JPG}
  \centering
  \caption{Measuring model of Consciousness}
  \label{fig31}
\end{figure*}

\begin{table}[h]
  \caption{Standardized Regression Weights for measuring model Consiousness}
  \label{table24}
  \centering
  \begin{tabular}{lcl|c}
  \hline
   \multicolumn{3}{c|}{Standardized Regression Weights} & Estimate \\
  \hline
  Q1 & $\longleftarrow$ & Consciousness & 0.807 \\
  Q2 & $\longleftarrow$ & Consciousness & 0.872 \\
  Q3 & $\longleftarrow$ & Consciousness & 0.835 \\
  Q4 & $\longleftarrow$ & Consciousness & 0.574 \\
  \hline
  \end{tabular}
\end{table}

\begin{table}[h]
  \caption{Squared Multiple Correlations for Consciousness}
  \label{table25}
  \centering
  \begin{tabular}{c|c}
  \hline
   Squared Multiple Correlations & Estimate \\
  \hline
  Q1  & 0.652 \\
  Q2  & 0.760 \\
  Q3  & 0.698 \\
  Q4  & 0.330 \\
  \hline
  \end{tabular}
\end{table}

\begin{table}[h]
  \caption{Model fit test for measuring model Consciousness}
  \label{table26}
  \centering 
  \begin{tabular}{|c|}
  \hline
  RMSEA = 0.063 \\
  CFI = 0.996 \\
  CMIN/DF = 2.963 \\
  \hline
  \end{tabular}
\end{table}

\textbf{Convergent Validity.} As mentioned in subsection~\ref{step5}, the commonly acceptable evaluation criterion of CR value in research should be above 0.7~\cite{ref32}. The result of the CR calculation is shown in Table~\ref{table35}. From the result, we can know that CR\_Consciousness$=0.857>0.7$; CR\_Training\_Experience$=0.882>0.7$; CR\_Knowledge$=1.001>0.7$; CR\_Attitude$=0.876>0.7$; This CR result can prove that the internal consistency of SEM is good. Also, according to Fornell and Larcker (1981)~\cite{ref31}, the commonly acceptable evaluation criterion of AVE value in research needs to be greater than 0.5. The result of the AVE calculation is also shown in Table~\ref{table35}. From the result, we can know that AVE\_Consciousness$=0.606>0.5$; AVE\_Training\_Experience$=0.651>0.5$; AVE\_Knowledge$=1.003>0.5$; AVE\_Attitude$=0.640>0.5$; This AVE result can prove that the explanatory power of the latent variables on the observed variables is good. In summary, the CR result and the AVE result show that SEM model 2 passed the Convergent Validity test. 

\begin{table}[h]
  \caption{Convergent Validity}
  \label{table35}
  \centering
  \begin{tabular}{lcl|cc|ccc}
  \hline
   & & & std.   & p-value           & SMC                  & CR                     & AVE                    \\
SafetyTips           & $\longleftarrow$       & Training\_Experience & -0.414 & 0.002                & \multicolumn{3}{l}{}    \\
SafetyTips           & $\longleftarrow$       & Consciousness        & -2.253 & ***                  & \multicolumn{3}{l}{} \\
SafetyTips           & $\longleftarrow$       & Knowledge            & 3.127  & ***                  & \multicolumn{3}{l}{}   \\
\hline
Q1                   & $\longleftarrow$       & Consciousness        & 0.809  &  & 0.654                & \multirow{4}{*}{0.857} & \multirow{4}{*}{0.606} \\
Q2                   & $\longleftarrow$       & Consciousness        & 0.842  & ***                  & 0.709                &                        &                        \\
Q3                   & $\longleftarrow$       & Consciousness        & 0.856  & ***                  & 0.733                &                        &                        \\
Q4                   & $\longleftarrow$       & Consciousness        & 0.572  & ***                  & 0.327                &                        &                        \\
Q6\_1                & $\longleftarrow$       & Training\_Experience & 0.792  &  & 0.627                & \multirow{4}{*}{0.882} & \multirow{4}{*}{0.651} \\
\hline
Q6\_2                & $\longleftarrow$       & Training\_Experience & 0.838  & ***                  & 0.702                &                        &                        \\
Q6\_3                & $\longleftarrow$       & Training\_Experience & 0.833  & ***                  & 0.694                &                        &                        \\
Q6\_4                & $\longleftarrow$       & Training\_Experience & 0.763  & ***                  & 0.582                &                        &                        \\
Q9                   & $\longleftarrow$       & Knowledge            & 0.785  &  & 0.616                & \multirow{2}{*}{1.001} & \multirow{2}{*}{1.003} \\
\hline
Q10                  & $\longleftarrow$       & Knowledge            & 0.804  & ***                  & 0.646                &                        &                        \\
Q17\_1               & $\longleftarrow$       & SafetyTips           & 0.817  &  & 0.667                & \multirow{4}{*}{0.876} & \multirow{4}{*}{0.64}  \\
\hline
Q17\_2               & $\longleftarrow$       & SafetyTips           & 0.824  & ***                  & 0.679                &                        &                        \\
Q17\_3               & $\longleftarrow$       & SafetyTips           & 0.835  & ***                  & 0.697                &                        &                        \\
Q17\_4               & $\longleftarrow$       & SafetyTips           & 0.718  & ***                  & 0.516                &                        &                       \\
  \hline
\multicolumn{5}{l}{*** means significant correlation at $p<0.001$.}
  \end{tabular}
\end{table}




\textbf{Discriminant Validity.} As mentioned in subsection~\ref{step5}, this study decided to use AVE to compare whether the mean AVE of two latent variables was greater than the squared correlation coefficient of the two latent variables~\cite{ref31} as the method of doing a Discriminant Validity test. Here, for operational convenience, we compare the positive square root of the AVE value of the latent variables and the correlation coefficient of the latent variables. The result of the AVE calculation is shown in Table~\ref{table36}. From the result, we can find that SQRT\_AVE\_Knowledge=1.001, this value is larger than the correlation coefficient of Knowledge with the other three latent variables. Then, SQRT\_AVE\_Consciousness=0.778, this value is larger than the correlation coefficient of Consciousness with Training experience and Attitude, but it is smaller than the correlation coefficient of Consciousness with Knowledge. Next is SQRT\_AVE\_Training\_experience=0.807, this value is larger than the correlation coefficient of Training experience with the other three latent variables. Finally, SQRT\_AVE\_Attitude=0.800, this value is larger than the correlation coefficient of Attitude with Training experience and Consciousness, but it is smaller than the correlation coefficient of Attitude with Knowledge. As Discriminant Validity is to see whether the differentiation between different dimensions and between topics meets the standard. All the problematic items appear on top of Knowledge and Perception on earthquakes, which I think is due to the fact that there are only two manifest variables of this latent variable, which are not enough to accurately represent Knowledge and Perception on earthquakes on the SEM model. This is also a reason that the model fitness is affected in this study due to the limitation of survey data. But in general, the Discriminant Validity of this SEM model 2 could still be fine.

\begin{table}[h]
  \caption{Discriminant Validity }
  \label{table36}
  \centering
\makebox[1 \textwidth][c]{  
\resizebox{1\textwidth}{!}{
\begin{tabular}{c|ccccc}
\hline
     & AVE   & Knowledge                             & Consciousness                         & Training\_Experience                  & \multicolumn{1}{c}{SafetyTips}                          \\
\hline
Knowledge            & 1.003 & \textcolor{red}{\textbf{1.001}} & \multicolumn{1}{l}{}                  & \multicolumn{1}{l}{}                  &                                                         \\
Consciousness        & 0.606 & 0.963***                              & \textcolor{red}{\textbf{0.778}} & \multicolumn{1}{l}{}                  &                                                         \\
Training\_Experience & 0.651 & 0.175***                              & 0                                     & \textcolor{red}{\textbf{0.807}} &                                                         \\
SafetyTips           & 0.64  & 0.885***                              & 0.760***                              & 0.133                                     & \textcolor{red}{\textbf{0.8}}\\
  \hline
\multicolumn{5}{l}{*** means significant correlation at $p<0.001$.}
  \end{tabular}
}}
\end{table}

\subsubsection{Hypothesis testing and conclusion analysis}
\begin{itemize}
\item[\textbf{H1}] Disaster Prevention Consciousness has a positive impact on respondents' attitudes toward Safety Tips.
\item[\textbf{$\longrightarrow$}] This Hypothesis is inconsistent with hypothesis H1. Disaster Prevention Consciousness has a negative impact on respondents' attitudes toward Safety Tips.
\item[\textbf{H2}] Knowledge and Perception of earthquakes have a positive impact on respondents' attitudes toward Safety Tips.
\item[\textbf{$\longrightarrow$}] This Hypothesis is consistent with hypothesis H2. 
\item[\textbf{H3}] Training Experience has a positive impact on respondents' attitudes toward Safety Tips.
\item[\textbf{$\longrightarrow$}] This Hypothesis is inconsistent with hypothesis H3. Training Experience does not have a significant impact on respondents' attitudes toward Safety Tips.
\end{itemize}

\subsubsection{Interpretation}
The following sections will explain how to interpret the aforesaid results. 

Disaster Prevention Consciousness has a negative impact on respondents' attitudes toward Safety Tips: for all those who have a higher consciousness about the disaster, it means they have a higher disaster imagination, sense, and are more likely to feel anxiety about the disaster, and they are more likely to come into contact with people, so they would prefer a more direct way to evacuate rather than seeking information, which makes their attitude toward safety tips could be negative.
Knowledge and Perception of earthquakes have a positive impact on respondents' attitudes toward Safety Tips: people with richer knowledge, are relatively more aware of the importance of information seeking for evacuation. Therefore, their attitudes towards Safety Tips could be better than others.

Training Experience does not have a significant impact on respondents' attitudes toward Safety Tips: respondents with more earthquake or disaster training experience should be more knowledgeable with earthquake evacuation ways, therefore their attitude toward safety suggestions may be negative as well. 




\section{Results for Objective 3}

\subsection{Selected Score and Selected Rate}
The result of the Selected score and the Selected rate is shown in \crefrange{table17}{table20}. From the result, we can find that there are some differences between foreign visitors and Japanese in scenarios 1 and 3, which could show that when the internet and telephone are available, people tend to have various behaviors. By checking the Selected Rate, we can find that evacuation behaviors are more used than information-seeking behaviors. And among the evacuation behaviors, 'Moving according to evacuation guidance' could be used most. This indicates that, regardless of the order factor, 'Moving according to evacuation guidance' is the most favored option. In addition, people are more likely to heed evacuation instructions if they are in the area of such recommendations. As a result, if Safety Tips can provide evacuation instructions, it will attract more people. Furthermore, if it can synchronize the user's location information, more people will use Safety Tips. By checking the Selected Score, we can find that evacuation behaviors always happen before information-seeking behaviors. And among the evacuation behaviors, 'Observe the surroundings' could have happened first. This is compounded by the fact that many people's first instinct in the event of a disaster is to observe others. Not only might they receive some evacuation suggestions, but keeping the same pace as others will make people feel more at ease psychologically.

Some lower used information-seeking behaviors during a disaster are 'Gather Information by calling out to Japanese people nearby', 'Contact staff at tourist Information centers to collect Information', 'Contact public transport staff to collect Information'. As a result, in the case of Internet\&Phone available, people do not choose to acquire information through methods that require verbal conversation, and instead prefer to obtain it on their own. On the other hand, because people can only obtain information through the verbal conversation when the Internet and phone are unavailable, their method of obtaining information depends on the scenario they are in. When people are in a tourist area, they usually ask Japanese people around them for information, and not many people choose the other three options of contacting staff at different spots. However, when people are moving by transportation, people still ask Japanese people around them for information, while contacting staff from public transportation is also a popular option. The lowest used evacuation behavior during a disaster is 'Stay at your current location. This is understandable after all, few people will just stay put and do nothing in the face of a disaster. It is important to note here that this result does not mean that everyone will necessarily do something to leave the place where it happened, but that people's priority evacuation behavior is less likely to be to stay where they are. People will, in most situations, choose to remain where they are after gathering information and following evacuation instructions. This section does not cover such circumstances.

%%%%%%%%%%%%%%%%%%%%%%%%%%
%\iffalse
\begin{table}[h]
  \caption[Result of Selected score and Selected rate in Scenario 1]{Result of Selected score and Selected rate in Scenario 1(No.: number of  selection, FV: Foreign Vistors, J: Japanese)}
  \label{table17}
  \centering 
  \begin{tabular}{cl|ccc|ccc}
                &   & \multicolumn{3}{c}{Selected Rate (\%)} & \multicolumn{3}{c}{Selected Score} \\
      No.     & \multicolumn{1}{c|}{Description} & All & FV & J & All & FV & J \\
 \hline
  1             & \begin{tabular}{l}Collect Information on the official websites\\of Japanese government agencies\end{tabular} & 31.6 &30.2 & 38.7 & 2.9 & 2.9 & 3.0 \\
  2             & \begin{tabular}{l}Collect Information with the disaster\\prevention app on your smartphone\end{tabular} & 27.9 & 25.4 & 40.7 & 2.9 & 2.8 & 3.2 \\
  3             & \begin{tabular}{l}Collect Information on news sites and\\disaster prevention portal sites\end{tabular} & 26.8 & 23.4 & 43.7 & 2.8 & 2.7 & 3.2 \\
  4             & \begin{tabular}{l}Collect Information on SNS\\(Twitter, Facebook, LINE, etc.)\end{tabular} & 19.9 & 18.2 & 28.7 & 2.8 & 2.7 & 3.0 \\
  5             & \begin{tabular}{l}Call the embassy of your country\\to collect Information\end{tabular} & 25.2 & 30.3 & N/A & 2.7 & 2.7 & N/A \\
  6             & \begin{tabular}{l}Collect Information from TV and radio\end{tabular} & 24.9 & 20.9 & 44.7 & 3.0 & 2.8 & 3.4 \\
  7             & \begin{tabular}{l}Check maps and digital signage to\\collect Information\end{tabular} & 14.4 & 15.0 & 11.7 & 2.8 & 2.8 & 2.9 \\
  8             & \begin{tabular}{l}Gather Information by calling out to\\Japanese people nearby\end{tabular} & 18.7 & 19.0 & 17.3 & 2.7 & 2.8 & 2.3 \\
  9             & \begin{tabular}{l}Contact staff at tourist Information\\centers to collect Information\end{tabular} & 16.0 & 18.3 & 4.3 & 2.9 & 2.9 & 2.3 \\
 10            & \begin{tabular}{l}Contact the hotel staff to collect\\Information\end{tabular} & 15.8 & 17.0 & 10.0 & 2.8 & 2.8 & 2.8 \\
 11            & \begin{tabular}{l}Contact public transport staff to\\collect Information\end{tabular} & 13.9 & 13.7 & 14.7 & 2.5 & 2.6 & 2.4 \\
 12            & \begin{tabular}{l}Stay at your current location\end{tabular} & 16.6 & 17.5 & 12.0 & 3.4 & 3.4 & 3.5 \\
 13            & \begin{tabular}{l}Secure necessary supplies\\(food, drink, etc.)\end{tabular}  & 32.9 & 33.0 & 32.3 & 3.0 & 3.1 & 2.8 \\
 14            & \begin{tabular}{l}Move to an open space such as a\\nearby park\end{tabular} & 39.7 & 41.8 & 29.0 & 3.3 & 3.3 & 3.0 \\
 15            & \begin{tabular}{l}Move according to evacuation\\guidance\end{tabular} & 53.7 & 54.9 & 47.7 & 3.4 & 3.5 & 3.2 \\
 16            & \begin{tabular}{l}Move to the evacuation center on\\your own\end{tabular} & 26.8 & 27.1 & 25.0 & 3.1 & 3.1 & 3.0 \\
 17            & \begin{tabular}{l}Move in sync with the movements\\of people around you\end{tabular} & 28.3 & 29.2 & 24.0 & 2.8 & 2.9 & 2.6 \\
 18            & \begin{tabular}{l}Observe the surroundings because\\you don't know what to do\end{tabular} & 45.2 & 44.7 & 48.0 & 3.6 & 3.6 & 3.4 \\
\hline
  \end{tabular}
\end{table}


\begin{table}[h]
  \caption[Result of Selected score and Selected rate in Scenario 2]{Result of Selected score and Selected rate in Scenario 2(No.: number of  selection, FV: Foreign Vistors, J: Japanese)}
  \label{table18}
  \centering 
  \begin{tabular}{cl|ccc|ccc}
                &   & \multicolumn{3}{c}{Selected Rate (\%)} & \multicolumn{3}{c}{Selected Score} \\
      No.     & \multicolumn{1}{c|}{Description} & All & FV & J & All & FV & J \\
 \hline
  8             & \begin{tabular}{l}Gather Information by calling out to\\Japanese people nearby\end{tabular} & 41.1 & 40.5 & 44.0 & 2.8 & 2.8 & 2.9 \\
  9             & \begin{tabular}{l}Contact staff at tourist Information\\centers to collect Information\end{tabular} & 33.8 & 36.8 & 18.7 & 2.7 & 2.7 & 2.4 \\
 10            & \begin{tabular}{l}Contact the hotel staff to collect\\Information\end{tabular} & 33.0 & 34.3 & 26.7 & 2.9 & 2.8 & 3.0 \\
 11            & \begin{tabular}{l}Contact public transport staff to\\collect Information\end{tabular} & 29.8 & 29.3 & 32.7 & 2.7 & 2.6 & 2.8 \\
 12            & \begin{tabular}{l}Stay at your current location\end{tabular} & 23.9 & 24.7 & 20.0 & 3.0 & 3.0 & 3.0 \\
 13            & \begin{tabular}{l}Secure necessary supplies\\(food, drink, etc.)\end{tabular}  & 44.5 & 44.4 & 45.0 & 2.9 & 2.9 & 3.0 \\
 14            & \begin{tabular}{l}Move to an open space such as a\\nearby park\end{tabular} & 52.5 & 54.5 & 44.7 & 3.2 & 3.2 & 3.1 \\
 15            & \begin{tabular}{l}Move according to evacuation\\guidance\end{tabular} & 66.6 & 65.9 & 70.0 & 3.4 & 3.4 & 3.5 \\
 16            & \begin{tabular}{l}Move to the evacuation center on\\your own\end{tabular} & 39.3 & 37.9 & 46.3 & 2.9 & 2.9 & 2.7 \\
 17            & \begin{tabular}{l}Move in sync with the movements\\of people around you\end{tabular} & 46.5 & 47.3 & 42.7 & 2.9 & 3.0 & 2.8 \\
 18            & \begin{tabular}{l}Observe the surroundings because\\you don't know what to do\end{tabular} & 61.6 & 60.7 & 66.0 & 3.6 & 3.5 & 3.8 \\
\hline
  \end{tabular}
\end{table}



\begin{table}[h]
  \caption[Result of Selected score and Selected rate in Scenario 3]{Result of Selected score and Selected rate in Scenario 3(No.: number of  selection, FV: Foreign Vistors, J: Japanese)}
  \label{table19}
  \centering 
  \begin{tabular}{cl|ccc|ccc}
                &   & \multicolumn{3}{c}{Selected Rate (\%)} & \multicolumn{3}{c}{Selected Score} \\
      No.     & \multicolumn{1}{c|}{Description} & All & FV & J & All & FV & J \\
 \hline
  1             & \begin{tabular}{l}Collect Information on the official websites\\of Japanese government agencies\end{tabular} & 29.1 & 28.6 & 31.1 & 3.0 & 3.0 & 3.0 \\
  2             & \begin{tabular}{l}Collect Information with the disaster\\prevention app on your smartphone\end{tabular} & 29.4 & 27.1 & 41.0 & 3.0 & 2.9 & 3.2 \\
  3             & \begin{tabular}{l}Collect Information on news sites and\\disaster prevention portal sites\end{tabular} & 27.6 & 24.5 & 43.0 & 2.8 & 2.7 & 3.2 \\
  4             & \begin{tabular}{l}Collect Information on SNS\\(Twitter, Facebook, LINE, etc.)\end{tabular} & 24.6 & 22.7 & 34.3 & 2.8 & 2.7 & 3.0 \\
  5             & \begin{tabular}{l}Call the embassy of your country\\to collect Information\end{tabular} & 24.2 & 29.0 & N/A & 2.9 & 2.9 & N/A \\
  6             & \begin{tabular}{l}Collect Information from TV and radio\end{tabular} & 22.4 & 19.3 & 37.7 & 2.9 & 2.8 & 3.0 \\
  7             & \begin{tabular}{l}Check maps and digital signage to\\collect Information\end{tabular} & 15.7 & 16.2 & 13.3 & 2.9 & 2.9 & 2.9 \\
  8             & \begin{tabular}{l}Gather Information by calling out to\\Japanese people nearby\end{tabular} & 20.7 & 21.1 & 18.3 & 2.7 & 2.7 & 2.7 \\
  9             & \begin{tabular}{l}Contact staff at tourist Information\\centers to collect Information\end{tabular} & 16.8 & 18.8 & 7.0 & 2.7 & 2.7 & 2.7 \\
 10            & \begin{tabular}{l}Contact the hotel staff to collect\\Information\end{tabular} & 15.4 & 17.1 & 7.0 & 2.8 & 2.8 & 2.9 \\
 11            & \begin{tabular}{l}Contact public transport staff to\\collect Information\end{tabular} & 23.9 & 22.6 & 30.7 & 3.0 & 3.0 & 3.1 \\
 12            & \begin{tabular}{l}Stay at your current location\end{tabular} & 13.7 & 14.3 & 10.3 & 3.6 & 3.7 & 2.9 \\
 13            & \begin{tabular}{l}Secure necessary supplies\\(food, drink, etc.)\end{tabular}  & 27.8 & 28.3 & 25.3 & 3.0 & 3.0 & 2.9 \\
 14            & \begin{tabular}{l}Move to an open space such as a\\nearby park\end{tabular} & 34.6 & 37.6 & 19.3 & 3.2 & 3.2 & 2.9 \\
 15            & \begin{tabular}{l}Move according to evacuation\\guidance\end{tabular} & 50.7 & 50.9 & 50.0 & 3.4 & 3.4 & 3.2 \\
 16            & \begin{tabular}{l}Move to the evacuation center on\\your own\end{tabular} & 26.6 & 27.9 & 20.3 & 2.9 & 2.9 & 2.8 \\
 17            & \begin{tabular}{l}Move in sync with the movements\\of people around you\end{tabular} & 32.4 & 33.6 & 26.7 & 2.9 & 2.9 & 2.8 \\
 18            & \begin{tabular}{l}Observe the surroundings because\\you don't know what to do\end{tabular} & 41.1 & 40.8 & 42.7 & 3.7 & 3.7 & 3.5 \\
\hline
  \end{tabular}
\end{table}


\begin{table}[h]
  \caption[Result of Selected score and Selected rate in Scenario 4]{Result of Selected score and Selected rate in Scenario 4(No.: number of  selection, FV: Foreign Vistors, J: Japanese)}
  \label{table20}
  \centering 
  \begin{tabular}{cl|ccc|ccc}
                &   & \multicolumn{3}{c}{Selected Rate (\%)} & \multicolumn{3}{c}{Selected Score} \\
      No.     & \multicolumn{1}{c|}{Description} & All & FV & J & All & FV & J \\
 \hline
  8             & \begin{tabular}{l}Gather Information by calling out to\\Japanese people nearby\end{tabular} & 42.3 & 43.4 & 37.0 & 2.8 & 2.8 & 2.8 \\
  9             & \begin{tabular}{l}Contact staff at tourist Information\\centers to collect Information\end{tabular} & 29.2 & 32.1 & 14.7 & 2.7 & 2.7 & 2.7 \\
 10            & \begin{tabular}{l}Contact the hotel staff to collect\\Information\end{tabular} & 27.2 & 29.5 & 15.7 & 2.9 & 2.9 & 2.8 \\
 11            & \begin{tabular}{l}Contact public transport staff to\\collect Information\end{tabular} & 40.9 & 39.4 & 48.3 & 3.1 & 3.0 & 3.4 \\
 12            & \begin{tabular}{l}Stay at your current location\end{tabular} & 24.3 & 24.8 & 21.7 & 3.1 & 3.2 & 3.0 \\
 13            & \begin{tabular}{l}Secure necessary supplies\\(food, drink, etc.)\end{tabular}  & 39.1 & 39.2 & 38.3 & 2.8 & 2.9 & 2.8 \\
 14            & \begin{tabular}{l}Move to an open space such as a\\nearby park\end{tabular} & 49.8 & 52.2 & 37.7 & 3.0 & 3.0 & 2.6 \\
 15            & \begin{tabular}{l}Move according to evacuation\\guidance\end{tabular} & 67.7 & 66.2 & 75.0 & 3.5 & 3.4 & 3.6 \\
 16            & \begin{tabular}{l}Move to the evacuation center on\\your own\end{tabular} & 41.1 & 40.9 & 42.0 & 2.8 & 2.8 & 2.5 \\
 17            & \begin{tabular}{l}Move in sync with the movements\\of people around you\end{tabular} & 50.6 & 50.9 & 49.0 & 2.9 & 3.0 & 2.8 \\
 18            & \begin{tabular}{l}Observe the surroundings because\\you don't know what to do\end{tabular} & 59.2 & 57.5 & 67.7 & 3.5 & 3.5 & 3.6 \\
\hline
  \end{tabular}
\end{table}
%\fi
\cleardoublepage
\subsection{Sankey Diagram}

Figure~\ref{fig26} depicts the Sankey Diagram for foreign visitors in scenarios 1 to 4, whereas Figure~\ref{fig27} depicts the Sankey Diagram for Japanese visitors in scenarios 1 to 4. Figure~\ref{fig28} shows a summary of the Sankey Diagram data, with blue denoting the action with the highest value. 

%%%%%%%%%%%%%%%%%%%%%%%%%%
%\iffalse
\begin{figure*}[h]
  \begin{subfigure}{0.5\textwidth}
    \centering
    \includegraphics[width=\textwidth]{Figure/Figure26a.jpg}
    \caption{In scenario 1}
    \label{fig26a}
  \end{subfigure}
  \begin{subfigure}{0.5\textwidth}
    \centering
    \includegraphics[width=\linewidth]{Figure/Figure26b.jpg}
    \caption{In scenario 2}
    \label{fig26b}
  \end{subfigure}
  \begin{subfigure}{0.5\textwidth}
    \centering
    \includegraphics[width=\linewidth]{Figure/Figure26c.jpg}
    \caption{In scenario 3}
    \label{fig26c}
  \end{subfigure}
  \begin{subfigure}{0.5\textwidth}
    \centering
    \includegraphics[width=\linewidth]{Figure/Figure26d.jpg}
    \caption{In scenario 4}
    \label{fig26d}
  \end{subfigure}
  \caption{Sankey diagram of foreign visitors }
  \label{fig26}
\end{figure*}

\begin{figure*}[h]
  \begin{subfigure}{0.5\textwidth}
    \centering
    \includegraphics[width=\textwidth]{Figure/Figure27a.jpg}
    \caption{In scenario 1}
    \label{fig27a}
  \end{subfigure}
  \begin{subfigure}{0.5\textwidth}
    \centering
    \includegraphics[width=\linewidth]{Figure/Figure27b.jpg}
    \caption{In scenario 2}
    \label{fig27b}
  \end{subfigure}
  \begin{subfigure}{0.5\textwidth}
    \centering
    \includegraphics[width=\linewidth]{Figure/Figure27c.jpg}
    \caption{In scenario 3}
    \label{fig27c}
  \end{subfigure}
  \begin{subfigure}{0.5\textwidth}
    \centering
    \includegraphics[width=\linewidth]{Figure/Figure27d.jpg}
    \caption{In scenario 4}
    \label{fig27d}
  \end{subfigure}
  \caption{Sankey diagram of Japanese }
  \label{fig27}
\end{figure*}

\begin{figure*}[h]
  \includegraphics[width=\linewidth]{Figure/Figure28.jpg}
  \centering
  \caption{Summary of Sankey diagram data}
  \label{fig28}
\end{figure*}
%\fi

We can learn from the data that No-face-to-face information seeking is more common than face-to-face information seeking. In the top three actions, following evacuation guidance behaviors are more common than self-evacuation behaviors. 

On the other hand, when the internet and telephone are available, Japanese prefer to take No-face-to-face information-seeking behaviors, while foreign visitors prefer to take following evacuation guidance behavior first, then trend to take No-face-to-face information-seeking behaviors. 

When the internet and phone are unavailable, both Japanese and foreign visitors show similar behavior. Both Japanese and foreigners prefer to take following evacuation guidance behaviors first. However, there are some differences in the behaviors that followed. Self-evacuation behaviors are preferred by the Japanese, while foreigners prefer face-to-face information-seeking behaviors before self-evacuation behaviors.

We can more clearly see the differences between foreign visitors and Japanese when we attribute specific actions to behavior patterns. The most common actions among Japanese in scenarios 1 and 3 are all No-face-to-face information seeking, but the most common behaviors among foreign tourists are following evacuation guidance behaviors, followed by No-face-to-face information seeking. 

I also used the Sankey diagram to analyze the flow of respondents' response actions in each country. The results are shown in \crefrange{fig32}{fig36}. But from the graph we can actually hardly see the difference, basically, the action selection is similar in terms of ratio for each cis. This proves that there is little variability among foreigners and people's actions are not influenced by differences in home country nationality.

\begin{figure*}[h]
  \begin{subfigure}{0.5\textwidth}
    \centering
    \includegraphics[width=\textwidth]{Figure/figure32a.png}
    \caption{In scenario 1}
    \label{fig32a}
  \end{subfigure}
  \begin{subfigure}{0.5\textwidth}
    \centering
    \includegraphics[width=\linewidth]{Figure/figure32b.png}
    \caption{In scenario 2}
    \label{fig32b}
  \end{subfigure}
  \begin{subfigure}{0.5\textwidth}
    \centering
    \includegraphics[width=\linewidth]{Figure/figure32c.png}
    \caption{In scenario 3}
    \label{fig32c}
  \end{subfigure}
  \begin{subfigure}{0.5\textwidth}
    \centering
    \includegraphics[width=\linewidth]{Figure/figure32d.png}
    \caption{In scenario 4}
    \label{fig32d}
  \end{subfigure}
  \caption{ Sankey diagram of people from Indonesia }
  \label{fig32}
\end{figure*}

\begin{figure*}[h]
  \begin{subfigure}{0.5\textwidth}
    \centering
    \includegraphics[width=\textwidth]{Figure/figure33a.png}
    \caption{In scenario 1}
    \label{fig33a}
  \end{subfigure}
  \begin{subfigure}{0.5\textwidth}
    \centering
    \includegraphics[width=\linewidth]{Figure/figure33b.png}
    \caption{In scenario 2}
    \label{fig33b}
  \end{subfigure}
  \begin{subfigure}{0.5\textwidth}
    \centering
    \includegraphics[width=\linewidth]{Figure/figure33c.png}
    \caption{In scenario 3}
    \label{fig33c}
  \end{subfigure}
  \begin{subfigure}{0.5\textwidth}
    \centering
    \includegraphics[width=\linewidth]{Figure/figure33d.png}
    \caption{In scenario 4}
    \label{fig33d}
  \end{subfigure}
  \caption{ Sankey diagram of people from South Korea }
  \label{fig33}
\end{figure*}

\begin{figure*}[h]
  \begin{subfigure}{0.5\textwidth}
    \centering
    \includegraphics[width=\textwidth]{Figure/figure34a.png}
    \caption{In scenario 1}
    \label{fig34a}
  \end{subfigure}
  \begin{subfigure}{0.5\textwidth}
    \centering
    \includegraphics[width=\linewidth]{Figure/figure34b.png}
    \caption{In scenario 2}
    \label{fig34b}
  \end{subfigure}
  \begin{subfigure}{0.5\textwidth}
    \centering
    \includegraphics[width=\linewidth]{Figure/figure34c.png}
    \caption{In scenario 3}
    \label{fig34c}
  \end{subfigure}
  \begin{subfigure}{0.5\textwidth}
    \centering
    \includegraphics[width=\linewidth]{Figure/figure34d.png}
    \caption{In scenario 4}
    \label{fig34d}
  \end{subfigure}
  \caption{ Sankey diagram of people from Thailand}
  \label{fig34}
\end{figure*}

\begin{figure*}[h]
  \begin{subfigure}{0.5\textwidth}
    \centering
    \includegraphics[width=\textwidth]{Figure/figure35a.png}
    \caption{In scenario 1}
    \label{fig35a}
  \end{subfigure}
  \begin{subfigure}{0.5\textwidth}
    \centering
    \includegraphics[width=\linewidth]{Figure/figure35b.png}
    \caption{In scenario 2}
    \label{fig35b}
  \end{subfigure}
  \begin{subfigure}{0.5\textwidth}
    \centering
    \includegraphics[width=\linewidth]{Figure/figure35c.png}
    \caption{In scenario 3}
    \label{fig35c}
  \end{subfigure}
  \begin{subfigure}{0.5\textwidth}
    \centering
    \includegraphics[width=\linewidth]{Figure/figure35d.png}
    \caption{In scenario 4}
    \label{fig35d}
  \end{subfigure}
  \caption{ Sankey diagram of people from the UK}
  \label{fig35}
\end{figure*}


\begin{figure*}[h]
  \begin{subfigure}{0.5\textwidth}
    \centering
    \includegraphics[width=\textwidth]{Figure/figure36a.png}
    \caption{In scenario 1}
    \label{fig36a}
  \end{subfigure}
  \begin{subfigure}{0.5\textwidth}
    \centering
    \includegraphics[width=\linewidth]{Figure/figure36b.png}
    \caption{In scenario 2}
    \label{fig36b}
  \end{subfigure}
  \begin{subfigure}{0.5\textwidth}
    \centering
    \includegraphics[width=\linewidth]{Figure/figure36c.png}
    \caption{In scenario 3}
    \label{fig36c}
  \end{subfigure}
  \begin{subfigure}{0.5\textwidth}
    \centering
    \includegraphics[width=\linewidth]{Figure/figure36d.png}
    \caption{In scenario 4}
    \label{fig36d}
  \end{subfigure}
  \caption{ Sankey diagram of people from China}
  \label{fig36}
\end{figure*}



%
% Some LaTeX commands I define for my own nomenclature.
% If you have to, it's better to change nomenclature once here than in a 
% million places throughout your thesis!



%======================================================================
\chapter{Objective 3: To explore patterns of information seeking and evacuation behaviors.}
\label{c6}

\section{Methodology}

For research objective 3, this study wanted to explore respondents' information-seeking behaviors and evacuation behaviors through their selections of behaviors. In order to understand which of the behaviors are preferred and which are selected more frequently, we chose to measure both the selected rate and the selected score. The calculation of the selected rate and the selected score will both do three times, for 300 Japanese samples, 1500 foreign visitors samples, and 1800 of all respondents' samples.

\subsection{Selected Rate}
Because the selected rate wants to explore which behaviors are used more, this part does not take the order factor into account. No matter the behavior is selected in which order, it will count as 1 point. The selected rate is equal to the Sum selected point divided by the sample number, shown in Figure~\ref{fig10}.

%%%%%%%%%%%%%%%%%%%%%%%%%%%%
%\iffalse
\begin{figure*}[h]
  \includegraphics[width=\linewidth]{Figure/Figure10.png}
  \centering
  \caption{Selected rate }
  \label{fig10}
\end{figure*}
%\fi

\subsection{Selected Score}

Since the Selected score wants to explore which behaviors are used first, this part needs to take the order factor into account. The higher the preference is, the higher the score will be. So the first selected action is scored as 5, the second is scored as 4, the third is scored as 3, the fourth is scored as 2, and the last selected action is scored as 1. The Selected score is equal to the total score divided by the Sum selected point, shown in Figure~\ref{fig11}.

%%%%%%%%%%%%%%%%%%%%%%%%%%%
%\iffalse
\begin{figure*}[h]
  \includegraphics[width=\linewidth]{Figure/Figure11.png}
  \centering
  \caption{Selected score}
  \label{fig11}
\end{figure*}
%\fi

\subsection{Sankey Diagram of behavior patterns}

In the analysis of the selected score and selected rate of study objective 3, we will find that all the results will be relatively scattered. This is because, in this study, the number of available selections is relatively large, which makes the results more scattered. However, from the results, we can see that there could be similarities in the behavior patterns of people. Here the word 'pattern' means behavior patterns, not specific behavior. For example, the behavior of  'collecting information' is the same, the difference is how to collect information, from official websites, from disaster prevention software, from disaster prevention websites, from SNS, etc. So how to divide detailed behaviors into patterns? From the available selection, we can find that the behavior is mainly divided into two kinds, which are information-seeking behavior and evacuation behavior. First, regarding information-seeking behavior, we can find that there are two main patterns, one is 'No-face-to-face information seeking' and the other is 'Face-to-face information seeking'. And for the evacuation behavior, we can also find two main patterns, one is ' Self-evacuation behaviors ' and the other is ' Following evacuation guidance behavior '. Then, we divided all the selections into the patterns they belong to, which can be found in Figure~\ref{fig12}.

%%%%%%%%%%%%%%%%%%%%%%%%%%%
%\iffalse
\begin{figure*}[h]
  \includegraphics[width=\linewidth]{Figure/Figure12.png}
  \centering
  \caption{behavior patterns}
  \label{fig12}
\end{figure*}
%\fi

After attributing all the actions to the 4 patterns, we went through the flow of the respondents' actions with the help of the Sankey diagram. The Sankey diagram is a flow chart that shows the flow from each set of values to another set of values. The thickness of the lines expresses the number of values present in the group. In the Sankey diagram, the number indicates the No. of action, so 1-5 means the first action to the fifth action. Capital letters indicate behavior patterns.' No-face-to-face information seeking' is A; 'Face-to-face information seeking' is B; 'Self-evacuation behaviors' is C; 'Following evacuation guidance behavior' is D. Thus, the behavior from the first to the fifth cohort can be clearly represented in the results of the Sankey diagram. As an example, A1 indicates that the 1st response action during the disaster is behavior pattern A, which is 'No-face-to-face information seeking'.






\section{Result}


\subsection{Selected Score and Selected Rate}
The result of the Selected score and the Selected rate is shown in \crefrange{table17}{table20}. From the result, we can find that there are some differences between foreign visitors and Japanese in scenarios 1 and 3, which could show that when the internet and telephone are available, people tend to have various behaviors. By checking the Selected Rate, we can find that evacuation behaviors are more used than information-seeking behaviors. And among the evacuation behaviors, 'Moving according to evacuation guidance' could be used most. This indicates that, regardless of the order factor, 'Moving according to evacuation guidance' is the most favored option. In addition, people are more likely to heed evacuation instructions if they are in the area of such recommendations.  By checking the Selected Score, we can find that evacuation behaviors always happen before information-seeking behaviors. And among the evacuation behaviors, 'Observe the surroundings' could have happened first. This is compounded by the fact that many people's first instinct in the event of a disaster is to observe others. Not only might they receive some evacuation suggestions, but keeping the same pace as others will make people feel more at ease psychologically. 

Some lower used information-seeking behaviors during a disaster are 'Gather Information by calling out to Japanese people nearby', 'Contact staff at tourist Information centers to collect Information', 'Contact public transport staff to collect Information'. As a result, in the case of Internet\&Phone available, people do not choose to acquire information through methods that require verbal conversation, and instead prefer to obtain it on their own. On the other hand, because people can only obtain information through the verbal conversation when the Internet and phone are unavailable, their method of obtaining information depends on the scenario they are in. When people are in a tourist area, they usually ask Japanese people around them for information, and not many people choose the other three options of contacting staff at different spots. However, when people are moving by transportation, people still ask Japanese people around them for information, while contacting staff from public transportation is also a popular option. The lowest used evacuation behavior during a disaster is 'Stay at your current location. This is understandable after all, few people will just stay put and do nothing in the face of a disaster. It is important to note here that this result does not mean that everyone will necessarily do something to leave the place where it happened, but that people's priority evacuation behavior is less likely to be to stay where they are. People will, in most situations, choose to remain where they are after gathering information and following evacuation instructions. This section does not cover such circumstances.

%%%%%%%%%%%%%%%%%%%%%%%%%%
%\iffalse
\begin{table}[h]
  \caption[Result of Selected score and Selected rate in Scenario 1]{Result of Selected score and Selected rate in Scenario 1(No.: number of  selection, FV: Foreign Vistors, J: Japanese)}
  \label{table17}
  \centering 
  \begin{tabular}{cl|ccc|ccc}
                &   & \multicolumn{3}{c}{Selected Rate (\%)} & \multicolumn{3}{c}{Selected Score} \\
      No.     & \multicolumn{1}{c|}{Description} & All & FV & J & All & FV & J \\
 \hline
  1             & \begin{tabular}{l}Collect Information on the official websites\\of Japanese government agencies\end{tabular} & 31.6 &30.2 & 38.7 & 2.9 & 2.9 & 3.0 \\
  2             & \begin{tabular}{l}Collect Information with the disaster\\prevention app on your smartphone\end{tabular} & 27.9 & 25.4 & 40.7 & 2.9 & 2.8 & 3.2 \\
  3             & \begin{tabular}{l}Collect Information on news sites and\\disaster prevention portal sites\end{tabular} & 26.8 & 23.4 & 43.7 & 2.8 & 2.7 & 3.2 \\
  4             & \begin{tabular}{l}Collect Information on SNS\\(Twitter, Facebook, LINE, etc.)\end{tabular} & 19.9 & 18.2 & 28.7 & 2.8 & 2.7 & 3.0 \\
  5             & \begin{tabular}{l}Call the embassy of your country\\to collect Information\end{tabular} & 25.2 & 30.3 & N/A & 2.7 & 2.7 & N/A \\
  6             & \begin{tabular}{l}Collect Information from TV and radio\end{tabular} & 24.9 & 20.9 & 44.7 & 3.0 & 2.8 & 3.4 \\
  7             & \begin{tabular}{l}Check maps and digital signage to\\collect Information\end{tabular} & 14.4 & 15.0 & 11.7 & 2.8 & 2.8 & 2.9 \\
  8             & \begin{tabular}{l}Gather Information by calling out to\\Japanese people nearby\end{tabular} & 18.7 & 19.0 & 17.3 & 2.7 & 2.8 & 2.3 \\
  9             & \begin{tabular}{l}Contact staff at tourist Information\\centers to collect Information\end{tabular} & 16.0 & 18.3 & 4.3 & 2.9 & 2.9 & 2.3 \\
 10            & \begin{tabular}{l}Contact the hotel staff to collect\\Information\end{tabular} & 15.8 & 17.0 & 10.0 & 2.8 & 2.8 & 2.8 \\
 11            & \begin{tabular}{l}Contact public transport staff to\\collect Information\end{tabular} & 13.9 & 13.7 & 14.7 & 2.5 & 2.6 & 2.4 \\
 12            & \begin{tabular}{l}Stay at your current location\end{tabular} & 16.6 & 17.5 & 12.0 & 3.4 & 3.4 & 3.5 \\
 13            & \begin{tabular}{l}Secure necessary supplies\\(food, drink, etc.)\end{tabular}  & 32.9 & 33.0 & 32.3 & 3.0 & 3.1 & 2.8 \\
 14            & \begin{tabular}{l}Move to an open space such as a\\nearby park\end{tabular} & 39.7 & 41.8 & 29.0 & 3.3 & 3.3 & 3.0 \\
 15            & \begin{tabular}{l}Move according to evacuation\\guidance\end{tabular} & 53.7 & 54.9 & 47.7 & 3.4 & 3.5 & 3.2 \\
 16            & \begin{tabular}{l}Move to the evacuation center on\\your own\end{tabular} & 26.8 & 27.1 & 25.0 & 3.1 & 3.1 & 3.0 \\
 17            & \begin{tabular}{l}Move in sync with the movements\\of people around you\end{tabular} & 28.3 & 29.2 & 24.0 & 2.8 & 2.9 & 2.6 \\
 18            & \begin{tabular}{l}Observe the surroundings because\\you don't know what to do\end{tabular} & 45.2 & 44.7 & 48.0 & 3.6 & 3.6 & 3.4 \\
\hline
  \end{tabular}
\end{table}


\begin{table}[h]
  \caption[Result of Selected score and Selected rate in Scenario 2]{Result of Selected score and Selected rate in Scenario 2(No.: number of  selection, FV: Foreign Vistors, J: Japanese)}
  \label{table18}
  \centering 
  \begin{tabular}{cl|ccc|ccc}
                &   & \multicolumn{3}{c}{Selected Rate (\%)} & \multicolumn{3}{c}{Selected Score} \\
      No.     & \multicolumn{1}{c|}{Description} & All & FV & J & All & FV & J \\
 \hline
  8             & \begin{tabular}{l}Gather Information by calling out to\\Japanese people nearby\end{tabular} & 41.1 & 40.5 & 44.0 & 2.8 & 2.8 & 2.9 \\
  9             & \begin{tabular}{l}Contact staff at tourist Information\\centers to collect Information\end{tabular} & 33.8 & 36.8 & 18.7 & 2.7 & 2.7 & 2.4 \\
 10            & \begin{tabular}{l}Contact the hotel staff to collect\\Information\end{tabular} & 33.0 & 34.3 & 26.7 & 2.9 & 2.8 & 3.0 \\
 11            & \begin{tabular}{l}Contact public transport staff to\\collect Information\end{tabular} & 29.8 & 29.3 & 32.7 & 2.7 & 2.6 & 2.8 \\
 12            & \begin{tabular}{l}Stay at your current location\end{tabular} & 23.9 & 24.7 & 20.0 & 3.0 & 3.0 & 3.0 \\
 13            & \begin{tabular}{l}Secure necessary supplies\\(food, drink, etc.)\end{tabular}  & 44.5 & 44.4 & 45.0 & 2.9 & 2.9 & 3.0 \\
 14            & \begin{tabular}{l}Move to an open space such as a\\nearby park\end{tabular} & 52.5 & 54.5 & 44.7 & 3.2 & 3.2 & 3.1 \\
 15            & \begin{tabular}{l}Move according to evacuation\\guidance\end{tabular} & 66.6 & 65.9 & 70.0 & 3.4 & 3.4 & 3.5 \\
 16            & \begin{tabular}{l}Move to the evacuation center on\\your own\end{tabular} & 39.3 & 37.9 & 46.3 & 2.9 & 2.9 & 2.7 \\
 17            & \begin{tabular}{l}Move in sync with the movements\\of people around you\end{tabular} & 46.5 & 47.3 & 42.7 & 2.9 & 3.0 & 2.8 \\
 18            & \begin{tabular}{l}Observe the surroundings because\\you don't know what to do\end{tabular} & 61.6 & 60.7 & 66.0 & 3.6 & 3.5 & 3.8 \\
\hline
  \end{tabular}
\end{table}



\begin{table}[h]
  \caption[Result of Selected score and Selected rate in Scenario 3]{Result of Selected score and Selected rate in Scenario 3(No.: number of  selection, FV: Foreign Vistors, J: Japanese)}
  \label{table19}
  \centering 
  \begin{tabular}{cl|ccc|ccc}
                &   & \multicolumn{3}{c}{Selected Rate (\%)} & \multicolumn{3}{c}{Selected Score} \\
      No.     & \multicolumn{1}{c|}{Description} & All & FV & J & All & FV & J \\
 \hline
  1             & \begin{tabular}{l}Collect Information on the official websites\\of Japanese government agencies\end{tabular} & 29.1 & 28.6 & 31.1 & 3.0 & 3.0 & 3.0 \\
  2             & \begin{tabular}{l}Collect Information with the disaster\\prevention app on your smartphone\end{tabular} & 29.4 & 27.1 & 41.0 & 3.0 & 2.9 & 3.2 \\
  3             & \begin{tabular}{l}Collect Information on news sites and\\disaster prevention portal sites\end{tabular} & 27.6 & 24.5 & 43.0 & 2.8 & 2.7 & 3.2 \\
  4             & \begin{tabular}{l}Collect Information on SNS\\(Twitter, Facebook, LINE, etc.)\end{tabular} & 24.6 & 22.7 & 34.3 & 2.8 & 2.7 & 3.0 \\
  5             & \begin{tabular}{l}Call the embassy of your country\\to collect Information\end{tabular} & 24.2 & 29.0 & N/A & 2.9 & 2.9 & N/A \\
  6             & \begin{tabular}{l}Collect Information from TV and radio\end{tabular} & 22.4 & 19.3 & 37.7 & 2.9 & 2.8 & 3.0 \\
  7             & \begin{tabular}{l}Check maps and digital signage to\\collect Information\end{tabular} & 15.7 & 16.2 & 13.3 & 2.9 & 2.9 & 2.9 \\
  8             & \begin{tabular}{l}Gather Information by calling out to\\Japanese people nearby\end{tabular} & 20.7 & 21.1 & 18.3 & 2.7 & 2.7 & 2.7 \\
  9             & \begin{tabular}{l}Contact staff at tourist Information\\centers to collect Information\end{tabular} & 16.8 & 18.8 & 7.0 & 2.7 & 2.7 & 2.7 \\
 10            & \begin{tabular}{l}Contact the hotel staff to collect\\Information\end{tabular} & 15.4 & 17.1 & 7.0 & 2.8 & 2.8 & 2.9 \\
 11            & \begin{tabular}{l}Contact public transport staff to\\collect Information\end{tabular} & 23.9 & 22.6 & 30.7 & 3.0 & 3.0 & 3.1 \\
 12            & \begin{tabular}{l}Stay at your current location\end{tabular} & 13.7 & 14.3 & 10.3 & 3.6 & 3.7 & 2.9 \\
 13            & \begin{tabular}{l}Secure necessary supplies\\(food, drink, etc.)\end{tabular}  & 27.8 & 28.3 & 25.3 & 3.0 & 3.0 & 2.9 \\
 14            & \begin{tabular}{l}Move to an open space such as a\\nearby park\end{tabular} & 34.6 & 37.6 & 19.3 & 3.2 & 3.2 & 2.9 \\
 15            & \begin{tabular}{l}Move according to evacuation\\guidance\end{tabular} & 50.7 & 50.9 & 50.0 & 3.4 & 3.4 & 3.2 \\
 16            & \begin{tabular}{l}Move to the evacuation center on\\your own\end{tabular} & 26.6 & 27.9 & 20.3 & 2.9 & 2.9 & 2.8 \\
 17            & \begin{tabular}{l}Move in sync with the movements\\of people around you\end{tabular} & 32.4 & 33.6 & 26.7 & 2.9 & 2.9 & 2.8 \\
 18            & \begin{tabular}{l}Observe the surroundings because\\you don't know what to do\end{tabular} & 41.1 & 40.8 & 42.7 & 3.7 & 3.7 & 3.5 \\
\hline
  \end{tabular}
\end{table}


\begin{table}[h]
  \caption[Result of Selected score and Selected rate in Scenario 4]{Result of Selected score and Selected rate in Scenario 4(No.: number of  selection, FV: Foreign Vistors, J: Japanese)}
  \label{table20}
  \centering 
  \begin{tabular}{cl|ccc|ccc}
                &   & \multicolumn{3}{c}{Selected Rate (\%)} & \multicolumn{3}{c}{Selected Score} \\
      No.     & \multicolumn{1}{c|}{Description} & All & FV & J & All & FV & J \\
 \hline
  8             & \begin{tabular}{l}Gather Information by calling out to\\Japanese people nearby\end{tabular} & 42.3 & 43.4 & 37.0 & 2.8 & 2.8 & 2.8 \\
  9             & \begin{tabular}{l}Contact staff at tourist Information\\centers to collect Information\end{tabular} & 29.2 & 32.1 & 14.7 & 2.7 & 2.7 & 2.7 \\
 10            & \begin{tabular}{l}Contact the hotel staff to collect\\Information\end{tabular} & 27.2 & 29.5 & 15.7 & 2.9 & 2.9 & 2.8 \\
 11            & \begin{tabular}{l}Contact public transport staff to\\collect Information\end{tabular} & 40.9 & 39.4 & 48.3 & 3.1 & 3.0 & 3.4 \\
 12            & \begin{tabular}{l}Stay at your current location\end{tabular} & 24.3 & 24.8 & 21.7 & 3.1 & 3.2 & 3.0 \\
 13            & \begin{tabular}{l}Secure necessary supplies\\(food, drink, etc.)\end{tabular}  & 39.1 & 39.2 & 38.3 & 2.8 & 2.9 & 2.8 \\
 14            & \begin{tabular}{l}Move to an open space such as a\\nearby park\end{tabular} & 49.8 & 52.2 & 37.7 & 3.0 & 3.0 & 2.6 \\
 15            & \begin{tabular}{l}Move according to evacuation\\guidance\end{tabular} & 67.7 & 66.2 & 75.0 & 3.5 & 3.4 & 3.6 \\
 16            & \begin{tabular}{l}Move to the evacuation center on\\your own\end{tabular} & 41.1 & 40.9 & 42.0 & 2.8 & 2.8 & 2.5 \\
 17            & \begin{tabular}{l}Move in sync with the movements\\of people around you\end{tabular} & 50.6 & 50.9 & 49.0 & 2.9 & 3.0 & 2.8 \\
 18            & \begin{tabular}{l}Observe the surroundings because\\you don't know what to do\end{tabular} & 59.2 & 57.5 & 67.7 & 3.5 & 3.5 & 3.6 \\
\hline
  \end{tabular}
\end{table}
%\fi
\cleardoublepage
\subsection{Sankey Diagram of behavior patterns}

Figure~\ref{fig26} depicts the Sankey Diagram for foreign visitors in scenarios 1 to 4, whereas Figure~\ref{fig27} depicts the Sankey Diagram for Japanese visitors in scenarios 1 to 4. Figure~\ref{fig28} shows a summary of the Sankey Diagram data, with blue denoting the action with the highest value. 

\begin{figure*}[h]
  \includegraphics[width=\linewidth]{Figure/Figure28.jpg}
  \centering
  \caption{Summary of Sankey diagram data}
  \label{fig28}
\end{figure*}

%%%%%%%%%%%%%%%%%%%%%%%%%%
%\iffalse
\begin{figure*}[h]
  \begin{subfigure}{0.5\textwidth}
    \centering
    \includegraphics[width=\textwidth]{Figure/Figure26a.jpg}
    \caption{In scenario 1}
    \label{fig26a}
  \end{subfigure}
  \begin{subfigure}{0.5\textwidth}
    \centering
    \includegraphics[width=\linewidth]{Figure/Figure26b.jpg}
    \caption{In scenario 2}
    \label{fig26b}
  \end{subfigure}
  \begin{subfigure}{0.5\textwidth}
    \centering
    \includegraphics[width=\linewidth]{Figure/Figure26c.jpg}
    \caption{In scenario 3}
    \label{fig26c}
  \end{subfigure}
  \begin{subfigure}{0.5\textwidth}
    \centering
    \includegraphics[width=\linewidth]{Figure/Figure26d.jpg}
    \caption{In scenario 4}
    \label{fig26d}
  \end{subfigure}
  \caption{Sankey diagram of foreign visitors }
  \label{fig26}
\end{figure*}

\begin{figure*}[h]
  \begin{subfigure}{0.5\textwidth}
    \centering
    \includegraphics[width=\textwidth]{Figure/Figure27a.jpg}
    \caption{In scenario 1}
    \label{fig27a}
  \end{subfigure}
  \begin{subfigure}{0.5\textwidth}
    \centering
    \includegraphics[width=\linewidth]{Figure/Figure27b.jpg}
    \caption{In scenario 2}
    \label{fig27b}
  \end{subfigure}
  \begin{subfigure}{0.5\textwidth}
    \centering
    \includegraphics[width=\linewidth]{Figure/Figure27c.jpg}
    \caption{In scenario 3}
    \label{fig27c}
  \end{subfigure}
  \begin{subfigure}{0.5\textwidth}
    \centering
    \includegraphics[width=\linewidth]{Figure/Figure27d.jpg}
    \caption{In scenario 4}
    \label{fig27d}
  \end{subfigure}
  \caption{Sankey diagram of Japanese }
  \label{fig27}
\end{figure*}


%\fi

We can learn from the data that No-face-to-face information seeking is more common than face-to-face information seeking. In the top three actions, following evacuation guidance behaviors are more common than self-evacuation behaviors. 

On the other hand, when the internet and telephone are available, Japanese prefer to take No-face-to-face information-seeking behaviors, while foreign visitors prefer to take following evacuation guidance behavior first, then trend to take No-face-to-face information-seeking behaviors. 

When the internet and phone are unavailable, both Japanese and foreign visitors show similar behavior. Both Japanese and foreigners prefer to take following evacuation guidance behaviors first. However, there are some differences in the behaviors that followed. Self-evacuation behaviors are preferred by the Japanese, while foreigners prefer face-to-face information-seeking behaviors before self-evacuation behaviors.

We can more clearly see the differences between foreign visitors and Japanese when we attribute specific actions to behavior patterns. The most common actions among Japanese in scenarios 1 and 3 are all No-face-to-face information seeking, but the most common behaviors among foreign tourists are following evacuation guidance behaviors, followed by No-face-to-face information seeking. 

I also used the Sankey diagram to analyze the flow of respondents' response actions in each country. The results are shown in \crefrange{fig32}{fig35}. But from the graph we can actually hardly see the difference, basically, the action selection is similar in terms of ratio for each cis. This proves that there is little variability among foreigners and people's actions are not influenced by differences in home country nationality.



\begin{figure*}[h]
  \begin{subfigure}{0.5\textwidth}
    \centering
    \includegraphics[width=\textwidth]{Figure/figure32a.png}
    \caption{Indonesia}
  \end{subfigure}
  \begin{subfigure}{0.5\textwidth}
    \centering
    \includegraphics[width=\linewidth]{Figure/figure33a.png}
    \caption{South Korea}
  \end{subfigure}
  \begin{subfigure}{0.5\textwidth}
    \centering
    \includegraphics[width=\linewidth]{Figure/figure34a.png}
    \caption{Thailand}
  \end{subfigure}
  \begin{subfigure}{0.5\textwidth}
    \centering
    \includegraphics[width=\linewidth]{Figure/figure35a.png}
    \caption{the UK}
  \end{subfigure}
  \begin{subfigure}{0.5\textwidth}
    \centering
    \includegraphics[width=\linewidth]{Figure/figure36a.png}
    \caption{China}
  \end{subfigure}
  \caption{ Sankey diagram of foreign vistors in senario 1 }
  \label{fig32}
\end{figure*}

\begin{figure*}[h]
  \begin{subfigure}{0.5\textwidth}
    \centering
    \includegraphics[width=\textwidth]{Figure/figure32b.png}
    \caption{Indonesia}
  \end{subfigure}
  \begin{subfigure}{0.5\textwidth}
    \centering
    \includegraphics[width=\linewidth]{Figure/figure33b.png}
    \caption{South Korea}
  \end{subfigure}
  \begin{subfigure}{0.5\textwidth}
    \centering
    \includegraphics[width=\linewidth]{Figure/figure34b.png}
    \caption{Thailand}
  \end{subfigure}
  \begin{subfigure}{0.5\textwidth}
    \centering
    \includegraphics[width=\linewidth]{Figure/figure35b.png}
    \caption{the UK}
  \end{subfigure}
  \begin{subfigure}{0.5\textwidth}
    \centering
    \includegraphics[width=\linewidth]{Figure/figure36b.png}
    \caption{China}
  \end{subfigure}
  \caption{ Sankey diagram of foreign vistors in senario 2 }
  \label{fig33}
\end{figure*}

\begin{figure*}[h]
  \begin{subfigure}{0.5\textwidth}
    \centering
    \includegraphics[width=\textwidth]{Figure/figure32c.png}
    \caption{Indonesia}
  \end{subfigure}
  \begin{subfigure}{0.5\textwidth}
    \centering
    \includegraphics[width=\linewidth]{Figure/figure33c.png}
    \caption{South Korea}
  \end{subfigure}
  \begin{subfigure}{0.5\textwidth}
    \centering
    \includegraphics[width=\linewidth]{Figure/figure34c.png}
    \caption{Thailand}
  \end{subfigure}
  \begin{subfigure}{0.5\textwidth}
    \centering
    \includegraphics[width=\linewidth]{Figure/figure35c.png}
    \caption{the UK}
  \end{subfigure}
  \begin{subfigure}{0.5\textwidth}
    \centering
    \includegraphics[width=\linewidth]{Figure/figure36c.png}
    \caption{China}
  \end{subfigure}
  \caption{ Sankey diagram of foreign vistors in senario 3 }
  \label{fig34}
\end{figure*}

\begin{figure*}[h]
  \begin{subfigure}{0.5\textwidth}
    \centering
    \includegraphics[width=\textwidth]{Figure/figure32d.png}
    \caption{Indonesia}
  \end{subfigure}
  \begin{subfigure}{0.5\textwidth}
    \centering
    \includegraphics[width=\linewidth]{Figure/figure33d.png}
    \caption{South Korea}
  \end{subfigure}
  \begin{subfigure}{0.5\textwidth}
    \centering
    \includegraphics[width=\linewidth]{Figure/figure34d.png}
    \caption{Thailand}
  \end{subfigure}
  \begin{subfigure}{0.5\textwidth}
    \centering
    \includegraphics[width=\linewidth]{Figure/figure35d.png}
    \caption{the UK}
  \end{subfigure}
  \begin{subfigure}{0.5\textwidth}
    \centering
    \includegraphics[width=\linewidth]{Figure/figure36d.png}
    \caption{China}
  \end{subfigure}
  \caption{ Sankey diagram of foreign vistors in senario 4 }
  \label{fig35}
\end{figure*}




















%% Some LaTeX commands I define for my own nomenclature.
% If you have to, it's better to change nomenclature once here than in a 
% million places throughout your thesis!



%======================================================================
\chapter{Conclusion}
\label{c7}
%======================================================================
\section{Conclusion}
This research focused on foreign visitors' information-seeking and evacuation behaviors, as well as their behavior patterns in the event of the 'Tokyo Metropolitan Earthquake'. The research also investigates how foreign visitors to Japan perceive the Safety Tips app and how their backgrounds influence their attitudes on it. An Internet-based survey of foreign visitors and Japanese who had visited Tokyo conducted by the Economic and Social Research Institute is used for this study. 

For objective 1, we find that Safety Tips could be more popular and well-known in Indonesia, China, and Thailand than in the UK and Korea. Also, among those respondents that know Safety Tips or heard them before, their usage rate is lower than 70\%. Also, over 77\% represent a positive attitude toward Safety Tips among 4 attitude-related questions in the survey. Finally, we can conclude that respondents who know exactly and used Safety tips before show more positive attitudes towards Safety Tips than other groupings of people. 

For objective 2, we concluded that Disaster Prevention Consciousness has a negative impact on respondents' attitudes toward Safety Tips, while Knowledge and Perception on earthquakes has a positive impact. Then, Training Experience does not have a significant impact on respondents' attitudes toward Safety Tips.

For objective 3, we concluded that No-face-to-face information seeking is more common than face-to-face way, and following evacuation guidance behaviors are more common than self-evacuation behaviors.  And when the internet and telephone are available, Japanese more rely on No-face-to-face information-seeking behaviors, while foreign visitors will evacuate following guidance first, then rely on No-face-to-face information-seeking behaviors. On the contrary, if the internet and phone are unavailable, both Japanese and foreign visitors rely on evacuation following guidance. We also find that the response action during evacuation of Japanese and foreign visitors could be different, but there are no differences between foreign visitors based on nationality. 

\section{Suggestion for Safety Tips}
\begin{itemize}
\item People who have a more detailed awareness of Safety Tips are more likely to use this application, so if we want to increase the usage of Safety Tips, it would be helpful to increase foreign visitors' awareness of this application.
\item If Safety Tips can provide evacuation instructions, it will attract more people. Furthermore, if it can synchronize the user's location information, more people will use Safety Tips.
\item Foreign visitors are prone to seek non-face-to-face information, and Safety Tips is a platform for providing No-face-to-face information-seeking services. Therefore, when a disaster occurs, Safety Tips need to ensure that sufficient disaster information is available and comprehensive to all users.
\item Since foreign visitors will first follow the evacuation guidance, it is possible to cooperate with hotels, information centers, and other organizations that are capable of providing evacuation guidance. If they can remind foreign visitors that the Safety Tips application is a platform that can offer them the information they require during the evacuation guidance, the use of Safety Tips will rise, and it will be able to better assist foreign visitors.
\end{itemize}

\section{Limitations}
\begin{itemize}
\item The training experience manifest variables used in the training experience is not a scale question. If we can use scale questions to represent Training Experience, the SEM model will fit better.
\item Knowledge and Perception on earthquakes have less than three manifest variables, which makes Knowledge and Perception on earthquakes not well explained by manifest variables. In general, each latent variable should have three or more manifest variables.
\item Because the survey used in this study was not designed for this study, the SEM model could not be perfectly constructed. if a questionnaire could be designed based on the underlying assumptions, the fit would be improved.
\end{itemize}



















%Add Chapters as much as you want!
%\include{private/list-of-symbols}

\renewcommand*{\bibname}{References}

% Add the References to the Table of Contents
\addcontentsline{toc}{chapter}{\textbf{References}}
%----------------------------------------------------------------------
% APPENDICES
%---------------------------------------------------------------------- 

%\addcontentsline{toc}{chapter}{APPENDICES} 
%\appendix
% Designate with \appendix declaration which just changes numbering style 
% from here on
% Add a title page before the appendices and a line in the Table of Contents

%

%----------------------------------------------------------------------
% END MATERIAL
%----------------------------------------------------------------------

% B I B L I O G R A P H Y
% -----------------------
%
% The following statement selects the style to use for references.  It controls the sort order of the entries in the bibliography and also the formatting for the in-text labels.

% This specifies the location of the file containing the bibliographic information.  
% It assumes you're using BibTeX (if not, why not?).
%\ifthenelse{\boolean{PrintVersion}}{
%\cleardoublepage % This is needed if the book class is used, to place the anchor in the correct page,
                 % because the bibliography will start on its own page.
%}{
%\clearpage       % Use \clearpage instead if the document class uses the "oneside" argument
%}
%\phantomsection  % With hyperref package, enables hyperlinking from the table of contents to bibliography             
% The following statement causes the title "References" to be used for the bibliography section:

\begin{thebibliography}{50}
\bibitem{Cahyanto2016StatedPO} Ignatius Cahyanto, Lori Pennington-Gray, Sivaramakrishnan Srinivasan, Corene J.
Matyas, and Jorge Villegas. Stated preferences of tourists for evacuating in the event
of a hurricane. 2016.

\bibitem{SEMres} 崇宏 星野, 謙介 岡田, and 忠彦 前田. 構造方程式モデリングにおける適合度指標とモデル改善について : 展望とシミュレーション研究による新たな知見. 行動計量学,32(2):209–235, 2005.

\bibitem{Streiner2006BuildingAB} David L. Streiner. Building a better model: An introduction to structural equation modelling. The Canadian Journal of Psychiatry, 51:317 – 324, 2006.


\bibitem{ref1} P. George, D.and Mallery. PSS for Windows step by step: A simple guide and reference 11.0 update (4th ed.). 2003.

\bibitem{ref2} R. P. McDonald and M.-H. R. Ho. Principles and practice in reporting structural equation analyses. Psychological Methods, 2002.

\bibitem{ref3}  R. Browne, M.W.and Cudeck. Alternative ways of assessing model fit. Newbury Park, 1993.

\bibitem{ref4}  G. Browne, M.W.and Mels. RAMONA user's guide. Unpublished report, Columbus, Ohio State University., 1990.

\bibitem{ref5}  J.H. Steiger. EzPATH: A supplementary module for SYSTAT and SYGRAPH. Evan ston, IL: SYSTAT., 1989.

\bibitem{ref6} PM. Bentler. Comparative fit indexes in structural models. Psychological bulletin, 1990.

\bibitem{ref7}  Veng Kheang Phun, Pharinet Pheng, Reiko Masui, Hironori Kato, and Tetsuo Yai. Impact of ride-hailing apps on traditional lamat services in asian developing cities: The phnom penh case. Asian Transport Studies, 6:100006, 2020.


\bibitem{ref8} 佐藤久美, 南宮智娜, 岡本耕平, et al. 災害時における訪日外国人観光客への情報提供に関する考察. 金城学院大学論集. 社会科学編, 16(2):112–122, 2020.


\bibitem{ref9}  Yi Wang, Miltos Kyriakidis, and Vinh N. Dang. Incorporating human factors in emergency evacuation – an overview of behavioral factors and models. International Journal of Disaster Risk Reduction, 60:102254, 2021.

\bibitem{ref10} Gesine Hofinger, Robert Zinke, and Laura K¨unzer. Human factors in evacuation simulation, planning, and guidance. Transportation Research Procedia, 2:603–611, 2014.


\bibitem{ref11}  Benjamin Cornwell. Bonded fatalities: Relational and ecological dimensions of a fire evacuation. The Sociological Quarterly, 44(4):617–638, 2003.

\bibitem{ref12} Ben Sheppard, G James Rubin, Jamie K Wardman, and Simon Wessely. Terrorism and dispelling the myth of a panic prone public. Journal of public health policy, 27(3):219–245, 2006.

\bibitem{ref13}  Veng Kheang Phun, Hironori Kato, and Tetsuo Yai. Traffic risk perception and behavioral intentions of paratransit users in phnom penh. Transportation research part F: traffic psychology and behaviour, 55:175–187, 2018.


\bibitem{ref14}  James B Schreiber, Amaury Nora, Frances K Stage, Elizabeth A Barlow, and Jamie King. Reporting structural equation modeling and confirmatory factor analysis results: A review. The Journal of educational research, 99(6):323–338, 2006.

\bibitem{ref15} Joseph F Hair, William C Black, Barry J Babin, and Rolph E Anderson. Multivariate Data Analysis (7th Edition). Englewood Cliffs, N J Prentice Hall, 2009.


\bibitem{ref16} Yi Fan, Jiquan Chen, Gabriela Shirkey, Ranjeet John, Susie R Wu, Hogeun Park, and Changliang Shao. Applications of structural equation modeling (sem) in ecological studies: an updated review. Ecological Processes, 5(1):1–12, 2016.

\bibitem{ref17} Li-tze Hu and Peter M Bentler. Cutoff criteria for fit indexes in covariance structure analysis: Conventional criteria versus new alternatives. Structural equation modeling: a multidisciplinary journal, 6(1):1–55, 1999.

\bibitem{ref18} Xitao Fan, Bruce Thompson, and Lin Wang. Effects of sample size, estimation methods, and model specification on structural equation modeling fit indexes. Structural equation modeling: a multidisciplinary journal, 6(1):56–83, 1999.

\bibitem{ref19} Ignatius Cahyanto, Lori Pennington-Gray, Brijesh Thapa, Sivaramakrishnan Srini vasan, Jorge Villegas, Corene J. Matyas, and Spiro K. Kiousis. An empirical evaluation of the determinants of tourist's hurricane evacuation decision making. Journal of Destination Marketing and Management, 2:253–265, 2014.

\bibitem{ref20} Rex B Kline. Principles and practice of structural equation modeling. Guilford publications, 2015.

\bibitem{ref21} Rick H Hoyle. Structural equation modeling for social and personality psychology. SAGE Publications Ltd, 2011.

\bibitem{ref22}  Peter M Bentler and Douglas G Bonett. Significance tests and goodness of fit in the analysis of covariance structures. Psychological bulletin, 88(3):588, 1980.

\bibitem{ref23} K Joreskog and Dag Sˆorbom. Lisrel 8: Structural equation modeling with the simplis language. Chicago: Scientific Software International, 1993.

\bibitem{ref24} Patrick J Curran, Kenneth A Bollen, Pamela Paxton, James Kirby, and Feinian Chen. The noncentral chi-square distribution in misspecified structural equation models: Finite sample results from a monte carlo simulation. Multivariate Behavioral Research, 37(1):1–36, 2002.

\bibitem{ref25}  Michael W Browne. Alternative ways of assessing model fit. Testing structural equation models, 1993.

\bibitem{ref26} Barbara G Tabachnick, Linda S Fidell, and Jodie B Ullman. Using multivariate statistics, volume 5. Pearson Boston, MA, 2007.

\bibitem{ref27}  Richard P Bagozzi and Youjae Yi. On the evaluation of structural equation models. Journal of the academy of marketing science, 16(1):74–94, 1988.



\bibitem{ref29} Peter M Bentler. Alpha, dimension-free, and model-based internal consistency reliability. Psychometrika, 74(1):137–143, 2009.

\bibitem{ref30} Tenko Raykov. Behavioral scale reliability and measurement invariance evaluation using latent variable modeling. Behavior Therapy, 35(2):299–331, 2004.

\bibitem{ref31}  Claes Fornell and David F Larcker. Structural equation models with unobservable variables and measurement error: Algebra and statistics, 1981.

\bibitem{ref32}  Hair J, Anderson R, Tatham R, and Black W. Multivariate data analysis. 5th Edition, Prentice Hall, New Jersey, 1998.

\bibitem{ref33} Scott R Colwell. The composite reliability calculator user's guide. Technical report, Technical Report, https://doi. org/10.13140/RG. 2.1. 4298.088, 2016.


\bibitem{ref34} James C Anderson and David W Gerbing. Structural equation modeling in practice: A review and recommended two-step approach. Psychological bulletin, 103(3):411, 1988.


\bibitem{ref35}  Richard P Bagozzi and Lynn W Phillips. Representing and testing organizational theories: A holistic construal. Administrative science quarterly, pages 459–489, 1982.


\bibitem{ref36} Carmines, E. G., \& McIver, J. P. (1981). Analyzing models with unobserved variables: Analysis of covariance structure. In G. W. Bohrnstedt and E. F. Borgatta (eds.), Social measurement: Current issues.

\bibitem{ref37} Ullman, J. B. (2001). Structural equation modeling. In B. G. Tabachnick and L. S. Fidell (2001), Using Multivariate Statistics (4th ed.): 653-771. Needham Heights, MA: Allyn and Bacon.

\bibitem{ref38} Kline, R. B. (2005). Principles and practice of structural equation modeling (2nd ed.). New York: Guilford.

\bibitem{ref39} Schumacker, R. E., \& Lomax, R. G. (2004). A beginner's guide to structural equation modeling (2nd ed.). Mahwah, NJ: Lawrence Erlbaum Associates.

\bibitem{ref40} Hair, J. F., Tatham, R. L., Anderson, R. E., \& Black, W. C. (1998). Multivariate Data Analysis (5th ed.). Upper Saddle River, NJ: Pearson Prentice Hall.

\bibitem{ref41} Jöreskog, K. G. (1970). A general method for analysis of covariance structures. Biometrika, 57, 239-251.

\bibitem{ref42} Wheaton,B., Muthén,B.,Alwin,D.F.\& Summers,G.F.(1977). Assessing reliability and stability in panel models. In Heise, D.R. [Ed.] Sociological methodology 1977. San Francisco: Jossey-Bass,84-136.

\bibitem{ref43} 李茂能(2006)。結構方程模式軟體Amos之簡介及其在測驗編製上之應用。台北:心理。

\bibitem{ref44} 陳正昌、程炳林、陳新豐、劉子鍵(2003)。多變量分析方法:統計軟體應用(三版)。台北:五南。

\bibitem{ref45} Ozeki, Miki, Kan Shimazaki, and Taiyoung Yi. "Exploring elements of disaster prevention consciousness: Based on interviews with anti-disaster professionals." Journal of Disaster Research 12.3 (2017): 631-638.
\end{thebibliography}

% Tip 5: You can create multiple .bib files to organize your references. 
% Just list them all in the \bibliogaphy command, separated by commas (no spaces).
%----------------------------------------------------------------------
\end{document}
%======================================================================



%%% Local Variables: 
%%% mode: latex
%%% TeX-master: t
%%% End: 

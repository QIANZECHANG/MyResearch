% Some LaTeX commands I define for my own nomenclature.
% If you have to, it's better to change nomenclature once here than in a 
% million places throughout your thesis!



%======================================================================
\chapter{Disaster response behaviors}
\label{c6}

\section{Methodology}

For research objective 3, this study wanted to explore respondents' information-seeking behaviors and evacuation behaviors through their selections of behaviors. In order to understand which of the behaviors are preferred and which are selected more frequently, we chose to measure both the selected rate and the selected score. The calculation of the selected rate and the selected score will both do three times, for 300 Japanese samples, 1500 foreign visitors samples, and 1800 of all respondents' samples.

\subsection{Selected Rate}
Because the selected rate wants to explore which behaviors are used more, this part does not take the order factor into account. No matter the behavior is selected in which order, it will count as 1 point. The selected rate is equal to the Sum selected point divided by the sample number, shown in Figure~\ref{fig10}.

%%%%%%%%%%%%%%%%%%%%%%%%%%%%
%\iffalse
\begin{figure*}[h]
  \includegraphics[width=\linewidth]{Figure/Figure10.png}
  \centering
  \caption{Selected rate }
  \label{fig10}
\end{figure*}
%\fi

\subsection{Selected Score}

Since the Selected score wants to explore which behaviors are used first, this part needs to take the order factor into account. The higher the preference is, the higher the score will be. So the first selected behavior is scored as 5, the second is scored as 4, the third is scored as 3, the fourth is scored as 2, and the last selected behavior is scored as 1. The Selected score is equal to the total score divided by the Sum selected point, shown in Figure~\ref{fig11}.

%%%%%%%%%%%%%%%%%%%%%%%%%%%
%\iffalse
\begin{figure*}[h]
  \includegraphics[width=\linewidth]{Figure/Figure11.png}
  \centering
  \caption{Selected score}
  \label{fig11}
\end{figure*}
%\fi

\subsection{Sankey diagram of behavior categorization and behavior patterns}

In the analysis of the selected score and selected rate of study objective 3, we will find that all the results will be relatively scattered. This is because, in this study, the number of available selections is relatively large, which makes the results more scattered. However, from the results, we can see that there could be similarities in the behavior categories of people. Here the word 'pattern' means behavior categories, not specific behavior. For example, the behavior of  'collecting information' is the same, the difference is how to collect information, from official websites, from disaster prevention software, from disaster prevention websites, from SNS, etc. So how to divide detailed behaviors into categories? From the available selection, we can find that the behavior is mainly divided into two kinds, which are information-seeking behavior and evacuation behavior. First, regarding information-seeking behavior, we can find that there are two main categories, one is 'No-face-to-face information seeking' and the other is 'Face-to-face information seeking'. And for the evacuation behavior, we can also find two main categories, one is ' Self-evacuation behaviors ' and the other is ' Following evacuation guidance behavior '. Then, each response behavior selected by respondents is coded as the above 4 categories shown in Figure~\ref{fig12}.

%%%%%%%%%%%%%%%%%%%%%%%%%%%
%\iffalse
\begin{figure*}[h]
  \includegraphics[width=\linewidth]{Figure/Figure12.png}
  \centering
  \caption{behavior categories}
  \label{fig12}
\end{figure*}
%\fi

After that, we use the Sankey diagram to help us clearly show the flow of the respondents' response behaviors, since the Sankey diagram is often used to visualize data flow or transfer. The thickness of the lines expresses the number of values present in the group. In the Sankey diagram, the number indicates the No. of behavior, so 1-5 means the first behavior to the fifth behavior. Capital letters indicate behavior categories.' No-face-to-face information seeking' is A; 'Face-to-face information seeking' is B; 'Self-evacuation behaviors' is C; 'Following evacuation guidance behavior' is D. Thus, the behavior from the first to the fifth cohort can be clearly represented in the results of the Sankey diagram. As an example, A1 indicates that the 1st response behavior during the disaster is behavior pattern A, which is 'No-face-to-face information seeking'.

Finally, the pattern is made with this categorization, one pattern means the sequential category of response behaviors that respondents answered in the survey. For example, DCDBB is one pattern, and this means the behavior flow shows as following evacuation guidance behavior $\rightarrow$ Self-evacuation behaviors $\rightarrow$ following evacuation guidance behavior $\rightarrow$ self-evacuation behaviors $\rightarrow$ self-evacuation behaviors. 







\section{Result}


\subsection{Selected Score and Selected Rate}
The result of the Selected score and the Selected rate is shown in \crefrange{table17}{table20}. From the result, we can find that there are some differences between foreign visitors and Japanese in scenarios 1 and 3, which could show that when the internet and telephone are available, people tend to have various behaviors. By checking the Selected Rate, we can find that evacuation behaviors are more used than information-seeking behaviors. And among the evacuation behaviors, 'Moving according to evacuation guidance' could be used most. This indicates that, regardless of the order factor, 'Moving according to evacuation guidance' is the most favored option. In addition, people are more likely to heed evacuation instructions if they are in the area of such recommendations.  By checking the Selected Score, we can find that evacuation behaviors always happen before information-seeking behaviors. And among the evacuation behaviors, 'Observe the surroundings' could have happened first. This is compounded by the fact that many people's first instinct in the event of a disaster is to observe others. Not only might they receive some evacuation suggestions, but keeping the same pace as others will make people feel more at ease psychologically. 

Some lower used information-seeking behaviors during a disaster are 'Gather Information by calling out to Japanese people nearby', 'Contact staff at tourist Information centers to collect Information', 'Contact public transport staff to collect Information'. As a result, in the case of Internet\&Phone available, people do not choose to acquire information through methods that require verbal conversation, and instead prefer to obtain it on their own. On the other hand, because people can only obtain information through the verbal conversation when the Internet and phone are unavailable, their method of obtaining information depends on the scenario they are in. When people are in a tourist area, they usually ask Japanese people around them for information, and not many people choose the other three options of contacting staff at different spots. However, when people are moving by transportation, people still ask Japanese people around them for information, while contacting staff from public transportation is also a popular option. The lowest used evacuation behavior during a disaster is 'Stay at your current location. This is understandable after all, few people will just stay put and do nothing in the face of a disaster. It is important to note here that this result does not mean that everyone will necessarily do something to leave the place where it happened, but that people's priority evacuation behavior is less likely to be to stay where they are. People will, in most situations, choose to remain where they are after gathering information and following evacuation instructions. This section does not cover such circumstances.

%%%%%%%%%%%%%%%%%%%%%%%%%%
%\iffalse
\begin{table}[h]
  \caption[Result of Selected score and Selected rate in Scenario 1]{Result of Selected score and Selected rate in Scenario 1(No.: number of  selection, FV: Foreign Vistors, J: Japanese)}
  \label{table17}
  \centering 
  \begin{tabular}{cl|ccc|ccc}
                &   & \multicolumn{3}{c}{Selected Rate (\%)} & \multicolumn{3}{c}{Selected Score} \\
      No.     & \multicolumn{1}{c|}{Description} & All & FV & J & All & FV & J \\
 \hline
  1             & \begin{tabular}{l}Collect Information on the official websites\\of Japanese government agencies\end{tabular} & 31.6 &30.2 & 38.7 & 2.9 & 2.9 & 3.0 \\
  2             & \begin{tabular}{l}Collect Information with the disaster\\prevention app on your smartphone\end{tabular} & 27.9 & 25.4 & 40.7 & 2.9 & 2.8 & 3.2 \\
  3             & \begin{tabular}{l}Collect Information on news sites and\\disaster prevention portal sites\end{tabular} & 26.8 & 23.4 & 43.7 & 2.8 & 2.7 & 3.2 \\
  4             & \begin{tabular}{l}Collect Information on SNS\\(Twitter, Facebook, LINE, etc.)\end{tabular} & 19.9 & 18.2 & 28.7 & 2.8 & 2.7 & 3.0 \\
  5             & \begin{tabular}{l}Call the embassy of your country\\to collect Information\end{tabular} & 25.2 & 30.3 & N/A & 2.7 & 2.7 & N/A \\
  6             & \begin{tabular}{l}Collect Information from TV and radio\end{tabular} & 24.9 & 20.9 & 44.7 & 3.0 & 2.8 & 3.4 \\
  7             & \begin{tabular}{l}Check maps and digital signage to\\collect Information\end{tabular} & 14.4 & 15.0 & 11.7 & 2.8 & 2.8 & 2.9 \\
  8             & \begin{tabular}{l}Gather Information by calling out to\\Japanese people nearby\end{tabular} & 18.7 & 19.0 & 17.3 & 2.7 & 2.8 & 2.3 \\
  9             & \begin{tabular}{l}Contact staff at tourist Information\\centers to collect Information\end{tabular} & 16.0 & 18.3 & 4.3 & 2.9 & 2.9 & 2.3 \\
 10            & \begin{tabular}{l}Contact the hotel staff to collect\\Information\end{tabular} & 15.8 & 17.0 & 10.0 & 2.8 & 2.8 & 2.8 \\
 11            & \begin{tabular}{l}Contact public transport staff to\\collect Information\end{tabular} & 13.9 & 13.7 & 14.7 & 2.5 & 2.6 & 2.4 \\
 12            & \begin{tabular}{l}Stay at your current location\end{tabular} & 16.6 & 17.5 & 12.0 & 3.4 & 3.4 & 3.5 \\
 13            & \begin{tabular}{l}Secure necessary supplies\\(food, drink, etc.)\end{tabular}  & 32.9 & 33.0 & 32.3 & 3.0 & 3.1 & 2.8 \\
 14            & \begin{tabular}{l}Move to an open space such as a\\nearby park\end{tabular} & 39.7 & 41.8 & 29.0 & 3.3 & 3.3 & 3.0 \\
 15            & \begin{tabular}{l}Move according to evacuation\\guidance\end{tabular} & 53.7 & 54.9 & 47.7 & 3.4 & 3.5 & 3.2 \\
 16            & \begin{tabular}{l}Move to the evacuation center on\\your own\end{tabular} & 26.8 & 27.1 & 25.0 & 3.1 & 3.1 & 3.0 \\
 17            & \begin{tabular}{l}Move in sync with the movements\\of people around you\end{tabular} & 28.3 & 29.2 & 24.0 & 2.8 & 2.9 & 2.6 \\
 18            & \begin{tabular}{l}Observe the surroundings because\\you don't know what to do\end{tabular} & 45.2 & 44.7 & 48.0 & 3.6 & 3.6 & 3.4 \\
\hline
  \end{tabular}
\end{table}


\begin{table}[h]
  \caption[Result of Selected score and Selected rate in Scenario 2]{Result of Selected score and Selected rate in Scenario 2(No.: number of  selection, FV: Foreign Vistors, J: Japanese)}
  \label{table18}
  \centering 
  \begin{tabular}{cl|ccc|ccc}
                &   & \multicolumn{3}{c}{Selected Rate (\%)} & \multicolumn{3}{c}{Selected Score} \\
      No.     & \multicolumn{1}{c|}{Description} & All & FV & J & All & FV & J \\
 \hline
  8             & \begin{tabular}{l}Gather Information by calling out to\\Japanese people nearby\end{tabular} & 41.1 & 40.5 & 44.0 & 2.8 & 2.8 & 2.9 \\
  9             & \begin{tabular}{l}Contact staff at tourist Information\\centers to collect Information\end{tabular} & 33.8 & 36.8 & 18.7 & 2.7 & 2.7 & 2.4 \\
 10            & \begin{tabular}{l}Contact the hotel staff to collect\\Information\end{tabular} & 33.0 & 34.3 & 26.7 & 2.9 & 2.8 & 3.0 \\
 11            & \begin{tabular}{l}Contact public transport staff to\\collect Information\end{tabular} & 29.8 & 29.3 & 32.7 & 2.7 & 2.6 & 2.8 \\
 12            & \begin{tabular}{l}Stay at your current location\end{tabular} & 23.9 & 24.7 & 20.0 & 3.0 & 3.0 & 3.0 \\
 13            & \begin{tabular}{l}Secure necessary supplies\\(food, drink, etc.)\end{tabular}  & 44.5 & 44.4 & 45.0 & 2.9 & 2.9 & 3.0 \\
 14            & \begin{tabular}{l}Move to an open space such as a\\nearby park\end{tabular} & 52.5 & 54.5 & 44.7 & 3.2 & 3.2 & 3.1 \\
 15            & \begin{tabular}{l}Move according to evacuation\\guidance\end{tabular} & 66.6 & 65.9 & 70.0 & 3.4 & 3.4 & 3.5 \\
 16            & \begin{tabular}{l}Move to the evacuation center on\\your own\end{tabular} & 39.3 & 37.9 & 46.3 & 2.9 & 2.9 & 2.7 \\
 17            & \begin{tabular}{l}Move in sync with the movements\\of people around you\end{tabular} & 46.5 & 47.3 & 42.7 & 2.9 & 3.0 & 2.8 \\
 18            & \begin{tabular}{l}Observe the surroundings because\\you don't know what to do\end{tabular} & 61.6 & 60.7 & 66.0 & 3.6 & 3.5 & 3.8 \\
\hline
  \end{tabular}
\end{table}



\begin{table}[h]
  \caption[Result of Selected score and Selected rate in Scenario 3]{Result of Selected score and Selected rate in Scenario 3(No.: number of  selection, FV: Foreign Vistors, J: Japanese)}
  \label{table19}
  \centering 
  \begin{tabular}{cl|ccc|ccc}
                &   & \multicolumn{3}{c}{Selected Rate (\%)} & \multicolumn{3}{c}{Selected Score} \\
      No.     & \multicolumn{1}{c|}{Description} & All & FV & J & All & FV & J \\
 \hline
  1             & \begin{tabular}{l}Collect Information on the official websites\\of Japanese government agencies\end{tabular} & 29.1 & 28.6 & 31.1 & 3.0 & 3.0 & 3.0 \\
  2             & \begin{tabular}{l}Collect Information with the disaster\\prevention app on your smartphone\end{tabular} & 29.4 & 27.1 & 41.0 & 3.0 & 2.9 & 3.2 \\
  3             & \begin{tabular}{l}Collect Information on news sites and\\disaster prevention portal sites\end{tabular} & 27.6 & 24.5 & 43.0 & 2.8 & 2.7 & 3.2 \\
  4             & \begin{tabular}{l}Collect Information on SNS\\(Twitter, Facebook, LINE, etc.)\end{tabular} & 24.6 & 22.7 & 34.3 & 2.8 & 2.7 & 3.0 \\
  5             & \begin{tabular}{l}Call the embassy of your country\\to collect Information\end{tabular} & 24.2 & 29.0 & N/A & 2.9 & 2.9 & N/A \\
  6             & \begin{tabular}{l}Collect Information from TV and radio\end{tabular} & 22.4 & 19.3 & 37.7 & 2.9 & 2.8 & 3.0 \\
  7             & \begin{tabular}{l}Check maps and digital signage to\\collect Information\end{tabular} & 15.7 & 16.2 & 13.3 & 2.9 & 2.9 & 2.9 \\
  8             & \begin{tabular}{l}Gather Information by calling out to\\Japanese people nearby\end{tabular} & 20.7 & 21.1 & 18.3 & 2.7 & 2.7 & 2.7 \\
  9             & \begin{tabular}{l}Contact staff at tourist Information\\centers to collect Information\end{tabular} & 16.8 & 18.8 & 7.0 & 2.7 & 2.7 & 2.7 \\
 10            & \begin{tabular}{l}Contact the hotel staff to collect\\Information\end{tabular} & 15.4 & 17.1 & 7.0 & 2.8 & 2.8 & 2.9 \\
 11            & \begin{tabular}{l}Contact public transport staff to\\collect Information\end{tabular} & 23.9 & 22.6 & 30.7 & 3.0 & 3.0 & 3.1 \\
 12            & \begin{tabular}{l}Stay at your current location\end{tabular} & 13.7 & 14.3 & 10.3 & 3.6 & 3.7 & 2.9 \\
 13            & \begin{tabular}{l}Secure necessary supplies\\(food, drink, etc.)\end{tabular}  & 27.8 & 28.3 & 25.3 & 3.0 & 3.0 & 2.9 \\
 14            & \begin{tabular}{l}Move to an open space such as a\\nearby park\end{tabular} & 34.6 & 37.6 & 19.3 & 3.2 & 3.2 & 2.9 \\
 15            & \begin{tabular}{l}Move according to evacuation\\guidance\end{tabular} & 50.7 & 50.9 & 50.0 & 3.4 & 3.4 & 3.2 \\
 16            & \begin{tabular}{l}Move to the evacuation center on\\your own\end{tabular} & 26.6 & 27.9 & 20.3 & 2.9 & 2.9 & 2.8 \\
 17            & \begin{tabular}{l}Move in sync with the movements\\of people around you\end{tabular} & 32.4 & 33.6 & 26.7 & 2.9 & 2.9 & 2.8 \\
 18            & \begin{tabular}{l}Observe the surroundings because\\you don't know what to do\end{tabular} & 41.1 & 40.8 & 42.7 & 3.7 & 3.7 & 3.5 \\
\hline
  \end{tabular}
\end{table}


\begin{table}[h]
  \caption[Result of Selected score and Selected rate in Scenario 4]{Result of Selected score and Selected rate in Scenario 4(No.: number of  selection, FV: Foreign Vistors, J: Japanese)}
  \label{table20}
  \centering 
  \begin{tabular}{cl|ccc|ccc}
                &   & \multicolumn{3}{c}{Selected Rate (\%)} & \multicolumn{3}{c}{Selected Score} \\
      No.     & \multicolumn{1}{c|}{Description} & All & FV & J & All & FV & J \\
 \hline
  8             & \begin{tabular}{l}Gather Information by calling out to\\Japanese people nearby\end{tabular} & 42.3 & 43.4 & 37.0 & 2.8 & 2.8 & 2.8 \\
  9             & \begin{tabular}{l}Contact staff at tourist Information\\centers to collect Information\end{tabular} & 29.2 & 32.1 & 14.7 & 2.7 & 2.7 & 2.7 \\
 10            & \begin{tabular}{l}Contact the hotel staff to collect\\Information\end{tabular} & 27.2 & 29.5 & 15.7 & 2.9 & 2.9 & 2.8 \\
 11            & \begin{tabular}{l}Contact public transport staff to\\collect Information\end{tabular} & 40.9 & 39.4 & 48.3 & 3.1 & 3.0 & 3.4 \\
 12            & \begin{tabular}{l}Stay at your current location\end{tabular} & 24.3 & 24.8 & 21.7 & 3.1 & 3.2 & 3.0 \\
 13            & \begin{tabular}{l}Secure necessary supplies\\(food, drink, etc.)\end{tabular}  & 39.1 & 39.2 & 38.3 & 2.8 & 2.9 & 2.8 \\
 14            & \begin{tabular}{l}Move to an open space such as a\\nearby park\end{tabular} & 49.8 & 52.2 & 37.7 & 3.0 & 3.0 & 2.6 \\
 15            & \begin{tabular}{l}Move according to evacuation\\guidance\end{tabular} & 67.7 & 66.2 & 75.0 & 3.5 & 3.4 & 3.6 \\
 16            & \begin{tabular}{l}Move to the evacuation center on\\your own\end{tabular} & 41.1 & 40.9 & 42.0 & 2.8 & 2.8 & 2.5 \\
 17            & \begin{tabular}{l}Move in sync with the movements\\of people around you\end{tabular} & 50.6 & 50.9 & 49.0 & 2.9 & 3.0 & 2.8 \\
 18            & \begin{tabular}{l}Observe the surroundings because\\you don't know what to do\end{tabular} & 59.2 & 57.5 & 67.7 & 3.5 & 3.5 & 3.6 \\
\hline
  \end{tabular}
\end{table}
%\fi
\cleardoublepage
\subsection{Sankey diagram of behavior categorization and behavior patterns}

Summary of the Sankey Diagram data is shown in Figure~\ref{fig28}, blue denoting the behavior with the highest rate. 

\begin{figure*}[h]
  \includegraphics[width=\linewidth]{Figure/Figure28.jpg}
  \centering
  \caption{Summary of Sankey diagram data}
  \label{fig28}
\end{figure*}

We can learn from the data that No-face-to-face information seeking is more common than face-to-face information seeking. In the top three behaviors, following evacuation guidance behaviors is more common than self-evacuation behaviors. 

The Sankey Diagram for foreign visitors and Japanese in scenarios 1 to 4 is depicted in \crefrange{fig26a}{fig26d}. By the Sankey diagram, we can clearly find that there are many differences between Japanese and foreign visitors. In scenario 1(Figure~\ref{fig26a}),  the internet and telephone are available, and respondents are staying in a tourist attrbehavior. The result shows Japanese always prefer to seek information, the rate of selecting  No-face-to-face information-seeking behaviors from the first order to the fifth order is always higher than other behavior categories. In Particular, the top three behaviors are almost 50\%.  However, for foreign visitors, the first and the second behavior show differences from the Japanese. Foreign visitors prefer to start by following evacuation guidance behaviors, and then start to seek information.  In scenario 2(Figure~\ref{fig26c}), respondents are still staying in a tourist attrbehavior, but the internet and telephone are unavailable this time. The result shows that the behaviors of Japanese and foreign visitors show some similarities. They both start from following evacuation guidance behaviors, and then face-to-face information seeking, finally self-evacuation behaviors.  In scenario 2(Figure~\ref{fig26b}) and scenario 4(Figure~\ref{fig26d}) respondents are moving by public transportation, the results show similar with scenario 1 and scenario 3. This can indicate that respondents' behaviors are not much affected by their situation, but affected by whether the internet and telephone are available or not. 

%%%%%%%%%%%%%%%%%%%%%%%%%%
%\iffalse
\begin{figure*}[h]
  \begin{subfigure}{0.5\textwidth}
    \centering
    \includegraphics[width=\textwidth]{Figure/Figure26a.jpg}
    \caption{Japanese}
  \end{subfigure}
  \begin{subfigure}{0.5\textwidth}
    \centering
    \includegraphics[width=\linewidth]{Figure/Figure27a.jpg}
    \caption{Foreign visitors}
  \end{subfigure}
  \caption{Sankey diagram of in senario 1 }
  \label{fig26a}
\end{figure*}

\begin{figure*}[h]
  \begin{subfigure}{0.5\textwidth}
    \centering
    \includegraphics[width=\textwidth]{Figure/Figure26b.jpg}
    \caption{Japanese}
  \end{subfigure}
  \begin{subfigure}{0.5\textwidth}
    \centering
    \includegraphics[width=\linewidth]{Figure/Figure27b.jpg}
    \caption{Foreign visitors}
  \end{subfigure}
  \caption{Sankey diagram of in senario 2 }
  \label{fig26b}
\end{figure*}

\begin{figure*}[h]
  \begin{subfigure}{0.5\textwidth}
    \centering
    \includegraphics[width=\textwidth]{Figure/Figure26c.jpg}
    \caption{Japanese}
  \end{subfigure}
  \begin{subfigure}{0.5\textwidth}
    \centering
    \includegraphics[width=\linewidth]{Figure/Figure27c.jpg}
    \caption{Foreign visitors}
  \end{subfigure}
  \caption{Sankey diagram of in senario 3 }
  \label{fig26c}
\end{figure*}

\begin{figure*}[h]
  \begin{subfigure}{0.5\textwidth}
    \centering
    \includegraphics[width=\textwidth]{Figure/Figure26d.jpg}
    \caption{Japanese}
  \end{subfigure}
  \begin{subfigure}{0.5\textwidth}
    \centering
    \includegraphics[width=\linewidth]{Figure/Figure27d.jpg}
    \caption{Foreign visitors}
  \end{subfigure}
  \caption{Sankey diagram of in senario 4 }
  \label{fig26d}
\end{figure*}


We can more clearly see the differences between foreign visitors and Japanese when we attribute specific behaviors to behavior patterns. The most common behaviors among Japanese in scenarios 1 and 3 are all No-face-to-face information seeking, but the most common behaviors among foreign tourists are following evacuation guidance behaviors, followed by No-face-to-face information seeking. In summary, when the internet and telephone are available, Japanese prefer to take No-face-to-face information-seeking behaviors, while foreign visitors prefer to follow evacuation guidance behavior first, then trend to take No-face-to-face information-seeking behaviors. When the internet and phone are unavailable, both Japanese and foreign visitors show similar behavior. Both Japanese and foreigners prefer to follow evacuation guidance behaviors first. However, there are some differences in the behaviors that followed. Self-evacuation behaviors are preferred by the Japanese, while foreigners prefer face-to-face information-seeking behaviors before self-evacuation behaviors.


We also used the Sankey diagram to analyze the flow of respondents' response behaviors in each country. The results are shown in Table~\ref{fig32} to Table~\ref{fig35}. But from the graph we can actually hardly see the difference, basically, the behavior selection is similar in terms of ratio for each cis. This proves that there is little variability among foreigners and people's behaviors are not influenced by differences in home country nationality.



\begin{figure*}[h]
  \begin{subfigure}{0.5\textwidth}
    \centering
    \includegraphics[width=\textwidth]{Figure/figure32a.png}
    \caption{Indonesia}
  \end{subfigure}
  \begin{subfigure}{0.5\textwidth}
    \centering
    \includegraphics[width=\linewidth]{Figure/figure33a.png}
    \caption{South Korea}
  \end{subfigure}
  \begin{subfigure}{0.5\textwidth}
    \centering
    \includegraphics[width=\linewidth]{Figure/figure34a.png}
    \caption{Thailand}
  \end{subfigure}
  \begin{subfigure}{0.5\textwidth}
    \centering
    \includegraphics[width=\linewidth]{Figure/figure35a.png}
    \caption{the UK}
  \end{subfigure}
  \begin{subfigure}{0.5\textwidth}
    \centering
    \includegraphics[width=\linewidth]{Figure/figure36a.png}
    \caption{China}
  \end{subfigure}
  \caption{ Sankey diagram of foreign vistors in senario 1 }
  \label{fig32}
\end{figure*}

\begin{figure*}[h]
  \begin{subfigure}{0.5\textwidth}
    \centering
    \includegraphics[width=\textwidth]{Figure/figure32b.png}
    \caption{Indonesia}
  \end{subfigure}
  \begin{subfigure}{0.5\textwidth}
    \centering
    \includegraphics[width=\linewidth]{Figure/figure33b.png}
    \caption{South Korea}
  \end{subfigure}
  \begin{subfigure}{0.5\textwidth}
    \centering
    \includegraphics[width=\linewidth]{Figure/figure34b.png}
    \caption{Thailand}
  \end{subfigure}
  \begin{subfigure}{0.5\textwidth}
    \centering
    \includegraphics[width=\linewidth]{Figure/figure35b.png}
    \caption{the UK}
  \end{subfigure}
  \begin{subfigure}{0.5\textwidth}
    \centering
    \includegraphics[width=\linewidth]{Figure/figure36b.png}
    \caption{China}
  \end{subfigure}
  \caption{ Sankey diagram of foreign vistors in senario 2 }
  \label{fig33}
\end{figure*}

\begin{figure*}[h]
  \begin{subfigure}{0.5\textwidth}
    \centering
    \includegraphics[width=\textwidth]{Figure/figure32c.png}
    \caption{Indonesia}
  \end{subfigure}
  \begin{subfigure}{0.5\textwidth}
    \centering
    \includegraphics[width=\linewidth]{Figure/figure33c.png}
    \caption{South Korea}
  \end{subfigure}
  \begin{subfigure}{0.5\textwidth}
    \centering
    \includegraphics[width=\linewidth]{Figure/figure34c.png}
    \caption{Thailand}
  \end{subfigure}
  \begin{subfigure}{0.5\textwidth}
    \centering
    \includegraphics[width=\linewidth]{Figure/figure35c.png}
    \caption{the UK}
  \end{subfigure}
  \begin{subfigure}{0.5\textwidth}
    \centering
    \includegraphics[width=\linewidth]{Figure/figure36c.png}
    \caption{China}
  \end{subfigure}
  \caption{ Sankey diagram of foreign vistors in senario 3 }
  \label{fig34}
\end{figure*}

\begin{figure*}[h]
  \begin{subfigure}{0.5\textwidth}
    \centering
    \includegraphics[width=\textwidth]{Figure/figure32d.png}
    \caption{Indonesia}
  \end{subfigure}
  \begin{subfigure}{0.5\textwidth}
    \centering
    \includegraphics[width=\linewidth]{Figure/figure33d.png}
    \caption{South Korea}
  \end{subfigure}
  \begin{subfigure}{0.5\textwidth}
    \centering
    \includegraphics[width=\linewidth]{Figure/figure34d.png}
    \caption{Thailand}
  \end{subfigure}
  \begin{subfigure}{0.5\textwidth}
    \centering
    \includegraphics[width=\linewidth]{Figure/figure35d.png}
    \caption{the UK}
  \end{subfigure}
  \begin{subfigure}{0.5\textwidth}
    \centering
    \includegraphics[width=\linewidth]{Figure/figure36d.png}
    \caption{China}
  \end{subfigure}
  \caption{ Sankey diagram of foreign vistors in senario 4 }
  \label{fig35}
\end{figure*}

The results of the behavior patterns are presented in Table~\ref{table40}, which describes the top ten patterns in each scenario. Considering that some respondents gave response behaviors without until 5th, and some respondents' choices of response behaviors are among choice 19 to choice 21 (available choices shown in Table~\ref{table4}) which can not be categorized. So, these samples were removed. Therefore, the total sample size for Japan is 206, and the total sample size for foreign visitors is 1230. The numbers in the buckle after Japanese and foreign visitors in the table indicate the number of patterns. From the results, we can find that the top ten patterns among Japanese account for about 20\% of the total patterns, but the overall percentage of foreign visitors is lower than that of Japanese. Of course, this has something to do with the fact that the sample size of foreign visitors is larger than that of Japanese. Among the foreign visitors, the top ten patterns in scenario 1 and scenario 3 accounted for less than those in scenario 2 and scenario 4. The top ten patterns in scenario 1 and scenario 3 accounted for about 10\%, and the top ten patterns in scenario 2 and scenario 4 accounted for about 17\%. In scenario 1, disaster response behavior of the Japanese mostly will firstly start from No-face-to-face information-seeking behaviors. The remaining half of the Japanese will choose self-evacuation behaviors or following evacuation guidance behavior. However, foreign visitors mostly will firstly start ‘Following evacuation guidance behavior, the remaining half will choose self-evacuation behaviors. In scenario 2, the disaster response behavior of both Japanese and foreign visitors mostly will firstly start from evacuation behavior. But Japanese prefer self-evacuation behaviors, while foreign visitors prefer following evacuation guidance behavior. In scenario 3, the disaster response behavior of the Japanese shows a similar result with scenario 1. But, foreign visitors show big differences compared with scenario 1. The results show that when foreign visitors are moving by public transportation, a group of foreign visitors prefer to seek information rather than evacuation, since the top three patterns started from information-seeking behaviors, and a total of top three patterns counted 38\% of top ten patterns. While evacuation still counted over half of it. In scenario 4, the disaster response behavior of both Japanese and foreign visitors shows a similar result with scenario 2, and they both prefer following evacuation guidance behavior, considering respondents are moving by public transportation.

\begin{table}[h]
  \caption{Result of behavior patterns}
  \label{table40}
\centering
\begin{tabular}{cc|cc|cc|cc}
\hline
\multicolumn{4}{c|}{S1}                                                     & \multicolumn{4}{c}{S2}                                                     \\
\multicolumn{2}{c}{Japanese (142)}  & \multicolumn{2}{c|}{Foreign visitors (480)} & \multicolumn{2}{c}{Japanese (111)}  & \multicolumn{2}{c}{Foreign visitors (209)} \\
\hline
AAAAB & 2.91\%                      & DCDCA  & 1.06\%                      & CDDCC & 2.43\%                      & DCDBB  & 2.36\%                      \\
CAAAC & 2.43\%                      & AAAAB  & 0.98\%                      & DCCDB & 2.43\%                      & BBBBC  & 1.95\%                      \\
AAAAD & 2.43\%                      & DCAAA  & 0.98\%                      & DDCCC & 1.94\%                      & DCCDC  & 1.71\%                      \\
AAADC & 1.94\%                      & DDCAA  & 0.98\%                      & DDBBB & 1.94\%                      & DCCBB  & 1.71\%                      \\
AADCC & 1.94\%                      & CDDCD  & 0.89\%                      & CCDCD & 1.94\%                      & DCDDC  & 1.71\%                      \\
AACDC & 1.94\%                      & CDCAA  & 0.89\%                      & CDCCD & 1.46\%                      & DDCBB  & 1.63\%                      \\
CDAAD & 1.46\%                      & CDAAA  & 0.89\%                      & BDDCC & 1.46\%                      & DDCCB  & 1.63\%                      \\
AAAAA & 1.46\%                      & DCDAC  & 0.81\%                      & DCCDC & 1.46\%                      & CDDCC  & 1.63\%                      \\
DCDAA & 1.46\%                      & DCBAA  & 0.81\%                      & CDDDB & 1.46\%                      & CCDDC  & 1.54\%                      \\
DCAAA & 1.46\%                      & DCAAD  & 0.81\%                      & CCDDB & \multicolumn{1}{c}{2.70\%}  & CDDCB  & 1.46\%                      \\
\hline
Total & \multicolumn{1}{c}{19.43\%} &        & \multicolumn{1}{c|}{9.10\%}  &       & \multicolumn{1}{c}{19.22\%} &        & \multicolumn{1}{c}{17.33\%} \\
\hline
\\
\hline
\multicolumn{4}{c|}{S3}                                                     & \multicolumn{4}{c}{S4}                                                     \\
\multicolumn{2}{c}{Japanese (147)}  & \multicolumn{2}{c|}{Foreign visitors (470)} & \multicolumn{2}{c}{Japanese (112)}  & \multicolumn{2}{c}{Foreign visitors (211)} \\
\hline
AAAAC & 2.91\%                      & AAAAA  & 2.36\%                      & DDBCC & 3.88\%                      & DDCCB  & 1.95\%                      \\
AAAAA & 2.43\%                      & BAAAA  & 1.14\%                      & BDDCC & 3.40\%                      & DCDBB  & 1.87\%                      \\
DAAAA & 2.43\%                      & AABAA  & 1.06\%                      & DDDBB & 2.43\%                      & DCDCC  & 1.87\%                      \\
DCAAA & 2.43\%                      & DCAAA  & 1.06\%                      & DDDCC & 2.43\%                      & BBBBC  & 1.79\%                      \\
ABAAD & 1.94\%                      & DCCDD  & 1.06\%                      & DBDCC & 2.43\%                      & DDDCC  & 1.71\%                      \\
AAADC & 1.94\%                      & DCDCC  & 0.98\%                      & CDDBC & 1.94\%                      & DDCBB  & 1.63\%                      \\
AAADD & 1.94\%                      & DCAAB  & 0.81\%                      & CDDDB & 1.94\%                      & DDCDC  & 1.46\%                      \\
DBAAA & 1.94\%                      & CCDDC  & 0.81\%                      & DDCDC & 1.94\%                      & DCCDD  & 1.46\%                      \\
CCDDC & 1.46\%                      & DDCAA  & 0.81\%                      & CDCDD & 1.94\%                      & DDCCD  & 1.38\%                      \\
CDAAA & 1.46\%                      & CDAAA  & 0.81\%                      & DBCCD & 1.94\%                      & DCCBB  & 1.38\%                      \\
\hline
Total & \multicolumn{1}{c}{20.88\%} &        & \multicolumn{1}{c|}{10.90\%} &       & \multicolumn{1}{c}{24.27\%} &        & \multicolumn{1}{c}{16.50\%} \\
\hline
\end{tabular}
\end{table}
















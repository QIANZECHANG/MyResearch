% Some LaTeX commands I define for my own nomenclature.
% If you have to, it's better to change nomenclature once here than in a 
% million places throughout your thesis!



%======================================================================
\chapter{Conclusion}
%======================================================================
\section{Conclusion}
This research focused on foreign visitors' information-seeking and evacuation behaviors, as well as their behavior patterns in the event of the 'Tokyo Metropolitan Earthquake'. The research also investigates how foreign visitors to Japan perceive the Safety Tips app and how their backgrounds influence their attitudes on it. An Internet-based survey of foreign visitors and Japanese who had visited Tokyo conducted by the Economic and Social Research Institute is used for this study. 

For objective 1, we find that Safety Tips could be more popular and well-known in Indonesia, China, and Thailand than in the UK and Korea. Also, among those respondents that know Safety Tips or heard them before, their usage rate is lower than 70\%. Also, over 77\% represent a positive attitude toward Safety Tips among 4 attitude-related questions in the survey. Finally, we can conclude that respondents who know exactly and used Safety tips before show more positive attitudes towards Safety Tips than other groupings of people. 

For objective 2, we concluded that Disaster Prevention Consciousness has a negative impact on respondents' attitudes toward Safety Tips, while Knowledge and Perception on earthquakes has a positive impact. Then, Training Experience does not have a significant impact on respondents' attitudes toward Safety Tips.

For objective 3, we concluded that No-face-to-face information seeking is more common than face-to-face way, and following evacuation guidance behaviors are more common than self-evacuation behaviors.  And when the internet and telephone are available, Japanese more rely on No-face-to-face information-seeking behaviors, while foreign visitors will evacuate following guidance first, then rely on No-face-to-face information-seeking behaviors. On the contrary, if the internet and phone are unavailable, both Japanese and foreign visitors rely on evacuation following guidance. We also find that the response action during evacuation of Japanese and foreign visitors could be different, but there are no differences between foreign visitors based on nationality. 

\section{Suggestion for Safety Tips}
\begin{itemize}
\item People who have a more detailed awareness of Safety Tips are more likely to use this application, so if we want to increase the usage of Safety Tips, it would be helpful to increase foreign visitors' awareness of this application.
\item Foreign visitors are prone to seek non-face-to-face information, and Safety Tips is a platform for providing No-face-to-face information-seeking services. Therefore, when a disaster occurs, Safety Tips need to ensure that sufficient disaster information is available and comprehensive to all users.
\item Since foreign visitors will first follow the evacuation guidance, it is possible to cooperate with hotels, information centers, and other organizations that are capable of providing evacuation guidance. If they can remind foreign visitors that the Safety Tips application is a platform that can offer them the information they require during the evacuation guidance, the use of Safety Tips will rise, and it will be able to better assist foreign visitors.
\end{itemize}

\section{Limitations}
\begin{itemize}
\item The training experience manifest variables used in the training experience is not a scale question. If we can use scale questions to represent Training Experience, the SEM model will fit better.
\item Knowledge and Perception on earthquakes have less than three manifest variables, which makes Knowledge and Perception on earthquakes not well explained by manifest variables. In general, each latent variable should have three or more manifest variables.
\item Because the survey used in this study was not designed for this study, the SEM model could not be perfectly constructed. if a questionnaire could be designed based on the underlying assumptions, the fit would be improved.
\end{itemize}



















% A B S T R A C T
% ---------------

\begin{center}\textbf{Abstract}\end{center}

Before the COVID-19, foreign visitors to Japan were likely to increase year after year. Given that Japan is prone to earthquakes, many surveys show that it is extremely difficult for foreigners to seek information and evacuate with appropriate behaviors during previous disasters in Japan. In addition, given the government's ongoing focus on security and safety issues in the tourism industry, it is necessary to understand foreign visitors' behaviors during disasters. To assist foreign visitors in Japan, the Japan Tourism Agency has developed an application called Safety Tips, which can notify disaster information in 14 languages.

The purpose of this study is to better understand the information-seeking and evacuation behavior of foreign visitors to Japan, as well as to explore their behavior patterns when a disaster occurs. This study also looked at how foreign visitors perceive Safety Tips and how their backgrounds influence their perception on them. The primary data for this study was an internet-based web survey that included demographic questions, personal experiences, and knowledge, also respondents' information seeking and evacuation behaviors in the Tokyo Metropolitan Earthquake scenarios, and finally their perception of Safety Tips. 

First, this study examined the usage experience of all respondents and discovered that Safety Tips is more popular and well-known in Indonesia, China, and Thailand than in the U.K. and Korea. Among respondents who know about Safety Tips or have heard about it before, the usage rate is less than 70\%. We also figure out the differences among different nationalities and their different perception based on their experience of usage. Respondents who know exactly and used Safety tips before show more positive perceptions than other groupings of people. Finally, by the results of Chi-square and ANOVA, we can also indicate that age, number of visited countries, number of visited Japan, and Japanese level significantly affect the response of Perception on Safety Tips. Secondly, this study used Structural Equation Modeling to investigate how personal attributes influence people's perceptions on safety tips. As a result of the findings, Disaster Prevention Consciousness and Training Experience have significant negative effects on respondents' perceptions on Safety Tips, while Knowledge and Perception on earthquakes has a significant positive effect on respondents' perceptions on Safety Tips. What is more, this study also compared the differences between information-seeking and evacuation behaviors and showed that evacuation behaviors should be utilized more often than information-seeking actions. Evacuation behaviors have also been prioritized over information-seeking activities. Furthermore, No-face-to-face information-seeking activities should be utilized more frequently than Face-to-face information-seeking behaviors. In the top three activities, following evacuation advice behaviors should be used more than self-evacuation behaviors. Finally, this study attempts to apply the findings of the study to provide Safety Tips with some acceptable recommendations for future development. 


\cleardoublepage
%\newpage

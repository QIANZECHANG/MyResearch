% Some LaTeX commands I define for my own nomenclature.
% If you have to, it's better to change nomenclature once here than in a 
% million places throughout your thesis!



%======================================================================
\chapter{Conclusion}
\label{c7}
%======================================================================
\section{Conclusion}
This research focused on foreign visitors' information-seeking and evacuation behaviors, as well as their behavior patterns in the event of the “Tokyo Metropolitan Earthquake”. The research also investigates how foreign visitors to Japan perceive the Safety Tips app and how their backgrounds influence their perceptions on it. An Internet-based survey of foreign visitors and Japanese who had visited Tokyo conducted by the Economic and Social Research Institute is used for this study. 

For objective 1, we find that Safety Tips could be more popular and well-known in Indonesia, China, and Thailand than in the UK and Korea. Also, among those respondents that know Safety Tips or heard them before, their usage rate is lower than 70\%. Also, over 77\% represent a positive perception on Safety Tips among 4 perception-related questions in the survey. Finally, we can conclude that respondents who know exactly and used Safety tips before show more positive perceptions ons Safety Tips than other groupings of people. Finally, age, number of visited countries, number of visited Japan, and Japanese level significantly affect the response of Attitude on Safety Tips.

For objective 2, we explored how respondents' perceptions on Safety Tips are influenced by their characteristics. We concluded that Consciousness and Training Experience has negative effects on perceptions on Safety Tips, while Knowledge has a positive effect.

For objective 3, we concluded that No-face-to-face information seeking is more common than face-to-face way, and following evacuation guidance behaviors are more common than self-evacuation behaviors.  And when the internet and telephone are available, Japanese more rely on No-face-to-face information-seeking behaviors, while foreign visitors will evacuate following guidance first, then rely on No-face-to-face information-seeking behaviors. On the contrary, if the internet and phone are unavailable, both Japanese and foreign visitors rely on evacuation following guidance.  We concluded that the response action during evacuation of Japanese and foreign visitors could be different, especially the first response action. But there are no significant differences between foreign visitors based on nationality. 


\section{Suggestion for Safety Tips}
\begin{itemize}
\item People who have a more detailed awareness of Safety Tips are more likely to use this application, so if we want to increase the usage of Safety Tips, it would be helpful to increase foreign visitors' awareness of this application.
\item If Safety Tips can provide evacuation instructions, it will attract more people. Furthermore, if it can synchronize the user's location information, more people will use Safety Tips.
\item Foreign visitors are prone to seek non-face-to-face information, and Safety Tips is a platform for providing No-face-to-face information-seeking services. Therefore, when a disaster occurs, Safety Tips need to ensure that sufficient disaster information is available and comprehensive to all users.
\item Since foreign visitors will first follow the evacuation guidance, it is possible to cooperate with hotels, information centers, and other organizations that are capable of providing evacuation guidance. If they can remind foreign visitors that the Safety Tips application is a platform that can offer them the information they require during the evacuation guidance, the use of Safety Tips will rise, and it will be able to better assist foreign visitors.
\end{itemize}

\section{Limitations}
The training experience manifest variables used in the training experience is not a scale question. If we can use scale questions to represent Training Experience, the SEM model will fit better. In general, each latent variable should have three or more manifest variables. But in this study, the manifest variables used for Knowledge and Perception on earthquakes are less than three, which makes Knowledge and Perception on earthquakes not well explained by manifest variables. 



















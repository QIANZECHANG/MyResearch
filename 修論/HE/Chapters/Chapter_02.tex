% Some LaTeX commands I define for my own nomenclature.
% If you have to, it's better to change nomenclature once here than in a 
% million places throughout your thesis!


%======================================================================
\chapter{Literature Review}
%======================================================================

As people's awareness of disaster prevention has increased in recent years, the number of disaster prevention and evacuation studies has increased year after year. On the other hand, using statistical knowledge to analyze some sociological issues is also one of the popular research directions. This research is a hybrid of the two domains mentioned above. This study focuses on people's evacuation behavior in terms of disasters, to promote a link between the respondents' demographic information elements, prior experiences with travel/evacuation dramas/disasters, and their behavior patterns in disasters. On the other side, the researchers measured how the above characteristics influenced their perceptions of the disaster prevention application called Safety Tips. Therefore, the analysis methods of many sociology-oriented studies for questionnaires, especially those containing scale-based questions, were also referred to. The following are the primary papers cited in this study.
\begin{itemize}
  \item A Study on the Dissemination Way of Disaster Information to Foreign Tourists to Japan~\cite{ref8}. 
  \item Incorporating human factors in emergency evacuation – An overview of behavioral factors and models~\cite{ref9}.
  \item Impact of ride-hailing apps on traditional LAMAT services in Asian developing cities: The Phnom Penh Case~\cite{ref7}.
\end{itemize}

\section{A Study on the Dissemination Way of Disaster Information to Foreign Tourists to Japan}

This paper provides an overview of how visitors to Japan obtain information about natural disasters from tourist guides and other sources. Furthermore, it investigates how foreign visitors to Japan are effectively provided with information about natural disasters in Japan through the behavior of foreign tourists and the responses of government agencies and other administrative bodies during the 2016 Kumamoto earthquake and the 2018 Hokkaido earthquake.

While tourists can get some information on disaster preparedness from guidebooks and other sources, the paper mentions that foreign visitors who have never experienced a real earthquake in the past may feel anxious when a disaster occurs in an unfamiliar country. One point raised in the paper that had previously gone unnoticed was that, while the priority in a disaster is to prevent death or injury, the post-disaster care needs of foreign visitors differ from those of Japanese citizens. The most immediate post-disaster needs of most foreign visitors in the Kumamoto and Hokkaido earthquakes were a desire to leave the disaster area as soon as possible and an attempt to gather transportation information. This is also a revelation for this study, which is why the analysis was conducted in order to investigate the differences between Japanese and foreign visitors. Because of the differences in backgrounds, developing Safety Tips based on the habits and habits of Japanese people may not be effective in assisting foreign visitors. As a result of this thesis, we have made it a priority to investigate the differences in evacuation behavior between Japanese and foreign visitors.

In addition, the paper investigates how to provide timely and accurate information to foreign visitors to Japan in the immediate aftermath of a natural disaster, despite language barriers. The authors make three points: 1) it is critical to provide information to foreign visitors in the event of a disaster; 2) there is a need to propose a method that makes it easier to provide information to foreign visitors across language barriers; and 3) because foreign visitors' reactions to disasters are influenced by previous disaster experiences, there is a need to provide a detailed description of the disaster in the context of the visitors' own culture. Furthermore, because Japanese maps, kanji, addresses, and other symbols were difficult for foreign visitors to understand, it was difficult for foreign visitors to locate evacuation centers simply by translating the maps into English. Furthermore, even if they find a structure that appears to be an evacuation shelter, it is difficult for them to determine whether the structure is a safe place to evacuate. As a result, national languages and easily recognizable signs were required to indicate that the building was an "evacuation center." This would not only confirm that the shelter was open to foreigners but would also assist the Japanese in recognizing it as a location where foreigners could flee. These suggestions made by the authors are significant for the present study. This is because this study will eventually use the results of the analysis to make some recommendations that will be beneficial to the development of Safety Tips. Several of the ideas mentioned in the paper can be further refined to be reflected in Safety Tips in order to better assist foreign visitors.

\section{Incorporating human factors in emergency evacuation – An overview of behavioral factors and models}
This research focuses on the relationship between people's evacuation behaviors and a variety of factors. To establish the hypotheses used in this study, it is important to refer to the results of previous studies on the relationship between people's selected behavior in evacuation and various factors. Many previous papers have addressed the subject of whether and how various factors affect evacuation behavior in evacuation, as stated in this study. The authors of this thesis, however, state that while these studies cover a wide spectrum of threats and disasters as well as evacuee behavior, it lacks a general classification of the elements that influence evacuation behavior. As a result, the authors present a clearer framework in this research to describe how aspects related to evacuation interact. As a result, the authors present a clearer framework in this research that explains how factors related to evacuation behavior influence people's decision to evacuate. Personal, environmental, and intervention variables are the three types of factors. 

Individual factors were divided into two subcategories by the authors: static factors and social factors. In the disaster evacuation process, static factors are inherent and stable. Evacuation is closely related to socio-demographic factors like age, race, or gender, as well as socio-economic factors like education level or family characteristics. The paper also mentioned three other personal characteristics: hazard experience, knowledge, and ability/deficiency. Given that the questionnaire questions in this study included demographic factors such as age and gender, as well as individual factors such as hazardous experience, knowledge, and ability, I believe this paper is very useful in developing the hypotheses underlying this study. 

In addition to the above-mentioned individual factors, social characteristics are receiving increased attention due to their importance in shaping crowd behavior and group phenomena~\cite{ref10}. People are more likely to cooperate with others rather than act alone~\cite{ref11}, and situational altruism has been observed rather than mass panic~\cite{ref12}. This study mentions disaster awareness in Consciousness, which focuses on respondents' perceptions of disasters during the evacuation process, such as their sense of unease, Other-directed type, and Disastrous Imagination, among others. As a result, this component of the factors is also very useful in developing the hypothesis underlying this study. 
Sensory cues/external stimuli, hazard features, and built/engineered environment are the three categories of environmental factors. Sensory cues/external stimuli typically prompt evacuees to take specific actions during an evacuation, typically beginning with information-seeking behavior. The temporal and spatial characteristics of the hazard involved are referred to as hazard characteristics. Architectural/engineering environmental factors range from building layout and ancillary factors to area network configuration features. While this aspect of the factors will not be directly relevant to this study, this study needs to understand how external factors influence people's evacuation behavior and why people typically begin their evacuation with information-seeking behavior. Because Safety Tips is an information-provider application, it is advantageous that people need to search for information to develop Safety Tips. 
The final section introduces a new category of factors proposed by the assignment in this study, known as interventional factors. These are the elements that capture people's outside influences. This includes information and actions taken in response to a crisis, such as official information notification and guidance type actions. Although this part of the factors will not be directly related to this study, the available selections of information-seeking behaviors and evacuation behaviors in our questionnaire are related to official information and evacuation guidance, so this part is also useful in comprehending this study.

\section{Impact of ride-hailing apps on traditional LAMAT services in Asian developing cities: The Phnom Penh Case}

This paper explores the impact of ride-hailing apps (RHAs) on a traditional service like LAMAT drivers. As a case study, this paper analyzes survey data collected from 177 traditional auto-rickshaw drivers in Phnom Penh from December 11 to 14, 2018. By examining the proportional difference in their operational service before and after the development of RHAs, this study investigates the influence of RHAs on the operational service of Remork drivers who did not adopt RHAs. The proportional difference is determined as (Mean2-Mean1)/Mean1, and a negative value shows that Mean2 is decreasing proportionally in comparison to Mean1. The results of paired t-tests were also used to see if all of the variables were substantially different. 

This study also used structural equation modeling to see if the effects listed above would encourage drivers to use ride-hailing applications (RHAs). The study's underlying premise was developed by a previous investigation, and five hypotheses were to be investigated in the SEM. H1 hypothesis stated that RHA Impacts have a favorable influence on RHA Intention. The H2 hypothesis was that Appreciation of RHAs has a positive effect on Intention to Use RHAs. H3 hypothesis stated that RHA Appreciation has a favorable impact on RHA Impacts. H4 hypothesis was that Appreciation of RHAs has a positive effect on Government Support RHAs. H5 hypothesis is that government support for RHAs has a beneficial impact on RHA use intentions. In this study, the author assessed the SEM's goodness of fit by multiple indices: $\chi^2$, the Root Mean Square Error of Approximation (RMSEA), the Goodness-of-Fit Index (GFI), Adjusted GFI (AGFI), TuckerLewis Index (TLI), and Comparative Fit Index (CFI). The results of the study showed that RHAs had an impact on the operational services of Remork drivers who did not use RHAs. Because of the presence of RHA, their daily trips decreased by 47.7\%, their daily trip customers increased by 47.6\%, and their monthly revenue increased by 43.2\% ($p<0.01$). It was further found that traditional Remork drivers had no intention of adopting RHAs, despite the impact of their operational services by the emergence of RHAs. The results show that they are not interested in working with RHAs because they expect lower revenues. In addition, they did not show much interest in RHA given the lack of financial capacity to upgrade their vehicles and the fact that they already had potential customers. My study mainly refers to the application of Structural Equation Modeling and the research procedures of this paper.


